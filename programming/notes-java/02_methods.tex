\section{Methods and other concepts}

Now, we want to start discussing some important aspects of the language. In particular, this section 
deals with:
\begin{itemize}
    \item Expressions
    \item Statements
    \item Code block
    \item Methods
\end{itemize}

\subsection{Expressions, statements and code blocks}

Expressions are the building blocks of all java programs. We know very well what expressions mean.
But we need to understand a bit of terminology. First of all, suppose we have a code of the form   
\begin{listing}{1}
public class Main {    
    public static  void main(String[] args) {
        int age = 35;
    }
}
\end{listing}
We say that the block \verb|age = 35| is an expressions, whilst \verb|int age = 35;| is the 
statement. Code blocks are made by everything between the curly brackets. 

One important aspect of code blocks, is that the variable defined inside a code block are not 
available outside it. For example: 
\begin{listing}{1}
public class Main {    
    public static void main(String[] args) {
        int age = 35;
        boolean birthday = true;
        if (birthday == true) {
            int newAge = age + 1;
            System.out.println("Your new age is" + newAge);
        }
        /* Now we cannot access the variable newAge
           it is defined inside the block code */ 
    }
}
\end{listing}

\subsection{Methods}

Methods is a way to avoid duplication of the code. We have been using methods since day one: the main 
method. See the following example:
\begin{listing}{1}
public class Main {

    public static void main(String[] args) {
        boolean gameOver = true;
        int score = 800;
        int levelCompleted = 5;
        int bonus = 100;

        calculateScore(gameOver, score, levelCompleted, bonus);

        score = 10000;
        levelCompleted = 8;
        bonus = 200;

        calculateScore(gameOver, score, levelCompleted, bonus);


    }

    public static int calculateScore(boolean gameOver,int score, 
    int levelCompleted, int bonus) {

        if(gameOver) {
            int finalScore = score + (levelCompleted * bonus);
            finalScore += 2000;
            System.out.println("Your final score was " + finalScore);
            return finalScore;
        }

        return -1;

    }
}
\end{listing}
In some programming languages, the value \(-1\) denotes Error.

Another example: Calculate the taxes. 
\begin{listing}{1}
public class Methods {

    public static void main (String[] args) {
      double salary = 1000d;
      //taxes(salary);
      System.out.println("You pay " + taxes(salary));
    }

    public static double taxes(double salary) {
       if (salary < 3000) { 
           //System.out.println("You don't have to pay anything");
           return 0;} else if (salary >=3000 && salary <6000) {
           //System.out.println("You pay 15%: " + salary * 0.15);
           return salary * 0.15;} else {
           //System.out.println("You pay 27%: " + salary * 0.27);
           return salary * 0.27;
           }
    }
    
}
\end{listing}


\subsubsection{Method overloading}

Method overloading describes how the same function can be used to do different things depending on
the signature. The example that follows helps the understanding of thid concept. It converts inches
and feet to centimeters.
\begin{listing}{1}
public class Conversion {
    public static void main(String[] args) {
    System.out.println(calcFeetAndInchesToCentimeters(5,7));
    System.out.println(calcFeetAndInchesToCentimeters(3.8,-5.6));
    System.out.println(calcFeetAndInchesToCentimeters(-3.8,5.6));
    System.out.println(calcFeetAndInchesToCentimeters(3.8,56));
    System.out.println(calcFeetAndInchesToCentimeters(3.8));
    System.out.println(calcFeetAndInchesToCentimeters(56));
    System.out.println(calcFeetAndInchesToCentimeters(-3.8));
    }

    public static double calcFeetAndInchesToCentimeters(double feet, double inches) {
    double centimeters = -1; 
    if (feet >= 0 && inches >= 0 && inches <= 12) {
    centimeters = (12 * feet + inches) * 2.54;
    }
    return centimeters;
    }

    public static double calcFeetAndInchesToCentimeters(double inches) {
    double feet = -1; 
    if (inches >= 0) {
    feet = inches / 12;
    }
    return calcFeetAndInchesToCentimeters(feet, 0);
    }

}
\end{listing}
