\section{Control flow statements}

In this section we learn some important aspects of the language. In particular, we will learn
\begin{itemize}
    \item Switch
    \item For
    \item While 
    \item Do-while
\end{itemize}

\subsection{Switch}

The switch statement is an alternative to the if-else statement. It works as follows:
\begin{listing}{1}
public class Switch {
    public static void main(String[] args) {
   
    int number = 2;

    switch (number) {
        case 1:
            System.out.println("The number is one");
            break;

        case 2:
            System.out.println("The number is two");

        case 3: case 4: 
            System.out.println("The number is three or four");

        default:
            System.out.println("Another number");
            break;
    }

   }
}
\end{listing}
The \verb|case| is equivalent to the \verb|if| and \verb|else if| statements, and the \verb|default|
is equivalent to the \verb|else| statement. We also have the \verb|break|, and it stops the code block.
On the other hand, it has some limitations, in particular, in the \verb|if-else|, we can test different
variables, while in the \verb|switch| case we can only test one variable. Observe that we can combine
the cases in a single line -- case 3 and 4 above. 

Here are two nice examples:
\begin{listing}{1}
public class Numbers {
    public static void main(String[] args) {
    printNumberInWord(7);
    }

    public static void printNumberInWord(int number) {
    switch (number) {
        case 0:
            System.out.println("zero".toUpperCase());
            break;

        case 1:
            System.out.println("one".toUpperCase());
            break;

        case 2:
            System.out.println("two".toUpperCase());
            break;

        case 3:
            System.out.println("three".toUpperCase());
            break;

        case 4:
            System.out.println("four".toUpperCase());
            break;

        case 5:
            System.out.println("five".toUpperCase());
            break;

        case 6:
            System.out.println("six".toUpperCase());
            break;

        case 7:
            System.out.println("seven".toUpperCase());
            break;

        case 8:
            System.out.println("eight".toUpperCase());
            break;

        case 9:
            System.out.println("nine".toUpperCase());
            break;

        default:
            System.out.println("other".toUpperCase());
            break;
    }
    }

}
\end{listing}

The second example is:
\begin{listing}{1}
public class LeapYear {
    public static void main(String[] args) {
    System.out.println(isLeapYear(1986));
    System.out.println(getDaysInMonth(2,1600));
    System.out.println(getDaysInMonth(-2,2000));
    System.out.println(getDaysInMonth(12,21600));
    System.out.println(getDaysInMonth(2,1986));
    }

    public static boolean isLeapYear(int year) { 
    boolean result = false; 
    if (year < 1 || year > 9999) {
    result = false;
    } else {
    if ((year % 4 == 0) && (year % 100 != 0)) {
    result = true;
    } else if (year % 100 == 0 && year % 400 == 0) {
    result = true;
    } else {
    result = false;
    }
    }
    return result;
    }

    public static int getDaysInMonth(int month, int year) { 
    if (month < 1 || month >12 || year < 1 || year > 9999) { 
    return -1;
    }
    int days;
    boolean leap = isLeapYear(year);

    switch (month) {
        case 1: case 3: case 5: case 7: case 8: case 10: case 12:
            days = 31;
            break;

        case 4: case 6: case 9: case 11: 
            days = 30;
            break;

        default:
            if (leap) {
            days = 29;
            } else { 
            days = 28;
            }
            break;
    }
    return days; 
    }
}
\end{listing}

\subsection{For}

The syntax is standard. See the example:
\begin{listing}{1}
public class Interest {
    public static void main(String[] args) {
   // for(initial; termination; increment)
        /* 
    for(int i = 0; i <= 9; i++) { 
    System.out.println(i + 1 + 
    "% Interest of 1000000 is " + calculateInterest(1000000, i + 1));
    }
    */
    for(int i = 9; i >= 0; i--) { 
    System.out.println(i + 1 + 
    "% Interest of 1000000 is " + calculateInterest(1000000, i + 1));
    }
    }
    public static double calculateInterest(double amount, double rate) {    
    return (amount * (rate / 100));
    }
}
\end{listing}

Another example
\begin{listing}{1}
public class PrimeChecker {
   public static void main(String[] args) {
   for (int i = 15; i <= 30; i++) {
      System.out.println(i + " is prime: " + isPrime(i));
   }
   }

   public static boolean isPrime(int n) {
   int test = 0;
   if (n < 2) {
      return false;
   } else {
   for (int i = 2; i <= n / 2; i++) {
      if (n % i == 0) {
      test += 1;
      } 
   }
   if (test == 0) {
      return true;
   } else {
      return false;
   }
   }
   }
}
\end{listing}
 The code above is not the best. One can remember that the \verb|return| break the loop, 
 so we can improve the code. 

 Finally, see the example:
\begin{listing}{1}
public class SumOdd {
   public static void main(String[] args) {
   System.out.println(isOdd(7));
   System.out.println(sumOdd(1,100));
   System.out.println(sumOdd(-1,100));
   System.out.println(sumOdd(101,101));
   System.out.println(sumOdd(100,1000));
   }

   public static boolean isOdd(int number) {
      if (number % 2 == 1 && number != 0) {
         return true;
      } else {
      return false;
      }
   }

   public static int sumOdd(int start, int end) {
      int sum = 0;

      if (end < start || start < 0 || end < 0) {
      return -1;
      } else {

      for (int index = start; index <= end; index++) {
      if (isOdd(index)) {
      sum += index; 
      }
      }; 
      return sum;
      }
   }
}
\end{listing}


\subsection{While \& Do While}

The syntax is similar to what we have done so far. See the example:
\begin{listing}{1}
public class EvenChecker {
   public static void main(String[] args) {
   
      int start = 0;
      int end = 10;

//      while (start <= end) {
//      System.out.println("Is " + start + " even? " + isEvenNumber(start));
//      start++;
//      }

      do {
      System.out.println("Is " + start + " even? " + isEvenNumber(start));
      start++;
      } while (start <= end);
   }

   public static boolean isEvenNumber(int number) {
   if (number % 2 == 0) {
      return true;
   } else {
   return false;
   }
   }

}
\end{listing}
The part with and without comments are equivalent.

\subsection{Parsing values from a string}

Sometimes, we need to convert a string to a different data type. We can do it 
with the parse method as follows:
\begin{listing}{1}
public class ParseString {
   public static void main(String[] args) {
   String numberAsString = "2000";
   System.out.println(numberAsString);

   int number = Integer.parseInt(numberAsString);
   System.out.println(number);

   number += 1;
   numberAsString += 1;
   System.out.println(number);
   System.out.println(numberAsString);
   }

}
\end{listing}
In this example, observe that if we try to add a number to a string, java automatically converts
the number to a string.


\subsection{User input}

In order to provide the user the ability to insert values, we use the method \verb|scanner|. See
\begin{listing}{1}
import java.util.Scanner;

public class UserInput {

   public static void main(String[] args) {

   Scanner scanner = new Scanner(System.in);

   System.out.println("Enter your year of birth: ");
   int yearOfBirth = scanner.nextInt();
   scanner.nextLine(); // it handles next line character
   int age = 2022 - yearOfBirth;

   System.out.println("Enter your name: ");
   String name = scanner.nextLine();

   System.out.println("Your name is " + name + " and you're " + age + " years old");

   scanner.close();
   }
}
\end{listing}
The keyword \verb|new| creates a new instance of scanner.

In order to avoid negative numbers and words instead numbers, one can do the following
\begin{listing}{1}
import java.util.Scanner;

public class Main {

    public static void main(String[] args) {
        Scanner scanner = new Scanner(System.in);

        System.out.println("Enter your year of birth:");

        boolean hasNextInt = scanner.hasNextInt();

        if(hasNextInt) {
            int yearOfBirth = scanner.nextInt();
            scanner.nextLine(); // handle next line character (enter key)

            System.out.println("Enter your name: ");
            String name = scanner.nextLine();
            int age = 2018 - yearOfBirth;

            if(age >= 0 && age <= 100) {
                System.out.println("Your name is " + name + ", and you are " + age + " years old.");
            } else {
                System.out.println("Invalid year of birth");
            }
        } else {
            System.out.println("Unable to parse year of birth.");
        }

        scanner.close();
    }
}begin{listing}{1}
\end{listing}

The next example has all the important features of our discussion.
\begin{listing}{1}
import java.util.Scanner;

public class InputSum {
   public static void main(String[] args) {
      
      int sum = 0;
      int counter = 1;

      // Next line defines the variable scanner
      Scanner scanner = new Scanner(System.in);

      System.out.println("Enter 10 integers");

      while (counter <= 10) {
      
      System.out.println("Enter number #" + counter + ":");

      // Next line, we define the variable that tests if the variable is an integer
      boolean isInteger = scanner.hasNextInt();

      if (isInteger) {

         // read the user number 
         int number = scanner.nextInt();
         sum += number;
         counter++;
         } else {
            System.out.println("Invalid Number");
         }

      // Here we call the scanner again. In the next loop, the user inserts another number
      scanner.nextLine();
      }

      System.out.println("The total is: " + sum);

      scanner.close();

   }
}
\end{listing}
