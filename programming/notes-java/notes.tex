\documentclass[a4paper,11pt]{amsart}

\pdfoutput=1 

% ========================================================================

\usepackage{/Users/thiago/.tex_templates/ams_math/definitions}

\hypersetup{
	pdftitle={integrable systems, programming and so on},
	pdfsubject={High Energy Physics, Python and so on},
	pdfauthor={Thiago Araujo},
	pdfkeywords={gauge; susy; strings; fields; cft; python},
	colorlinks=true,linkcolor=link,citecolor=link,urlcolor=link,linktocpage
}


\begin{document}

% ========================================================================
% BEGIN COVER: title, author, affiliation, abstract


\title[Notes on Java]{Notes on Java}

\author{Thiago Araujo}

%\address{\noindent AFFILIATION}
\email{\texttt{\href{thgr.araujo@gmail.com}{thgr.araujo@gmail.com}}} 

%\keywords{Temperley-Lieb, XXZ spin chain, KP, matrix integrals}
% \subjclass[2020]{37K10, 82B20, 82B23}
%\date{\today}

\begin{abstract}
Here I collect some basic facts about Java. Here I follow the course of Tim Buchalka and notes of 
the W3 School.

\bigskip

\noindent \textbf{Keywords:} Java 
\end{abstract}

\maketitle

\setcounter{tocdepth}{1}
\tableofcontents


% END COVER 
% ========================================================================



% ========================================================================
\section*{Introduction}
% ========================================================================

In these notes, I collect some basic facts about Java.
Here I follow the course of Tim Buchalka and notes of 
the W3 School.
% ========================================================================

\section{Fundamental aspects of the language}

We need to understand the basics of the language, we first write the \verb|file.java|
and run the command 
\begin{listing}{1}
javac file.java
\end{listing}
With this command, the file is compiled to a \verb|file.class| file, and we can now run
\begin{listing}{1}
java file     
\end{listing}    
We do not need to specify the extension \verb|.class|, since it is the only extension the compiler 
can run.

\subsection{Hello World}

See the source file (\verb|HelloWorld.java|), we have the following structure
\begin{listing}{1}
public class HelloWorld {
    /* Principal method 
     * of the class */
    public static void main (String[] args){
        System.out.println("Hello World!");
        System.out.println("This is a second sentence");
    }// end of the method
}// end of the class
\end{listing}
So, we define a class and that it is public. In what follows we define the methods associated to 
this class. It defines the type of return, name of the method and the type of arguments it accepts. 

The import point is that we have some keywords above. For example
\begin{itemize}
    \item \verb|public| is an access modifier. It specify the scope of our code. How other parts 
of the program can access the code.
    \item \verb|class| defines the class block. Anything between the brackets \verb|{ }| is part 
of the class.
    \item \verb|HelloWorld| is the name of the class. It needs to have the name of the \verb|.java|
file. 
\end{itemize}
Between the brackets we have the methods. Methods are collections of statements that perform 
operations. In our simple code, we also have the \verb|main| method. It is a special method that 
java looks 
when we run a program. 
\begin{itemize}
    \item \verb|static| is a keyword we will discuss in the future. 
    \item \verb|void| is a keyword that denotes that the method will nor return anything.    
\end{itemize}

\subsubsection{Keywords}

There are some keywords that are protected in java. There is a wikipedia entry for these keywords: 
\href{https://en.wikipedia.org/wiki/List_of_Java_keywords}{Java Keywords}.

\subsubsection{Comments}

There are different ways to add comments. For example, we can have comments in single lines. We can
also have comments spanning multiple lines. These are
\begin{listing}{1}
// These are single line comment

/* This is a 
multiple line comment */ 
\end{listing}

\subsection{Variables}

It is a way to store information in our computer. For example, let us define a variable:
\begin{listing}{1}
public class HelloWorld {
    /* Principal method 
     * of the class */
    public static void main (String[] args){
        System.out.println("Hello World!");
        System.out.println("This is a second sentence");

        int my_first_number = 5;
        System.out.println(my_first_number);
    }// end of the method
}// end of the class
\end{listing}

\subsection{Primitive types}

There are different types of data. So far we have been working with integers, but there are seven 
other primitive, the building blocks of java coding, data types. The primitive types are
\begin{enumerate}
    \item byte
    \item short
    \item int
    \item long
    \item float
    \item double
    \item char
    \item boolean
\end{enumerate}
Besides the course, see also~\cite{w3school}.

\subsubsection{Byte, short, int and long}

Let us consider the first four primitive types. The byte type has 8 bits, the short has 16 bits, 
the integer has 32 bits and the long has 64 bits. We can test the size of these data types with 
the following code:
\begin{listing}{1}
public class Primitive {
    public static void main(String[] args) {
        int myValue = 10000;
        int myMinValue = Integer.MIN_VALUE;
        int myMaxValue = Integer.MAX_VALUE;
        System.out.println("Integer minimum value =" + myMinValue);
        System.out.println("Integer maximum value =" + myMaxValue);

        byte myMinByteValue = Byte.MIN_VALUE;
        byte myMaxByteValue = Byte.MAX_VALUE;
        System.out.println("Byte minimum value =" + myMinByteValue);
        System.out.println("Byte maximum value =" + myMaxByteValue);

        short myMinShortValue = Short.MIN_VALUE;
        short myMaxShortValue = Short.MAX_VALUE;
        System.out.println("Short minimum value =" + myMinShortValue);
        System.out.println("Short maximum value =" + myMaxShortValue);

        long myMinLongValue = Long.MIN_VALUE;
        long myMaxLongValue = Long.MAX_VALUE;
        System.out.println("Long minimum value =" + myMinLongValue);
        System.out.println("Long maximum value =" + myMaxLongValue);
    }
}
\end{listing}
It is important to pay attention to these definitions, otherwise we can find errors in our code. 
In general, java assumes that we are using integers in our calculations. 
\paragraph{Casting} Furthermore, sometimes we need to convert types of numbers. For example, suppose we take the 
minimum byte number and divide by two, it certainly fits in the integers, but java does not know that
so we need to tell him about it. This is the concept of \emph{casting}. For example 
\begin{listing}{1}
byte newByteValue = (byte) (myMinByteValue / 2);
\end{listing}

\subsubsection{Floads and doubles}

Now we want to consider two more primitive data types. These deal with decimal values. Floats have 
single precision (so they occupy 32 bits) while Double have double precision (64 bits). 
It is important to know that double is the default data type for decimal numbers.

We can consider the example (we can put the code below inside a class, see the codes) 
\begin{listing}{1}
      float myMinFloatValue = Float.MIN_VALUE;
      float myMaxFloatValue = Float.MAX_VALUE;
      System.out.println("Float minimum value =" + myMinFloatValue);
      System.out.println("Float maximum value =" + myMaxFloatValue);

      double myMinDoubleValue = Double.MIN_VALUE;
      double myMaxDoubleValue = Double.MAX_VALUE;
      System.out.println("Double minimum value =" + myMinDoubleValue);
      System.out.println("Double maximimum value =" + myMaxDoubleValue);

      int myIntValue = 5;
      float myFloatValue = (float) 5.25; // Equivalent to 5.25f
      double myDoubleValue = 5.25; //We can also write 5.25d
\end{listing}
We can also test precision of these data structures as 
\begin{listing}{1}
      float myMinFloatValue = Float.MIN_VALUE;
      float myMaxFloatValue = Float.MAX_VALUE;
      System.out.println("Float minimum value =" + myMinFloatValue);
      System.out.println("Float maximum value =" + myMaxFloatValue);

      double myMinDoubleValue = Double.MIN_VALUE;
      double myMaxDoubleValue = Double.MAX_VALUE;
      System.out.println("Double minimum value =" + myMinDoubleValue);
      System.out.println("Double maximimum value =" + myMaxDoubleValue);

      int myIntValue = 5 / 3;
      float myFloatValue = 5f / 3f;
      double myDoubleValue = 5d / 3d;

      System.out.println("MyIntValue = " + myIntValue);
      System.out.println("MyFloatValie = " + myFloatValue);
      System.out.println("MyDoubleValue = " + myDoubleValue);
\end{listing}

We can use float and doubles to do most of our programs, but for precise calculations, we can use
BigDecimal numbers, that is more precise. We shall eventually return to this point. 


\subsubsection{Char and Boolean}

Now we consider the last two types of data. The \verb|char| type stores a single character. It 
occupies 16 bits, since it allows us to store data in Unicode. \verb|Boolean|, these are True or 
False, Yes ou No, 1 or 2, as usual.


\subsubsection{Strings}

Finally, we also have \verb|strings|. It is not a primitive data, it is a class; but enjoys a favoritism in java, 
so we can work with it more easily. It is essentially a sequence of characters.
One funny thing is the following, we can concatenate strings and numbers, because java converts the 
appropriate numbers to strings. On the other hand, it does not do the converse, that is, java does 
not translate (at least immediately), strings into numbers. For example 
\begin{listing}{1}
    String myString = "10";
    int myInt = 50;
    String lastString = myString + myInt;
    int lastInt = myString + myInt // it gives an error
    System.out.println("MyString is equal to " + lastString)
\end{listing}

In java, strings are immutable. So, we cannot change the string after it was created. When we 
replace the string, as we have done above, it basically creates a new string and discarted.

Moreover, the method of appeding strings as we did above is inneficient.We will learn better ways
in the future, in particular something called \verb|StringBuffer|.

\subsection{Operators, operands and expressions}

Operators are special symbols to do specifics operations. Operands are the objects subject to the 
operations. Expressions are formed with variables, literals and methods. 
We have the following operators, see~\cite{w3school}:
\begin{enumerate}
    \item Arithmetic: \verb|+|, \verb|-|, \verb|*|, \verb|/|, \verb|%|, \verb|++|, \verb|--|
    \item Assignment: \verb|=|, \verb|+=|, \verb|-=|, \verb|*=|, \verb|/=|, \verb|%=|
    \item Comparison: \verb|==|, \verb|!=|, \verb|>|, \verb|<|, \verb|>=|, \verb|>=|
    \item Logical: \verb|&&| (and), \(\Vert\) (or), \verb|!| (not)
    \item Bitwise: \verb|>>|, \verb|<<|, \verb|&|, \verb|>>>|, \verb|^|,\(\vert\)
\end{enumerate}
The arithmetic and assignement operators are somewhat straigtforward. 
Let us consider now the comparisson operators.

A summary of these operators are here\cite{oracle}. See also a table of 
operator precedence~\cite{precedence}.

\subsubsection{Conditional statements and logical operators}

In order to consider the condisional operators, we need to understand some conditionals in java. 

\paragraph{If} The easiest condisionals are given by the \verb|if| and \verb|else| 
statements. In particular, we follow the example
\begin{listing}{1}
public class Operators {
    public static void main(String[] args) {
        boolean isAlien = false;
        if (isAlien == false) {
            System.out.println("It is not an alien");
            System.out.println("and I am scared of aliens");
        }
        // Conditional if and logical "and" and "or" operators.
        int topScore = 80;
        if (topScore == 100) {
        System.out.println("Yout got the highest score"); 
        }
        int secondTopScore = 90;
        if ((topScore > secondTopScore) && (topScore <100)) {
            System.out.println("you got the second highest score, 
            and it was smaller than 100");
        }
        if ((topScore < 90) || (secondTopScore > 90)) {
            System.out.println("One or both conditions are true");
        }
        boolean isTired = false;
        if (!isTired) {
            System.out.println("This is a final conditional");
        }
    }
}
\end{listing}
In this example, we have instances of the logical operators ``and'' (\verb|&&|), 
``or'' (\(\Vert\)) and ``not'' (\verb|!|). Observe that in the test of ``not'', we can write
\verb|!isTired| or \verb|isTired == False| if we want to test if it is False, or 
\verb|isTired| or \verb|isTired == True| if we want to test if it is True. 

\paragraph{Else} We can also exceptions for the if blocks. For 
example:
\begin{listing}{1}
public class Else {
    public static void main(String[] args) {
        int age = 35;
        if (age > 40) {
         System.out.println("You are not young any longer"); 
        } else if (age < 40 && age < 35){
        System.out.println("You do not have much time");
        } else {
            System.out.println("You're okay");
        }
    }
}
\end{listing}




\subsubsection{Ternary operators}

The ternary operator is defined by the question mark (\verb|?|). The syntax is the following:
\begin{verbatim}
variable = Expression ? Output1 : Output2
\end{verbatim}
And the idea is that it tests if the \verb|Expression| is True. In this case, variable assumes the value \verb|Output1|, if the \verb|Expression| is False, the variable assumes the value \verb|Output2|.
\begin{listing}{1}
public class Operators {
    public static void main(String[] args) {
        boolean isCar = false;
        boolean wasCar = isCar ? true : false;
        if (wasCar) {
            System.out.println("This is true");
        }
        // Below, we can also write ... = (!isCar == True) ? ...
        int outputTrue = !isCar ? 1 : 2;
        System.out.println("The output is " + outputTrue);
        int outputFalse = isCar ? 1 : 2;
        System.out.println("The output is " + outputFalse);
    }
}
\end{listing}


\subsubsection{Challenge}

Here is the proposed challenge:
\begin{listing}{1}
public class Operators {
    public static void main(String[] args) {
//        ========================= CHALLENGE ====================
//
//        The idea is to do the following:
//        1) Create two doubles with values 20.00 and 80.00 
//        2) Add both number together and multiply by 100.00
//        3) Find the remainder of this last operation by 40.00
//        4) Create a boolean which: 
//              a) is True if the remainder is zero
//              b) is False if the remainder is not zero
//        5) Write an if-else statement that displays a message
//        if the boolean of #4 not true
//
//
        double variable1 = 20d;
        double variable2 = 20d;
        double variable3 = (variable1 + variable2) * 100d;
        System.out.println(variable3);
        double remainder = (double) (variable3 % 40);
        System.out.println("The remainder is " + remainder);
        boolean teste = (remainder == 0) ? true : false;
        if (!teste) {
            System.out.println("You got some remainder.");
        }
    }
}
\end{listing}


\section{Methods and other concepts}

Now, we want to start discussing some important aspects of the language. In particular, this section 
deals with:
\begin{itemize}
    \item Expressions
    \item Statements
    \item Code block
    \item Methods
\end{itemize}

\subsection{Expressions, statements and code blocks}

Expressions are the building blocks of all java programs. We know very well what expressions mean.
But we need to understand a bit of terminology. First of all, suppose we have a code of the form   
\begin{listing}{1}
public class Main {    
    public static  void main(String[] args) {
        int age = 35;
    }
}
\end{listing}
We say that the block \verb|age = 35| is an expressions, whilst \verb|int age = 35;| is the 
statement. Code blocks are made by everything between the curly brackets. 

One important aspect of code blocks, is that the variable defined inside a code block are not 
available outside it. For example: 
\begin{listing}{1}
public class Main {    
    public static void main(String[] args) {
        int age = 35;
        boolean birthday = true;
        if (birthday == true) {
            int newAge = age + 1;
            System.out.println("Your new age is" + newAge);
        }
        /* Now we cannot access the variable newAge
           it is defined inside the block code */ 
    }
}
\end{listing}

\subsection{Methods}

Methods is a way to avoid duplication of the code. We have been using methods since day one: the main 
method. See the following example:
\begin{listing}{1}
public class Main {

    public static void main(String[] args) {
        boolean gameOver = true;
        int score = 800;
        int levelCompleted = 5;
        int bonus = 100;

        calculateScore(gameOver, score, levelCompleted, bonus);

        score = 10000;
        levelCompleted = 8;
        bonus = 200;

        calculateScore(gameOver, score, levelCompleted, bonus);


    }

    public static int calculateScore(boolean gameOver,int score, 
    int levelCompleted, int bonus) {

        if(gameOver) {
            int finalScore = score + (levelCompleted * bonus);
            finalScore += 2000;
            System.out.println("Your final score was " + finalScore);
            return finalScore;
        }

        return -1;

    }
}
\end{listing}
In some programming languages, the value \(-1\) denotes Error.

Another example: Calculate the taxes. 
\begin{listing}{1}
public class Methods {

    public static void main (String[] args) {
      double salary = 1000d;
      //taxes(salary);
      System.out.println("You pay " + taxes(salary));
    }

    public static double taxes(double salary) {
       if (salary < 3000) { 
           //System.out.println("You don't have to pay anything");
           return 0;} else if (salary >=3000 && salary <6000) {
           //System.out.println("You pay 15%: " + salary * 0.15);
           return salary * 0.15;} else {
           //System.out.println("You pay 27%: " + salary * 0.27);
           return salary * 0.27;
           }
    }
    
}
\end{listing}


\subsubsection{Method overloading}

Method overloading describes how the same function can be used to do different things depending on
the signature. The example that follows helps the understanding of thid concept. It converts inches
and feet to centimeters.
\begin{listing}{1}
public class Conversion {
    public static void main(String[] args) {
    System.out.println(calcFeetAndInchesToCentimeters(5,7));
    System.out.println(calcFeetAndInchesToCentimeters(3.8,-5.6));
    System.out.println(calcFeetAndInchesToCentimeters(-3.8,5.6));
    System.out.println(calcFeetAndInchesToCentimeters(3.8,56));
    System.out.println(calcFeetAndInchesToCentimeters(3.8));
    System.out.println(calcFeetAndInchesToCentimeters(56));
    System.out.println(calcFeetAndInchesToCentimeters(-3.8));
    }

    public static double calcFeetAndInchesToCentimeters(double feet, double inches) {
    double centimeters = -1; 
    if (feet >= 0 && inches >= 0 && inches <= 12) {
    centimeters = (12 * feet + inches) * 2.54;
    }
    return centimeters;
    }

    public static double calcFeetAndInchesToCentimeters(double inches) {
    double feet = -1; 
    if (inches >= 0) {
    feet = inches / 12;
    }
    return calcFeetAndInchesToCentimeters(feet, 0);
    }

}
\end{listing}


\section{Control flow statements}

In this section we learn some important aspects of the language. In particular, we will learn
\begin{itemize}
    \item Switch
    \item For
    \item While 
    \item Do-while
\end{itemize}

\subsection{Switch}

The switch statement is an alternative to the if-else statement. It works as follows:
\begin{listing}{1}
public class Switch {
    public static void main(String[] args) {
   
    int number = 2;

    switch (number) {
        case 1:
            System.out.println("The number is one");
            break;

        case 2:
            System.out.println("The number is two");

        case 3: case 4: 
            System.out.println("The number is three or four");

        default:
            System.out.println("Another number");
            break;
    }

   }
}
\end{listing}
The \verb|case| is equivalent to the \verb|if| and \verb|else if| statements, and the \verb|default|
is equivalent to the \verb|else| statement. We also have the \verb|break|, and it stops the code block.
On the other hand, it has some limitations, in particular, in the \verb|if-else|, we can test different
variables, while in the \verb|switch| case we can only test one variable. Observe that we can combine
the cases in a single line -- case 3 and 4 above. 

Here are two nice examples:
\begin{listing}{1}
public class Numbers {
    public static void main(String[] args) {
    printNumberInWord(7);
    }

    public static void printNumberInWord(int number) {
    switch (number) {
        case 0:
            System.out.println("zero".toUpperCase());
            break;

        case 1:
            System.out.println("one".toUpperCase());
            break;

        case 2:
            System.out.println("two".toUpperCase());
            break;

        case 3:
            System.out.println("three".toUpperCase());
            break;

        case 4:
            System.out.println("four".toUpperCase());
            break;

        case 5:
            System.out.println("five".toUpperCase());
            break;

        case 6:
            System.out.println("six".toUpperCase());
            break;

        case 7:
            System.out.println("seven".toUpperCase());
            break;

        case 8:
            System.out.println("eight".toUpperCase());
            break;

        case 9:
            System.out.println("nine".toUpperCase());
            break;

        default:
            System.out.println("other".toUpperCase());
            break;
    }
    }

}
\end{listing}

The second example is:
\begin{listing}{1}
public class LeapYear {
    public static void main(String[] args) {
    System.out.println(isLeapYear(1986));
    System.out.println(getDaysInMonth(2,1600));
    System.out.println(getDaysInMonth(-2,2000));
    System.out.println(getDaysInMonth(12,21600));
    System.out.println(getDaysInMonth(2,1986));
    }

    public static boolean isLeapYear(int year) { 
    boolean result = false; 
    if (year < 1 || year > 9999) {
    result = false;
    } else {
    if ((year % 4 == 0) && (year % 100 != 0)) {
    result = true;
    } else if (year % 100 == 0 && year % 400 == 0) {
    result = true;
    } else {
    result = false;
    }
    }
    return result;
    }

    public static int getDaysInMonth(int month, int year) { 
    if (month < 1 || month >12 || year < 1 || year > 9999) { 
    return -1;
    }
    int days;
    boolean leap = isLeapYear(year);

    switch (month) {
        case 1: case 3: case 5: case 7: case 8: case 10: case 12:
            days = 31;
            break;

        case 4: case 6: case 9: case 11: 
            days = 30;
            break;

        default:
            if (leap) {
            days = 29;
            } else { 
            days = 28;
            }
            break;
    }
    return days; 
    }
}
\end{listing}

\subsection{For}

The syntax is standard. See the example:
\begin{listing}{1}
public class Interest {
    public static void main(String[] args) {
   // for(initial; termination; increment)
        /* 
    for(int i = 0; i <= 9; i++) { 
    System.out.println(i + 1 + 
    "% Interest of 1000000 is " + calculateInterest(1000000, i + 1));
    }
    */
    for(int i = 9; i >= 0; i--) { 
    System.out.println(i + 1 + 
    "% Interest of 1000000 is " + calculateInterest(1000000, i + 1));
    }
    }
    public static double calculateInterest(double amount, double rate) {    
    return (amount * (rate / 100));
    }
}
\end{listing}

Another example
\begin{listing}{1}
public class PrimeChecker {
   public static void main(String[] args) {
   for (int i = 15; i <= 30; i++) {
      System.out.println(i + " is prime: " + isPrime(i));
   }
   }

   public static boolean isPrime(int n) {
   int test = 0;
   if (n < 2) {
      return false;
   } else {
   for (int i = 2; i <= n / 2; i++) {
      if (n % i == 0) {
      test += 1;
      } 
   }
   if (test == 0) {
      return true;
   } else {
      return false;
   }
   }
   }
}
\end{listing}
 The code above is not the best. One can remember that the \verb|return| break the loop, 
 so we can improve the code. 

 Finally, see the example:
\begin{listing}{1}
public class SumOdd {
   public static void main(String[] args) {
   System.out.println(isOdd(7));
   System.out.println(sumOdd(1,100));
   System.out.println(sumOdd(-1,100));
   System.out.println(sumOdd(101,101));
   System.out.println(sumOdd(100,1000));
   }

   public static boolean isOdd(int number) {
      if (number % 2 == 1 && number != 0) {
         return true;
      } else {
      return false;
      }
   }

   public static int sumOdd(int start, int end) {
      int sum = 0;

      if (end < start || start < 0 || end < 0) {
      return -1;
      } else {

      for (int index = start; index <= end; index++) {
      if (isOdd(index)) {
      sum += index; 
      }
      }; 
      return sum;
      }
   }
}
\end{listing}


\subsection{While \& Do While}

The syntax is similar to what we have done so far. See the example:
\begin{listing}{1}
public class EvenChecker {
   public static void main(String[] args) {
   
      int start = 0;
      int end = 10;

//      while (start <= end) {
//      System.out.println("Is " + start + " even? " + isEvenNumber(start));
//      start++;
//      }

      do {
      System.out.println("Is " + start + " even? " + isEvenNumber(start));
      start++;
      } while (start <= end);
   }

   public static boolean isEvenNumber(int number) {
   if (number % 2 == 0) {
      return true;
   } else {
   return false;
   }
   }

}
\end{listing}
The part with and without comments are equivalent.

\subsection{Parsing values from a string}

Sometimes, we need to convert a string to a different data type. We can do it 
with the parse method as follows:
\begin{listing}{1}
public class ParseString {
   public static void main(String[] args) {
   String numberAsString = "2000";
   System.out.println(numberAsString);

   int number = Integer.parseInt(numberAsString);
   System.out.println(number);

   number += 1;
   numberAsString += 1;
   System.out.println(number);
   System.out.println(numberAsString);
   }

}
\end{listing}
In this example, observe that if we try to add a number to a string, java automatically converts
the number to a string.


\subsection{User input}

In order to provide the user the ability to insert values, we use the method \verb|scanner|. See
\begin{listing}{1}
import java.util.Scanner;

public class UserInput {

   public static void main(String[] args) {

   Scanner scanner = new Scanner(System.in);

   System.out.println("Enter your year of birth: ");
   int yearOfBirth = scanner.nextInt();
   scanner.nextLine(); // it handles next line character
   int age = 2022 - yearOfBirth;

   System.out.println("Enter your name: ");
   String name = scanner.nextLine();

   System.out.println("Your name is " + name + " and you're " + age + " years old");

   scanner.close();
   }
}
\end{listing}
The keyword \verb|new| creates a new instance of scanner.

In order to avoid negative numbers and words instead numbers, one can do the following
\begin{listing}{1}
import java.util.Scanner;

public class Main {

    public static void main(String[] args) {
        Scanner scanner = new Scanner(System.in);

        System.out.println("Enter your year of birth:");

        boolean hasNextInt = scanner.hasNextInt();

        if(hasNextInt) {
            int yearOfBirth = scanner.nextInt();
            scanner.nextLine(); // handle next line character (enter key)

            System.out.println("Enter your name: ");
            String name = scanner.nextLine();
            int age = 2018 - yearOfBirth;

            if(age >= 0 && age <= 100) {
                System.out.println("Your name is " + name + ", and you are " + age + " years old.");
            } else {
                System.out.println("Invalid year of birth");
            }
        } else {
            System.out.println("Unable to parse year of birth.");
        }

        scanner.close();
    }
}begin{listing}{1}
\end{listing}

The next example has all the important features of our discussion.
\begin{listing}{1}
import java.util.Scanner;

public class InputSum {
   public static void main(String[] args) {
      
      int sum = 0;
      int counter = 1;

      // Next line defines the variable scanner
      Scanner scanner = new Scanner(System.in);

      System.out.println("Enter 10 integers");

      while (counter <= 10) {
      
      System.out.println("Enter number #" + counter + ":");

      // Next line, we define the variable that tests if the variable is an integer
      boolean isInteger = scanner.hasNextInt();

      if (isInteger) {

         // read the user number 
         int number = scanner.nextInt();
         sum += number;
         counter++;
         } else {
            System.out.println("Invalid Number");
         }

      // Here we call the scanner again. In the next loop, the user inserts another number
      scanner.nextLine();
      }

      System.out.println("The total is: " + sum);

      scanner.close();

   }
}
\end{listing}


%%%%%%%%%%%%%%%%%%%%%%%%%%%%%%%%%%%%%%%%
%%%%%%%%%%%%%%%%%%%%%%%%%%%%%%%%%%%%%%%%
%%%%%%%%%%%%%%%%%%%%%%%%%%%%%%%%%%%%%%%%
%%%%%%%%%%%%%%%%%%%%%%%%%%%%%%%%%%%%%%%%

\section{Object Oriented Programming}

Object Oriented Programming (OOP) is about creating representation of things in the real world using codes. We do it using {\bf classes} and {\bf objects}.

\paragraph{\(\#\) \bf class} Blueprints for objects. Classes can contain methods (functions) and attributes (similar to keys in a dictionary). 

\paragraph{ \(\#\) \bf instance} - objects that are constructed from a class blueprint that contain their class's methods and properties.

\subsection{Definitions}

With OOP we can classify and organize our codes. 

\subsubsection{Example} Suppose we want to make a poker game. We could have the following entities: \(\bullet\) Game, \(\bullet\) Player, \(\bullet\) Card, \(\bullet\) Deck, \(\bullet\) Hand, \(\bullet\) Chip, \(\bullet\) Bet and so on. We could hard code all these entities, but it is oftentimes convenient to define classes and work with methods associated to these classes. 

I can consider a different explanation, aimed for mathematicians. Suppose I want to work with matrices. We could hard code these objects and hard code all operations associated to them, but it is better to define the class of matrices, and methods associated to them. These methods can be the transposition, trace and so on. 

\subsubsection{Defining a class}

Suppose we want to create a game. We want to build a class for users. The syntax is the following:
\begin{verbatim}
> class User: # Classes are oftentimes capitalized
>    def __init__(self, first, last, age):
>        self.name = first
>        self.last = last
>        self.age = age
\end{verbatim} 

We always use the syntax \verb|__init__| in the definition of the class. The \verb|self| is a dummy variable, but it is traditionally written as \verb|self| and it reffers to the objects (instances) themselves. Python calls the \verb|__init__| method whenever we create an instance of class (instatiate).

Now we can build these users using the methods, that is 
\begin{verbatim}
> user1 = User('Joe', 'Smith', 68)
> user2 = User('Blanka', 'Lopez', 41)
\end{verbatim}

And finally, we can retried the information of the users using the methods associated to them. For example, the code
\begin{verbatim}
> print(user1.first, user1.last)
> print(user2.first, user2.last)
\end{verbatim}
prints the names. Observe that the method does not use the brackets \verb|()|.

Another example is the following. Consider the class \verb|Comments| in a social network. That is 
\begin{verbatim}
> class Comments:
>      def __init__(self, username, text, likes=0) # likes default value is 0. 
>      self.username = username	
>      self.text = text
>      self.likes = likes
\end{verbatim}
Then, we run this code as
\begin{verbatim}
> c = Comment("davey123", 'lol you\'re so silly', 3)
> print(c.username)
> print(c.text)
> print(c.likes)
\end{verbatim}

\begin{shaded}
Here I need to comment something important: The underscores have meaning. In particular, we have things in the form: 
\begin{itemize}
	\item \verb|_name| : just one underscore in front of a method is a message to other developers. It says that this method whould be private, although python does not have full fledged secrete methods. It is basically a method to be used inside the definition of the class. 
	\item \verb|__name__|: These are special methods of python. Leave them alone. 
	\item \verb|__name| : Name mangling. This is something I will eventually learn somewhere. Google name mangling to understant a bit before any formal definition.
\end{itemize}
\end{shaded}	



\subsubsection{Instance Methods}

Here I want to consider some instance methods. It basically defines the methods inside these classes, and they act on the objects we define. 

In this case, we use them with the syntax
\begin{verbatim}
> object.my_method(argument)
\end{verbatim}

For example, let us keep with the \verb|User| class we defined above. Let us define some methods. 
\begin{verbatim}
> class User: # Classes are oftentimes capitalized
>    def __init__(self, first, last, age):
>        self.name = first
>        self.last = last
>        self.age = age
>
>    def full_name(self): # full_name method
>        return f"{self.first} {self.last}"
>
>    def initials(self): # initials method
>        return f"{self.first[0]} {self.last[0]}"
>
>    def likes(self, thing): # likes method. This methods has an argument
>        return f"{self.first} likes {thing}"
>
>    def is_senior(self): # Is senhor method has conditionals
>        return self.age >= 65    
>
>    def birthday(self):
>        self.age +=1
>        return f"Happy {self.age}th, {self.first}"
\end{verbatim}

Now, we can run these methods. First we need to define the users. 
\begin{verbatim}
> user1 = User("Joe", "Smith", 68)        
> user2 = User("Blanka", "Lopez", 41)
\end{verbatim}
Then, 
\begin{verbatim}
> print(user2.full_name())
> print(user1.likes("Ice Cream"))
>
> print(user1.initials())
> print(user2.initials())
>
> print(user2.is_senior())
> print(user1.age)
> print(user1.birthday())
> print(user1.age)	
\end{verbatim}

Another example is the bank account. See
\begin{verbatim}
> class BankAccount:
>     def __init__(self, owner, balance=0.0):
>         self.owner = owner
>         self.balance = balance
> 	
>     def deposit(self, add):
>         self.balance += add
>         return self.balance 
>
>     def withdraw(self, rem):
>         self.balance -= rem
>         return self.balance
> 
> acct = BankAccount("Darcy")
> 
> print(acct.owner)
> print(acct.balance)
> print(acct.deposit(10))
> print(acct.withdraw(3))
> print(acct.balance)	
\end{verbatim}


%%%%%%%%%%%%%%%%%%%%%%%%%%%%%%%%%%%%%%%%%%
%%%%%%%%%%%%%%%%%%%%%%%%%%%%%%%%%%%%%%%%%%

\subsubsection{Class attributes}

We now want to define attributes for the classes themselves, and not only for the instances. Let's come back to the User example. 
\begin{listing}{1}
class User:

    active_users = 0 # We define a class attribute

    def __init__(self,first,last,age):
        self.first = first
        self.last = last
        self.age = age
        User.active_users += 1

    def logout(self): 
        User.active_users -= 1 
        return f"{self.first} has logged out"  

    def full_name(self):
        return f"{self.first} {self.last}"

    def initials(self):
        return f"{self.first[0]}.{self.last[0]}."

    def likes(self, thing):
        return f"{self.first} likes {thing}"

    def is_senior(self):
        return self.age >= 65    

    def birthday(self):
        self.age +=1
        return f"Happy {self.age}th, {self.first}"
\end{listing}

In the case above, \verb|active|users| is a class attribute. Whenever a new user logs in or logs out we need to update the value of this attribute. We can test the code above with 

\begin{listingcont}{1}
print(User.active_users)

user1 = User("Joe", "Smith", 68)        
user2 = User("Blanka", "Lopez", 41)

print(User.active_users)

print(user2.logout())

print(User.active_users)
\end{listingcont}

Class attributes can be used as validation. For example, suppose we have a class for pets, and we want to forbid certain animals, for example, an alligator as a pet, then we can do the following:
\begin{listing}{1}
class Pet:

allowed = ['cat', 'dog', 'fish', 'rat']

def __init__(self, name, species):
	if species not in Pet.allowed:
		raise ValueError(f"You can't have a {species} pet")
	self.name = name
	self.species = species

def set_species(self, species):
	if species not in Pet.allowed:
		raise ValueError(f"You can't have a {species} pet")
	self.species = species	
\end{listing}

The second method, \verb|set_species| changes the species of our pet, but within the allowed class. If we deactivate the conditional part, we could change it the species of our pet to an animal that is not in the list of allowed pets. 

\begin{listingcont}
tonny = Pet("Tonny", "cat")

print(Pet.allowed)
print(tonny.allowed)

print(tonny.species)
new_species = tonny.set_species('dog')
print(tonny.species)

print(tonny.allowed == Pet.allowed)	
\end{listingcont}

Another example is the following:
\begin{listing}{1}
class Chicken:

    species = ['Partridge Silkie', 'Speckled Sussex']
    total_eggs = 0

    def __init__(self, name, species, eggs=0):
        # if species not in Chicken.allowed:
        #     raise ValueError(f"This is not a known {species}")
        self.name = name
        self.species = species
        self.eggs = eggs

    def lay_eggs(self):
        self.eggs += 1
        Chicken.total_eggs += 1

c1 = Chicken("Alice", 'Partridge Silkie')
c2 = Chicken("Amelia", 'Speckled Sussex') 

print(Chicken.total_eggs)
c1.lay_eggs()
print(Chicken.total_eggs)
c1.lay_eggs()
c1.lay_eggs()
print(Chicken.total_eggs)	
\end{listing}

%%%%%%%%%%%%%%%%%%%%%%%%%%%%%%%%%%%%%%%%%%
%%%%%%%%%%%%%%%%%%%%%%%%%%%%%%%%%%%%%%%%%%

\subsubsection{Class methods}

Now we want to define methods which are concerned with the class as as whole, and not with the instances themselves. We have encountered class methods before. The \verb|fromkeys| method is an example. 

Let us return to our old friend class of Users. Then 
\begin{listing}{1}
class User: 

    active_users = 0 # We define a class attribute

    @classmethod
    def display_active_users(cls):
        return f"There are currently {cls.active_users} active users."

    @classmethod
    def from_string(cls, data_str):
        first,last,age = data_str.split(",")        
        return cls(first,last,int(age))

    def __init__(self,first,last,age):
        self.first = first
        self.last = last
        self.age = age
        User.active_users += 1
		
	def __repr__(self):
		return f"{self.first}"	

    def logout(self): 
        User.active_users -= 1 
        return f"{self.first} has logged out"  

...	
\end{listing}
where we have omitted the unnecessary methods. The class methods are defined with \verb|@classmethod|. We have also defined an instance method to logout users and method with some mysterious \verb|__repr__| that we will explain below. Then, run the following code 
\begin{listingcont}
user1 = User("Joe", "Smith", 68)        
user2 = User("Blanka", "Lopez", 41)
print(User.display_active_users())
print(user2.logout())
print(User.display_active_users())	
\end{listingcont}
We see that initially there were 2 active users, then one logged out, and remains just 1 active user. 

The second class method is more interesting. Suppose we have a string that is in the format of comma separated values (csv) and we want to build a new user from it. We basically have to separate this string in 3 entries, the first name, last name and age. This method does it for us. Run 
\begin{listingcont}
tom = User.from_string("Tom, Jones, 89")
print(tom.first) 
print(tom.full_name(), ",", tom.age, "years old")
\end{listingcont}

There is one final piece of information in the code above we need to know. The \verb|__repr__| method is quite useful if we want to have some control over the output. First, comment this part of the code and try to run the following code
\begin{listingcont}
print(tom)
\end{listingcont}
we get something of the form \verb|<__main__.User object at XXX>|. On the other hand, we use the \verb|__repr__| method to control this outcome. In that case we obtain the first name as an output. 


\subsubsection{Exercise: Cards \& Deck Classes}

Here I want to consider the problem of section \# 25 of Colt's course. It is basically the definition of two classes with the following properties:

\textbf{-- Card class}
\begin{itemize}
    \item Each instance of Card  should have a suit ("Hearts", "Diamonds", "Clubs", or "Spades").
    \item Each instance of Card  should have a value ("A", "2", "3", "4", "5", "6", "7", "8", "9", "10", "J", "Q", "K").
    \item Card 's \verb|__repr__| method should return the card's value and suit (e.g. "A of Clubs", "J of Diamonds", etc.)
\end{itemize}


\textbf{-- Deck class}
\begin{itemize}
    \item Each instance of Deck should have a cards attribute with all 52 possible instances of Card.
    \item Deck  should have an instance method called count  which returns a count of how many cards remain in the deck.
    \item Deck 's \verb|__repr__| method should return information on how many cards are in the deck (e.g. "Deck of 52 cards", "Deck of 12 cards", etc.)
    \item Deck  should have an instance method called \verb|_deal|  which accepts a number and removes at most that many cards from the end of the deck (it may need to remove fewer if you request more cards than are currently in the deck!). If there are no cards left, this method should return a ValueError  with the message "All cards have been dealt".
    \item Deck  should have an instance method called shuffle  which will shuffle a full deck of cards. If there are cards missing from the deck, this method should raise a \verb|ValueError| with the message "Only full decks can be shuffled". shuffle should return the shuffled deck.
    \item Deck  should have an instance method called \verb|deal_card| which uses the \verb|_deal| method to deal a single card from the deck and return that single card.
    \item Deck  should have an instance method called \verb|deal_hand| which accepts a number and uses the \verb|_deal| method to deal a list of cards from the deck and return that list of cards.
\end{itemize}

\paragraph{My Solution}

Here is my solution
\begin{listing}{1}
from random import shuffle

class Card:
    allowed_suits = ('Hearts', 'Diamonds', 'Clubs', 'Spades')
    allowed_values =  ('A', '2', '3', '4', '5', '6', 
    '7', '8', '9', '10', 'J', 'Q', 'K')

    def __init__(self, value, suit):
        if value not in Card.allowed_values:            
            raise ValueError(f"Value needs to be 'A', '2', '3', '4', '5', 
            '6', '7', '8', '9', '10', 'J', 'Q', 'K'")
        if suit not in Card.allowed_suits:
            raise ValueError(f"Suit needs to be 'Hearts', 
            'Diamonds', 'Clubs' or 'Spades'")
        self.value = value
        self.suit = suit

    def __repr__(self):
        return f'{self.value} of {self.suit}'      
\end{listing}
and 
\begin{listingcont}
class Deck:
    def __init__(self):
        allowed_suits = ('Hearts', 'Diamonds', 'Clubs', 'Spades')
        allowed_values =  ('A', '2', '3', '4', '5', '6', '7', '8', '9',
        '10', 'J', 'Q', 'K')        
        self.cards = [Card(v,s) for v in allowed_values for s in allowed_suits]

    def __repr__(self):
        return f"Deck of {self.count()} cards"
        # I should favor the .format() format.
                    
    def count(self):
        return len(self.cards)

    def _deal(self, num):        
        cards = self.cards 
        hand = []
        count = 0
        if len(cards) == 0:
            raise ValueError("All cards have been dealt")
        while count < num:
            hand.append(cards.pop(-1))
            if len(cards) == 0:
                break
            count +=1
        return hand

    def shuffle(self):    
        if self.count() < 52:
            raise ValueError("Only full decks can be shuffled")

        shuffle(self.cards)
        return self            

    def deal_card(self):
        return self._deal(1)[0]

    def deal_hand(self, n):
        return self._deal(n)    
\end{listingcont}

I can also execute some tests
\begin{listingcont}

card = Card('A', 'Spades')
print(card)

my_deck = Deck()
print(my_deck.cards)
print(10 * '*')
my_deck.shuffle()
print(my_deck.cards)

card = my_deck.deal_card()
print(card)

print(my_deck.deal_hand(50))
print(my_deck.count())
print(my_deck.deal_hand(5))
print(my_deck.deal_hand(10))    
\end{listingcont}


\paragraph{Colt's solution}

Here I should compare with Colt's solution. 
\begin{listing}{1}
from random import shuffle

class Card:
    def __init__(self, value, suit):
        self.value = value
        self.suit = suit

    def __repr__(self):
        # return "{} of {}".format(self.value, self.suit)
        return f"{self.value} of {self.suit}"    
\end{listing}
and 
\begin{listingcont}
class Deck:
	def __init__(self):
		suits = ["Hearts", "Diamonds", "Clubs", "Spades"]
		values = ['A','2','3','4','5','6','7','8','9','10','J','Q','K']
		self.cards = [Card(value, suit) for suit in suits for value in values]

	def __repr__(self):
		return f"Deck of {self.count()} cards"

	def count(self):
		return len(self.cards)

	def _deal(self, num):
		count = self.count()
		actual = min([count,num])
		if count == 0:
			raise ValueError("All cards have been dealt")
		cards = self.cards[-actual:]
		self.cards = self.cards[:-actual]
		return cards

	def deal_card(self):
		return self._deal(1)[0]

	def deal_hand(self, hand_size):
		return self._deal(hand_size)

	def shuffle(self):
		if self.count() < 52:
			raise ValueError("Only full decks can be shuffled")

		shuffle(self.cards)
		return self    
\end{listingcont}
with tests
\begin{listingcont}

d = Deck()
d.shuffle()
card = d.deal_card()
print(card)
hand = d.deal_hand(50)
card2 = d.deal_card()
print(card2)
print(d.cards)
card2 = d.deal_card()

# print(d.cards)
\end{listingcont}



\subsection{Inheritance}

Suppose we want to build users with different functionalities. For example, ordinary users and moderators. These two share a lot in common, but moderators have additional functionalities. We could define two distinct classes, but it is easier to use this idea of \emph{inheritance}.

As a simpler example 
\begin{listing}{1}
class Animal:
    cool = True # This is a class attribute

    def make_sound(self, sound):
        print(sound)
    
class Cat(Animal):
    pass 

gandalf = Cat()
gandalf.make_sound
gandalf.cool
\end{listing}
I could use the \verb|isinstance(instance, object)| function that verifies that the instance belongs to both classes. 
\begin{listingcont}
print(isinstance(gandalf, Animal))
> True
print(isinstance(gandalf, Cat))
> True
print(isinstance(gandalf, list))
> False
\end{listingcont}


\subsubsection{Properties}

Consider the class Human:
\begin{listing}{1}
class Human: 
    def __init__(self, first, last, age):
        self.first = first 
        self.last = last
        if age >= 0:
            self.age = age
        else:
            self.age = 0
\end{listing}
where the conditionals avoid, for example, negative ages. The solution above does not prevent, on the other hand, that we change the age afterwords, for example if we replace \verb|jane.age = -100|. 
\begin{listingcont}
jane = Human("Jane", "Goodall", 50)
print(jane.age)
jane.age = -100
print(jane.age)
\end{listingcont}
Then, we need to define some properties. First of all, we need to do some modifications
\begin{listing}{1}
class Human: 
    def __init__(self, first, last, age):
        self.first = first 
        self.last = last
        if _age >= 0:
            self._age = age
        else:
            self.age = 0

# Not the best way of doing that:

#       def get_age(self):
#          return self._age
#       def set_age(self, new_age):
#           if _age >= 0:
#               self._age = age
#           else:
#               self.age = 0
 
# The best way of doing that:

    @property
    def age(self):
        return self._age

    @age.setter 
    def age(self, value):
        if value >=0:
            self._age =value
        else:
            raise ValueError("age can't be negative")
\end{listing}
then we run
\begin{listingcont}
jane = Human("Jane", "Goodall", 50)
print(jane.age)
\end{listingcont}

\subsubsection{Introduction to Super()}

The Cat class that we have considered above is useless as it was. We want to give more functionalities to it. 
\begin{listing}{1}
    class Animal:
    def __init__(self, name, species):
        self.name = name 
        self.species = species

    def __repr__(self):
        return f"{self.name} is a {self.species}"
        
    def make_sound(self, sound):
        print(sound)
    
class Cat(Animal):
    def __init__(self, name, breed, toy): 
# Repetition we want to avoid:
#       self.name = name 
#       self.species = "Cat" 
# This is one possibility to avoid repetition:
#       Animal.__init__(self, name, species="Cat") 
# This is the pythonic way:
        super().__init__(name, species="Cat")
        self.breed = breed
        self.toy = toy

        def play(self):
            print(f"{self.name} plays with {self.toy}")

gandalf = Cat("Gandalf", "Cat", "Scottish Fold", "String")
print(gandalf)
gandalf.play() # Cat method
gandalf.make_sound("Meow") # Animal method
\end{listing}

\subsubsection{Example of Inheritance: Users \& Moderators}

Let me come back to the original \verb|Users| class:
\begin{listing}{1}
class User: 

    active_users = 0 # We define a class attribute

    @classmethod
    def display_active_users(cls):
        return f"There are currently {cls.active_users} active users."

    @classmethod
    def from_string(cls, data_str):
        first,last,age = data_str.split(",")        
        return cls(first,last,int(age))

    def __init__(self,first,last,age):
        self.first = first
        self.last = last
        self.age = age
        User.active_users += 1
        
    def __repr__(self):
        return f"{self.first}"	

    def logout(self): 
        User.active_users -= 1 
        return f"{self.first} has logged out"  

    def full_name(self): # full_name method
        return f"{self.first} {self.last}"

    def initials(self): # initials method
        return f"{self.first[0]} {self.last[0]}"

    def likes(self, thing): # likes method. This methods has an argument
        return f"{self.first} likes {thing}"

    def is_senior(self): # Is senhor method has conditionals
        return self.age >= 65    

    def birthday(self):
        self.age +=1
        return f"Happy {self.age}th, {self.first}"
\end{listing}

We can now define the class of \verb|Moderators| and print some results:
\begin{listingcont}
class Moderator(User):
    total_mods = 0
    def __init__(self, first, last, age, community):
        super().__init__(first, last, age)
        self.community = community
        Moderator.total_mods += 1

    @classmethod
    def display_active_mods(cls):
        return f"There are currently {cls.total_mods} active mods."

    def remove_post(self):
        return f"{self.full__name()} removed a 
        post from the {self.community} community"

`user1 = User('Tom', 'Morello', 57)
user2 = User('Adam', 'Jones', 57)
user3 = User('Danny', 'Carey', 60)
jasmine = Moderator('Thiago', "Araujo", 34, 'Bands')      
jasmine = Moderator('Aline', "Lima", 35, 'Piano')      
print(User.display_active_users())
print(Moderator.display_active_mods())'  
\end{listingcont}


\subsubsection{Multiple Inheritance}

It is not recommended to use multiple inheritance. It is better to organize the ideas a bit better. Consider the example
\begin{listing}{1}
class Aquatic:
    def __init__(self, name):
        self.name = name 

    def swim(self):
        return f"{self.name} is swimming"

    def greet(self):
        return f"I am {self.name} of the sea!"        

class Ambulatory:
    def __init__(self, name):
        self.name = name 

    def walk(self):
        return f"{self.name} is walking"

    def greet(self):
        return f"I am {self.name} of the land"

class Penguin(Ambulatory, Aquatic):
    def __init__(self, name):
        # super().__init__(name = name) In this particular case, 
        #it is better to be explicit
        Ambulatory.__init__(self, name=name)
        Aquatic.__init__(self, name=name)
\end{listing}
and we can try the code 
\begin{listingcont}
jaws = Aquatic('Jaws')
lassie = Ambulatory('Lassie')
captain_cook = Penguin('Captain Cook')

print(captain_cook.swim())
print(captain_cook.walk())    
\end{listingcont}

\subsubsection{Method Resolution Order (MRO)}

Whenever we create a class, Python sets a MRO for that class which is the order in which Python will loop for methods on instances of that class.

We can see it in three ways
\begin{itemize}
    \item \verb|__mro__| attribute on the class 
    \item Use the method \verb|mro()| on the class 
    \item Use the builtin \verb|help(cls)| method
\end{itemize}

Using the Penguin example above, we can run
\begin{listing}{1}
Penguin.__mro__    
Penguin.mro()
help(Penguin)
\end{listing}

Another example
\begin{shaded}
\begin{listing}{1}
class A:
    def do_something(self):
        print("Method defined in: A")    

class B(A):
    def do_something(self):
        print("Method defined in: B")        

class C(A):
    def do_something(self):
        print("Method defined in: C")

class D(B, C):
    def do_something(self):
        print("Method defined in: D")        

thing = D()        
thing.do_something()

help(D)
\end{listing}    
The methods are inherited from \verb|D|, \verb|B|, \verb|C|, \verb|A|. We can comment the methods to see as it happens.
\end{shaded}


\subsection{Polymorphism}

There are two types of polymorphism.

\subsubsection{Polymorphism \& Inheritance}

\# 1. When the class method works in a similar way for different classes. For example, when we have a method in a parent class and that is overridden by a subclass. This is called \emph{method overriding}. For example:
\begin{listing}{1}
class Animal():
    def speak(self):
        raise NotImplementedError('Subclass needs to implement this method'
        
class Dog(Animal):
    def speak(self):
        return 'woof'        

class Cat(Animal):
    def speak(self):
        return 'meow'  
\end{listing}
It is an example of polymorphism.

\subsubsection{Special methods}

2. It also is when the same operation works for different kinds of objects. The \verb|len| method for example. For Example
\begin{listing}{1}
8 + 2
> 10
'8' + '2'
> '82'    
\end{listing}
In the first case, it is a sum of integers, and in the second example it is a concatenation of strings. 

\paragraph{Example}

Let us now define our own special method. These are examples of dunder methods: \verb|__something__|, such as \verb|__init__|, \verb|__repr__|, \verb|__len__| and so forth.
\begin{listing}{1}
from copy import copy

class Human:
    def __init__(self, first, last, age):
        self.first = first
        self.last = last
        self.age = age

    def __repr__(self):
        return f"Human named {self.first} {self.last}"

    def __len__(self):
        return self.age      

    def __add__(self, other):
        if isinstance(other, Human):
            return Human(first = 'Newborn', last = self.last, age = 0)
        return "You can't add that!"

    def __mul__(self, other):
        if isinstance(other, int):
            return [copy(self) for i in range(other)]
        return "Can't multiply"

j = Human('James', 'T. Kirk', 35) 
k = Human('Nyota', 'Uhura', 30) 
print(j)
print(len(j))

print(j + k)
print(j * 2)
\end{listing}


% ========================================================================
% REFERENCES
% ========================================================================

\clearpage

\bibliographystyle{utphys}
\bibliography{library.bib}

\end{document}
