\documentclass[a4paper,11pt]{amsart}

\pdfoutput=1 

% ========================================================================

\usepackage{/Users/thiago/.tex_templates/ams_math/definitions}

\hypersetup{
	pdftitle={integrable systems, programming and so on},
	pdfsubject={High Energy Physics, Python and so on},
	pdfauthor={Thiago Araujo},
	pdfkeywords={gauge; susy; strings; fields; cft; python},
	colorlinks=true,linkcolor=link,citecolor=link,urlcolor=link,linktocpage
}


\begin{document}

% ========================================================================
% BEGIN COVER: title, author, affiliation, abstract


\title[Notes on Java]{Notes on Java}

\author{Thiago Araujo}

%\address{\noindent AFFILIATION}
\email{\texttt{\href{thgr.araujo@gmail.com}{thgr.araujo@gmail.com}}} 

%\keywords{Temperley-Lieb, XXZ spin chain, KP, matrix integrals}
% \subjclass[2020]{37K10, 82B20, 82B23}
%\date{\today}

\begin{abstract}
Here I collect some basic facts about Java. Here I follow the course of Tim Buchalka and notes of 
the W3 School.

\bigskip

\noindent \textbf{Keywords:} Java 
\end{abstract}

\maketitle

\setcounter{tocdepth}{1}
\tableofcontents


% END COVER 
% ========================================================================



% ========================================================================
\section*{Introduction}
% ========================================================================

In these notes, I collect some basic facts about Java.
Here I follow the course of Tim Buchalka and notes of 
the W3 School.
% ========================================================================

\section{Fundamental aspects}

These notes are written from the course of Colt Steele on Git and Github.

\subsection{Username \& Email}

Here I want to define the username and email to my git repositories. 
First thing we do it the following
\begin{verbatim}
> git config --global user.name "NAME"
> git config --global user.email "E-MAIL"
\end{verbatim}
that define, respectively, the username and email. We can change it anytime we want, 
but keep it consistent. We can check it with
\begin{verbatim}
> git confic user.name
> git confic user.email
\end{verbatim}

\subsection{First Commands}

\subsubsection{Status} The first thing we need to understand is the command 
\begin{verbatim}
> git status
\end{verbatim}
It gives information on the current status of a git repository and its contents. 

\subsubsection{Init} The next command is 
\begin{verbatim}
> git init
\end{verbatim}
It makes the directory we are in a git repository. We run this command one time per project. 

The init creates a directory .git (it is hidden). If we delete this folder, 
it destroys the git history, so the directory is not a repository anymore. 
A common git mistake is the following: Once we create a repository, git watches 
everything that happens in the repository, even the creation of nested folders. 
For this reason, we do not run the git init in a folder inside a repository, 
it avoids mistakes.

\subsection{Commiting}

\subsubsection{Add} Now we want to commit the modifications we do on our projects.
\begin{verbatim}
> git add file1 file2 file3
\end{verbatim}
Once we have modified our project, we can pack the modifications or part 
of them using the add command. 

Also, we can add all modifications at once with 
\begin{verbatim}
> git add .
\end{verbatim}


\subsubsection{Commit}

Now we can commit the modifications we have done. 
\begin{verbatim}
> git commit
\end{verbatim}
In this option, we will need to provide the message using vim. 
Or we can write the message as follows
\begin{verbatim}
> git commit -m "my message"
\end{verbatim}

\subsubsection{Log}

It is very straightforward. This command gives a log of commits of a project. 
\begin{verbatim}
> git log    
\end{verbatim}

There are many options available (see documentation). One particular 
useful option is the following
\begin{verbatim}
> git log --abbrev-commit    
\end{verbatim}
that gives a shorter commit name. Another useful option is 
\begin{verbatim}
> git log --oneline    
\end{verbatim}
It gives a short version of the commit message.

\subsubsection{Amend}

Suppose we want to fix a mistake we make in a previous commit. Then we can do the following
\begin{listing}{1}
> git commit -m 'some message'
> git add forgotten_file 
> git commit --amend
\end{listing}

\subsubsection{Gitignore}

Here I just add the file \verb|.gitignore| and list the files and folders 
we do not want to track. 
There are several patterns to follow.Some patterns are 
\begin{listing}{1}
.DS_Store # it ignores files called .DS_Store
folderName/ # It ignores an entire directory
*.log # it ignores everything with the .log extension
\end{listing}

\subsection{Branching}

This is one of the last fundamental tool we learn in git. Everything 
else is secondary. The idea here is that if we need to modify our project, 
or work simultaneously with other people, we can create other branches 
of the project and merge them afterwards.

The default branch when we create a repository -- that is, when we git init -- 
is called \verb|master branch|. When we type \verb|git log|, we see that 
the most recent \verb|commit| is called \verb|HEAD|. As a matter of the fact, 
it is written something of the form \verb|HEAD -> master|. It refers
to the thing we are currently working.

\subsubsection{Git branch}

Here we can start with our first command. It is 
\begin{listing}{1}
> git branch
\end{listing}
that gives the existing branches in our repository. Moreover, the \verb|*| 
indicates our current branch.

\subsubsection{Creating and switching branches}

In order to create new branches, we use the same syntax, except that now 
we add the name of the branch, that is
\begin{listing}{1}
> git branch <new_branch>
\end{listing}
Observe that this command just creates another branch for us, 
but it does not change our work to the new branch. We can change the brach with
\begin{listing}{1}
> git switch <branch_name>
\end{listing}
It is important to notice that this command creates a new branch based on the current head.

There are another command to switch between branches, it is 
\begin{listing}{1}
> git checkout <branch_name>
\end{listing}   
This command does a zillion other things, so the \verb|switch| 
was introduced to be a simpler alternative. 

Finally, we can create a switch to the new branch at once; 
we just need to introduce a flag \verb|-c|, 
\begin{listing}{1}
> git switch -c <branch_name>
> git checkout -b <branch_name>
\end{listing} 

Finally, there is something important we need to understand. If we switch to new 
branches without committing the modifications, these would be lost. So, we need 
to commit (or stash them) them, and just after that we switch to new branches. 
Actually, sometimes, when the files do not exist in other branches - and there 
is no risk of conflicts - the files follow us to the new branch.

\subsubsection{Deleting and renaming branches}

We can delete the branch with the command 
\begin{listing}{1}
> git branch -d <branch_name>
\end{listing}   
It works if the branch is fully merged. Otherwise,  
instead the \verb|-d|, we can put \verb|-D| which denotes \verb|force|. We cannot delete inside the branch.

Moreover, we can rename the branch with
\begin{listing}{1}
> git branch -m <new_branch_name>
\end{listing}
In order to rename, we need to be inside the branch.

\subsection{Merging branches}

Finally, we have the last fundamental concept of git: Merging different branches. We need to 
keep in mind that the master brach should not be used for experiments; in other words, we must use 
the master branch just for ``permanent'' coding. 

This discussion ps somewhat straightforward. We use the following command 
\begin{listing}{1}
> git merge <name_modified_branch> 
\end{listing}
We need to be in the branch we want to update. Moreover, it is worth noticing that the merge is 
temporary, and it means that no branch is destroyed in the process. 

On the other hand, there are at least three different situations we need to distinguish when we 
merge branches.

\subsubsection{Fast-forward}

The first case is very easy. It is basically the case where the older commit is kept as it was, 
and the new branch basically adds new information. In this case, the commit just brings the old 
commit to a new point, and the two branches are similar. Evidently, if we change any branch, they 
diverges again. Remember that we need to switch to the branch we need to update, and  

\subsubsection{Merging without conflicts}

The second type of merging is that when two branches are modified, but no conflicts are found. 
It happens, for example, when the branches touch different files or add new files. 

In this case, git creates a new commit unifying all modifications.

\subsubsection{Merging with conflicts}

The last case is when there are conflicts. In this case, git denounces the conflicts, and we need 
to manually resolve them.







\section{Methods and other concepts}

Now, we want to start discussing some important aspects of the language. In particular, this section 
deals with:
\begin{itemize}
    \item Expressions
    \item Statements
    \item Code block
    \item Methods
\end{itemize}

\subsection{Expressions, statements and code blocks}

Expressions are the building blocks of all java programs. We know very well what expressions mean.
But we need to understand a bit of terminology. First of all, suppose we have a code of the form   
\begin{listing}{1}
public class Main {    
    public static  void main(String[] args) {
        int age = 35;
    }
}
\end{listing}
We say that the block \verb|age = 35| is an expressions, whilst \verb|int age = 35;| is the 
statement. Code blocks are made by everything between the curly brackets. 

One important aspect of code blocks, is that the variable defined inside a code block are not 
available outside it. For example: 
\begin{listing}{1}
public class Main {    
    public static void main(String[] args) {
        int age = 35;
        boolean birthday = true;
        if (birthday == true) {
            int newAge = age + 1;
            System.out.println("Your new age is" + newAge);
        }
        /* Now we cannot access the variable newAge
           it is defined inside the block code */ 
    }
}
\end{listing}

\subsection{Methods}

Methods is a way to avoid duplication of the code. We have been using methods since day one: the main 
method. See the following example:
\begin{listing}{1}
public class Main {

    public static void main(String[] args) {
        boolean gameOver = true;
        int score = 800;
        int levelCompleted = 5;
        int bonus = 100;

        calculateScore(gameOver, score, levelCompleted, bonus);

        score = 10000;
        levelCompleted = 8;
        bonus = 200;

        calculateScore(gameOver, score, levelCompleted, bonus);


    }

    public static int calculateScore(boolean gameOver,int score, 
    int levelCompleted, int bonus) {

        if(gameOver) {
            int finalScore = score + (levelCompleted * bonus);
            finalScore += 2000;
            System.out.println("Your final score was " + finalScore);
            return finalScore;
        }

        return -1;

    }
}
\end{listing}
In some programming languages, the value \(-1\) denotes Error.

Another example: Calculate the taxes. 
\begin{listing}{1}
public class Methods {

    public static void main (String[] args) {
      double salary = 1000d;
      //taxes(salary);
      System.out.println("You pay " + taxes(salary));
    }

    public static double taxes(double salary) {
       if (salary < 3000) { 
           //System.out.println("You don't have to pay anything");
           return 0;} else if (salary >=3000 && salary <6000) {
           //System.out.println("You pay 15%: " + salary * 0.15);
           return salary * 0.15;} else {
           //System.out.println("You pay 27%: " + salary * 0.27);
           return salary * 0.27;
           }
    }
    
}
\end{listing}


\subsubsection{Method overloading}

Method overloading describes how the same function can be used to do different things depending on
the signature. The example that follows helps the understanding of thid concept. It converts inches
and feet to centimeters.
\begin{listing}{1}
public class Conversion {
    public static void main(String[] args) {
    System.out.println(calcFeetAndInchesToCentimeters(5,7));
    System.out.println(calcFeetAndInchesToCentimeters(3.8,-5.6));
    System.out.println(calcFeetAndInchesToCentimeters(-3.8,5.6));
    System.out.println(calcFeetAndInchesToCentimeters(3.8,56));
    System.out.println(calcFeetAndInchesToCentimeters(3.8));
    System.out.println(calcFeetAndInchesToCentimeters(56));
    System.out.println(calcFeetAndInchesToCentimeters(-3.8));
    }

    public static double calcFeetAndInchesToCentimeters(double feet, double inches) {
    double centimeters = -1; 
    if (feet >= 0 && inches >= 0 && inches <= 12) {
    centimeters = (12 * feet + inches) * 2.54;
    }
    return centimeters;
    }

    public static double calcFeetAndInchesToCentimeters(double inches) {
    double feet = -1; 
    if (inches >= 0) {
    feet = inches / 12;
    }
    return calcFeetAndInchesToCentimeters(feet, 0);
    }

}
\end{listing}


\section{Control flow statements}

In this section we learn some important aspects of the language. In particular, we will learn
\begin{itemize}
    \item Switch
    \item For
    \item While 
    \item Do-while
\end{itemize}

\subsection{Switch}

The switch statement is an alternative to the if-else statement. It works as follows:
\begin{listing}{1}
public class Switch {
    public static void main(String[] args) {
   
    int number = 2;

    switch (number) {
        case 1:
            System.out.println("The number is one");
            break;

        case 2:
            System.out.println("The number is two");

        case 3: case 4: 
            System.out.println("The number is three or four");

        default:
            System.out.println("Another number");
            break;
    }

   }
}
\end{listing}
The \verb|case| is equivalent to the \verb|if| and \verb|else if| statements, and the \verb|default|
is equivalent to the \verb|else| statement. We also have the \verb|break|, and it stops the code block.
On the other hand, it has some limitations, in particular, in the \verb|if-else|, we can test different
variables, while in the \verb|switch| case we can only test one variable. Observe that we can combine
the cases in a single line -- case 3 and 4 above. 

Here are two nice examples:
\begin{listing}{1}
public class Numbers {
    public static void main(String[] args) {
    printNumberInWord(7);
    }

    public static void printNumberInWord(int number) {
    switch (number) {
        case 0:
            System.out.println("zero".toUpperCase());
            break;

        case 1:
            System.out.println("one".toUpperCase());
            break;

        case 2:
            System.out.println("two".toUpperCase());
            break;

        case 3:
            System.out.println("three".toUpperCase());
            break;

        case 4:
            System.out.println("four".toUpperCase());
            break;

        case 5:
            System.out.println("five".toUpperCase());
            break;

        case 6:
            System.out.println("six".toUpperCase());
            break;

        case 7:
            System.out.println("seven".toUpperCase());
            break;

        case 8:
            System.out.println("eight".toUpperCase());
            break;

        case 9:
            System.out.println("nine".toUpperCase());
            break;

        default:
            System.out.println("other".toUpperCase());
            break;
    }
    }

}
\end{listing}

The second example is:
\begin{listing}{1}
public class LeapYear {
    public static void main(String[] args) {
    System.out.println(isLeapYear(1986));
    System.out.println(getDaysInMonth(2,1600));
    System.out.println(getDaysInMonth(-2,2000));
    System.out.println(getDaysInMonth(12,21600));
    System.out.println(getDaysInMonth(2,1986));
    }

    public static boolean isLeapYear(int year) { 
    boolean result = false; 
    if (year < 1 || year > 9999) {
    result = false;
    } else {
    if ((year % 4 == 0) && (year % 100 != 0)) {
    result = true;
    } else if (year % 100 == 0 && year % 400 == 0) {
    result = true;
    } else {
    result = false;
    }
    }
    return result;
    }

    public static int getDaysInMonth(int month, int year) { 
    if (month < 1 || month >12 || year < 1 || year > 9999) { 
    return -1;
    }
    int days;
    boolean leap = isLeapYear(year);

    switch (month) {
        case 1: case 3: case 5: case 7: case 8: case 10: case 12:
            days = 31;
            break;

        case 4: case 6: case 9: case 11: 
            days = 30;
            break;

        default:
            if (leap) {
            days = 29;
            } else { 
            days = 28;
            }
            break;
    }
    return days; 
    }
}
\end{listing}

\subsection{For}

The syntax is standard. See the example:
\begin{listing}{1}
public class Interest {
    public static void main(String[] args) {
   // for(initial; termination; increment)
        /* 
    for(int i = 0; i <= 9; i++) { 
    System.out.println(i + 1 + 
    "% Interest of 1000000 is " + calculateInterest(1000000, i + 1));
    }
    */
    for(int i = 9; i >= 0; i--) { 
    System.out.println(i + 1 + 
    "% Interest of 1000000 is " + calculateInterest(1000000, i + 1));
    }
    }
    public static double calculateInterest(double amount, double rate) {    
    return (amount * (rate / 100));
    }
}
\end{listing}

Another example
\begin{listing}{1}
public class PrimeChecker {
   public static void main(String[] args) {
   for (int i = 15; i <= 30; i++) {
      System.out.println(i + " is prime: " + isPrime(i));
   }
   }

   public static boolean isPrime(int n) {
   int test = 0;
   if (n < 2) {
      return false;
   } else {
   for (int i = 2; i <= n / 2; i++) {
      if (n % i == 0) {
      test += 1;
      } 
   }
   if (test == 0) {
      return true;
   } else {
      return false;
   }
   }
   }
}
\end{listing}
 The code above is not the best. One can remember that the \verb|return| break the loop, 
 so we can improve the code. 

 Finally, see the example:
\begin{listing}{1}
public class SumOdd {
   public static void main(String[] args) {
   System.out.println(isOdd(7));
   System.out.println(sumOdd(1,100));
   System.out.println(sumOdd(-1,100));
   System.out.println(sumOdd(101,101));
   System.out.println(sumOdd(100,1000));
   }

   public static boolean isOdd(int number) {
      if (number % 2 == 1 && number != 0) {
         return true;
      } else {
      return false;
      }
   }

   public static int sumOdd(int start, int end) {
      int sum = 0;

      if (end < start || start < 0 || end < 0) {
      return -1;
      } else {

      for (int index = start; index <= end; index++) {
      if (isOdd(index)) {
      sum += index; 
      }
      }; 
      return sum;
      }
   }
}
\end{listing}


\subsection{While \& Do While}

The syntax is similar to what we have done so far. See the example:
\begin{listing}{1}
public class EvenChecker {
   public static void main(String[] args) {
   
      int start = 0;
      int end = 10;

//      while (start <= end) {
//      System.out.println("Is " + start + " even? " + isEvenNumber(start));
//      start++;
//      }

      do {
      System.out.println("Is " + start + " even? " + isEvenNumber(start));
      start++;
      } while (start <= end);
   }

   public static boolean isEvenNumber(int number) {
   if (number % 2 == 0) {
      return true;
   } else {
   return false;
   }
   }

}
\end{listing}
The part with and without comments are equivalent.

\subsection{Parsing values from a string}

Sometimes, we need to convert a string to a different data type. We can do it 
with the parse method as follows:
\begin{listing}{1}
public class ParseString {
   public static void main(String[] args) {
   String numberAsString = "2000";
   System.out.println(numberAsString);

   int number = Integer.parseInt(numberAsString);
   System.out.println(number);

   number += 1;
   numberAsString += 1;
   System.out.println(number);
   System.out.println(numberAsString);
   }

}
\end{listing}
In this example, observe that if we try to add a number to a string, java automatically converts
the number to a string.


\subsection{User input}

In order to provide the user the ability to insert values, we use the method \verb|scanner|. See
\begin{listing}{1}
import java.util.Scanner;

public class UserInput {

   public static void main(String[] args) {

   Scanner scanner = new Scanner(System.in);

   System.out.println("Enter your year of birth: ");
   int yearOfBirth = scanner.nextInt();
   scanner.nextLine(); // it handles next line character
   int age = 2022 - yearOfBirth;

   System.out.println("Enter your name: ");
   String name = scanner.nextLine();

   System.out.println("Your name is " + name + " and you're " + age + " years old");

   scanner.close();
   }
}
\end{listing}
The keyword \verb|new| creates a new instance of scanner.

In order to avoid negative numbers and words instead numbers, one can do the following
\begin{listing}{1}
import java.util.Scanner;

public class Main {

    public static void main(String[] args) {
        Scanner scanner = new Scanner(System.in);

        System.out.println("Enter your year of birth:");

        boolean hasNextInt = scanner.hasNextInt();

        if(hasNextInt) {
            int yearOfBirth = scanner.nextInt();
            scanner.nextLine(); // handle next line character (enter key)

            System.out.println("Enter your name: ");
            String name = scanner.nextLine();
            int age = 2018 - yearOfBirth;

            if(age >= 0 && age <= 100) {
                System.out.println("Your name is " + name + ", and you are " + age + " years old.");
            } else {
                System.out.println("Invalid year of birth");
            }
        } else {
            System.out.println("Unable to parse year of birth.");
        }

        scanner.close();
    }
}begin{listing}{1}
\end{listing}

The next example has all the important features of our discussion.
\begin{listing}{1}
import java.util.Scanner;

public class InputSum {
   public static void main(String[] args) {
      
      int sum = 0;
      int counter = 1;

      // Next line defines the variable scanner
      Scanner scanner = new Scanner(System.in);

      System.out.println("Enter 10 integers");

      while (counter <= 10) {
      
      System.out.println("Enter number #" + counter + ":");

      // Next line, we define the variable that tests if the variable is an integer
      boolean isInteger = scanner.hasNextInt();

      if (isInteger) {

         // read the user number 
         int number = scanner.nextInt();
         sum += number;
         counter++;
         } else {
            System.out.println("Invalid Number");
         }

      // Here we call the scanner again. In the next loop, the user inserts another number
      scanner.nextLine();
      }

      System.out.println("The total is: " + sum);

      scanner.close();

   }
}
\end{listing}


\section{Object Oriented Programming}

Now we try to understand the basics of Object Oriented Programming (OOP).

\subsection{Classes} 

Starting with classes, these are blueprints for the objects we will create. We have been creating 
classes since day one. Moreover, we have been creating variables inside methods, local variables, 
but now we will consider variables which are defined on the classes themselves. These variables 
are called \emph{fields}. The fields are traditionally kept private. 
\begin{listing}{1}
public class Car {

    private int doors; 
    // it is conventional to define these fields in private. 
    // It is the encapsulation
    private int wheels; 
    private String model; 
    private String engine; 
    private String colour; 

    // Setter
    public void setModel(String model) {
        String validModel = model.toLowerCase();
        if (validModel.equals("carrera") || validModel.equals("commodore")) {
                this.model = model; 
        } else {
                this.model = "Unknown"; 
        }
        // Observe that we have the field 'model' and the parameter 'model'. 
        // The keyword 'this' refers to the field.  
    }

    // Getter
    public String getModel() {
        return this.model;
    }
}
\end{listing}
Now we can create an instance of this class in a Main file. Private means that these variables
are not available outside the class Car. We can run this class in the main method 
\begin{listing}{1}
public class Main {
    public static void main(String[] args) {
        Car porsche = new Car();
        Car holden = new Car();
        porsche.setModel("Carrera");
        System.out.println("Model is " + porsche.getModel());
    }
}
\end{listing}

\subsection{Constructors}

See the class \verb|BankAccount| in my notes. It is a lot of typing if we want to initialize instances.
Instead using setters, one can save some time using constructors. 
\begin{listing}{1}
public class BankAccount {

    private long accountNumber;
    private long balance;
    private long phoneNumber;
    private String customerName;
    private String email;

    // constructor
    public BankAccount() {
    // empty constructor
    // default values can be imposed as 
    // this(1111, 0, "default", "default", "default"); 
    }

    public BankAccount(long accountNumber, long balance, String phoneNumber, String customerName, 
        string email) {
        // We can use setters here, but it is not recommended 
        this.accountNumber = accountNumber;
        this.balance = balance;
        this.phoneNumber = phoneNumber;
        this.customerName = customerName;
        this.email = email;
    }

    . . .
}
\end{listing}
Observe that we had to include an empty constructor to be able to use the \verb|new BankAccount()| 
as before. If we do not define a constructor, Java define the empty constructor for us. 
It is simple to conclude that this is a method overloading.
Actually, it is a good practice to write several constructors to allow several different configurations, 
but it is recommended to use the \verb|this(foo, bar)| to enter the default values. 
Then, in the main class, we can pass the constructor as 
\begin{listing}{1}
BankAccount melissaAccount = new BankAccount(23579, 1000, "2345", "Melissa Araujo", "myemail@melissa.com");
\end{listing}


\subsection{Inheritance}

The idea is to create classes that inherit features of other classes. For example,
one can try to build a general class for animals, and create other classes for 
specific types of animals. Say, there are some general features that all animals 
should satisfy, but there are others thar only small classes of animals have. 
See how it works. We start defining the class animals
\begin{listing}{1}
public class Animal {
    
    private String name;
    private int brain;
    private int body;
    private int size;
    private int weight;

    public void set(String name){
        this.name = name;
    }
    public void set(int brain){
        this.brain = brain;
    }
    public void setBody(int body) {
        this.body = body;
    }
    public void setSize(int size) {
        this.size = size;
    }
    public void setWeight(int weight) {
        this.weight = weight;
    }

    public String getName() {
        return name;
    }
    public int getBrain() {
        return brain;
    }
    public int getBody() {
        return body;
    }
    public int getSize() {
        return size;
    }
    public int getWeight() {
        return weight;
    }


    public Animal(String name, int brain, int body, int size, int weight) {
        this.name = name;
        this.brain = brain;
        this.body = body;
        this.size = size;
        this.weight = weight;
    }

    public void eat() {
        System.out.println("method called...");
    }

    public void move(int speed) {
        System.out.println("Animal is moving at " + speed);
    }
}
\end{listing}
Next, we define animals, we explain the syntax later 
\begin{listing}{1}
public class Dog extends Animal { 

    private int eyes;
    private int legs;
    private int tail;
    private int teeth;
    private String coat;
    
    // public Animal(String name, int brain, int body, int size, int weight) {
        // super(name, brain, body, size, weight);
    // }
    // Actually, we do not need to use all the arguments, since some of them are 
    // obvious for dogs, for example, the number of brains, we do 

    public Dog(String name, int size, int weight, int eyes, int legs, int tail, int teeth, String coat) {
        super(name, 1, 1, size, weight);
            this.eyes = eyes;
            this.legs = legs;
            this.tail = tail;
            this.teeth = teeth;
            this.coat = coat;
    }
    
    // We can override methods for specific methods. 

    private void chew() {
       System.out.println("Dog.chew() method");
    }
    
    @Override
    public void eat() {
        System.out.println("Dog.eat() called");
        chew();
        super.eat();
    }
    
    public void walk() {
        System.out.println("Dog.walk() called");
        super.move(5);
    }
    
    public void run() {
        System.out.println("Dog.run() called");
        move(10);
    }

    private void moveLegs(int speed) {
        System.out.println("Dog.moveLegs() called");
    }

    @Override
    public void move(int speed) {
        System.out.println("Dog.move() called");
        moveLegs(speed);
        super.move(speed);
    }

}
\end{listing}
We use the keyword \verb|extends| to initialize the class \verb|Animal|. Then we 
write the constructor, and we add \verb|super| to call the Animals features.
Without the super keyword, java looks for the method inside the subclass, the 
one that has been overriden. Moreover, the methods of Animals are also available 
here. 

\subsection{Some terminology}

Here we discuss some terminology we have been using so far. 

\subsubsection{Reference \& Object \& Instance \& Class}

We have been using these keywords for a while, and now it is time to understand the differences 
with a little more detail. Use an anology of building a house. 

\emph{Classes} are blueprints for houses. Using these blueprints, one can build as many houses as 
one wants. Each house we build is an \emph{instance}, also known as an \emph{object}. Each house 
has an address, this is the \emph{reference}.
\begin{listing}{1}
    . . .
    House blueHouse = new House("blue");
    House anotherHouse = blueHouse;
    . . .
\end{listing}
In this example, we have an object of type House, and two references: \verb|blueHouse| and 
\verb|anotherHouse|.

\subsubsection{This \& Super}

The keyword \verb|super| is used to access the parent class members, that is, variables and methods.
The keyword \verb|this| is used to call the current class members (variables and methods). This is 
required when we have a parameter with the same name as an instance vatiable (field). 

The keyword \verb|this| is commonly used in setters and getters, while the keyword \verb|super| is 
commonly used with method overriding, that is, then we call a method with the same name from a
method in the parent class.
\begin{listing}{1}
    . . . // We are in a SubClass (child class 
@Override 
    public void printMethod() {
        super.printMethod(); // it class a method from SuperClass (parent)
    }
\end{listing}
Without this \verb|super| keyword, the method would be recursive. 

We also have the \verb|this()| and \verb|super()| calls - notice the brackets. We use this() to call
a constructor from another overloaded constructor in the same class. Is can only be used in a 
constructor, and in fact, it must be the first statement of this constructor. It helps us to reduce
code repetition (DRY - don't repeat yourself).

The only way to call a parent constructor is using the super(). If we don't add it, the compiler 
inserts it for us. Is also must be the first statement. We cannot use both, this() and super(), 
in the same constructor. 

Let us consider some example. First, the bad example: 
\begin{listing}{1}
class Rectangle {
    private int x;
    private int y;
    private int width;
    private int height;

    public Rectangle () {
        this.x = 0;
        this.y = 0;
        this.weight = 0; 
        this.height = 0;
    }

    public Rectangle (int width, int height) {
        this.x = 0;
        this.y = 0;
        this.weight = weight; 
        this.height = height;
    }

    public Rectangle (int x, int y, int weight, int height) {
        this.x = x;
        this.y = y;
        this.weight = weight; 
        this.height = height;
    }

}
\end{listing}
Now, the good example:
\begin{listing}{1}
class Rectangle {
    private int x;
    private int y;
    private int width;
    private int height;

    // 1st constructor
    public Rectangle () {
        this(0, 0); // calls the 2nd constructor
    }

    // 2nd constructor
    public Rectangle (int width, int height) {
        this(0, 0, width, height); // calls the 3rd constructor
    }

    // 3rd constructor
    public Rectangle (int x, int y, int weight, int height) {
        // initialize variables 
        this.x = x;
        this.y = y;
        this.weight = weight; 
        this.height = height;
    }

}
\end{listing}
No matter what constructor we call, the variables will always be initialized by the 3rd constructor.

Now, let us consider an example of the super() call. First, the parent class Shape:
\begin{listing}{1}
class Shape { 
    private int x; 
    private int y; 

    public Shape(int x, int y) {
        this.x = x;
        this.y = y; 
    }
}
\end{listing}
Now, the child class Rectangle: 
\begin{listingcont}
class Rectangle extends Shape {
    private int width; 
    private height; 

    // 1st constructor
    public Rectangle (int x, int y) {
        this(x, y, 0, 0); // calls the 2nd constructor
    }

    // 2nd constructor
    public Rectangle (int x, int y, int weight, int height) {
        // initialize variables 
        super(x, y); // calls constructor from parent class (Shape)
        this.weight = weight; 
        this.height = height;
    }

}
\end{listingcont}


\subsubsection{Method Overloading \& Method Overriding}

\paragraph{Overloading} Method overloading means providing two or more separate methods in a 
class with the same name but different parameters. Method returns type may or may not be 
different and that allows us to reuse the same method name. It allows to avoid duplication. 
It does not have anything to do with Polymorphism, but Java Devs often refer to overloading as 
Compile Time Polymorphism. It can be used inside the same class and/or in child classes, since child 
classes inherit the methods from their parent class. 

\paragraph{Overriding} It denotes the definition of a method in a child class that already exists 
in the parent class with same signature (same name and arguments). It is also called Runtime 
Polymorphism and Dynamic Method Dispatch. It is recommended to put @Override immediately above 
the method. It can be defined just in child classes.

Let consider some simple examples: 
\begin{listing}{1}
class Dog { 
    public void bark() {
        System.out.println("woof");
    }

    /Overloading
    public void bark(int number) {
        for (int i = 0; i < number, i++) {
            System.out.println("woof");
        }
    }
}

class GermanShepherd extends Dog {
    
    // Overriding 
    @Override 
    public void bark() { 
        system.out.println("woof woof woof");
    }
}
\end{listing}

The main differences are summarized in figure~\ref{fig:over}.
\begin{figure}[htb!]
	\includegraphics[width=0.7\textwidth]{over.png}
    \caption{Comparison Overloading and Overring.}
	\label{fig:over}
\end{figure}

Finally, it might be confusing the terminology \emph{covariant} above, but it 
denotes how the return type varies as we go to a SubClass. For example, consider 
two classes
\begin{listing}{1}
class Burguer {
// fields, methods, ...
}

class HealthyBurguer extends Burguer {
// fields, methods, ...
}
\end{listing}

\begin{listing}{1}
class BurguerFactory {
    public Burguer createBurguer() {
        return new Burguer();
    }
}

class HealthyBurguerFactory {
    public HealthyBurguer createBurguer() {
        return new HealthyBurguer();
    }
}
\end{listing}


\subsubsection{Static Methods \& Instance Methods}

\paragraph{Static Methods} These methods are declared using a verb|static| modifier. These 
methods can't access instance methods and instances variables directly. They are
usually used for operations that don't require any data from an instance of the 
class (from 'this'). When we see a method that does not use instance variables, that
methods should be declared as static. 
\begin{listing}{1}
class Calculator {
    public static void printSum(int a, int b) {
        System.out.println("sum = " + (a + b));
    }
}
\end{listing}
and
\begin{listing}{1}
public class Main {
    public static void main(String[] args) {
        Calculator.printSum(5,10); // call static method 
        printHello(); // call static method 
    }

    public static void printHello() {
        System.out.println("Hello");
    }
}
\end{listing}
Observe that we do not need an instance to apply the method. It is what makes this
a static class.

\paragraph{Instance Methods} These methods belong to an instance of a class. To use 
an instance method, we need to use the \verb|new| keyword.
\begin{listing}{1}
class Dog {
    public void bark() {
        System.out.println("woof");
    }
}
\end{listing}
and
\begin{listing}{1}
public class Main {
    public static void main(String[] args) {
        Dog rex = new Dog(); // create instance
        rex.bark(); // call instance method 
    }
}
\end{listing}

Should we use static or instance methods? 
\begin{figure}[htb!]
	\includegraphics[width=0.7\textwidth]{methods.png}
    \caption{Shold we use static or instance methods?}
	\label{fig:methods}
\end{figure}


\subsubsection{Static Variables \& Instance Variables}

\paragraph{Static} These are defined using the keyword static. They are also known as static member
variables. They are not used very often, but might be useful in some particular cases, for 
example, when we need an imput of the user, that is, in scanners. 

In any case, see the behaviour of these variables in the following example: 
\begin{listing}{1}
class Dog { 
    private static String name; 

    public Dog(String name) {
        Dog.name = name;
    }

    public void printName() {
        System.out.println("name = " + name);
    }
}
\end{listing}
and 
\begin{listing}{1}
public class Main {
    public static void main(String[] args) {
        Dog rex = new Dog("rex"); // create instance rex
        Dog fluffy = new Dog("fluffy"); // create instance fluffy
        rex.printName(); // prints fluffy
        fluffy.printName(); // prints fluffy
    }
}
\end{listing}
It prints \verb|fluffy| in both cases, because we have changed the variable name to fluffy when we 
create the instance fluffy. 

\paragraph{Instance} We do not use the keyword static. Moreover, instance variables are also known 
as fields or member variables. Instance variables belong to the instance of a class. It represents 
the state of a given instance. Consider the previous example, but without the keyword static. Then 
\begin{listing}{1}
class Dog { 
    private String name; 

    public Dog(String name) {
        this.name = name;
    }

    public void printName() {
        System.out.println("name = " + name);
    }
}
\end{listing}
and 
\begin{listing}{1}
public class Main {
    public static void main(String[] args) {
        Dog rex = new Dog("rex"); // create instance rex
        Dog fluffy = new Dog("fluffy"); // create instance fluffy
        rex.printName(); // prints rex
        fluffy.printName(); // prints fluffy
    }
}
\end{listing}


% ========================================================================
% REFERENCES
% ========================================================================

\clearpage

\bibliographystyle{utphys}
\bibliography{library.bib}

\end{document}
