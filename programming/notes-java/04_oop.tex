\section{Object Oriented Programming}

Now we try to understand the basics of Object Oriented Programming (OOP).

\subsection{Classes} 

Starting with classes, these are blueprints for the objects we will create. We have been creating 
classes since day one. Moreover, we have been creating variables inside methods, local variables, 
but now we will consider variables which are defined on the classes themselves. These variables 
are called \emph{fields}. The fields are traditionally kept private. 
\begin{listing}{1}
public class Car {

    private int doors; 
    // it is conventional to define these fields in private. 
    // It is the encapsulation
    private int wheels; 
    private String model; 
    private String engine; 
    private String colour; 

    // Setter
    public void setModel(String model) {
        String validModel = model.toLowerCase();
        if (validModel.equals("carrera") || validModel.equals("commodore")) {
                this.model = model; 
        } else {
                this.model = "Unknown"; 
        }
        // Observe that we have the field 'model' and the parameter 'model'. 
        // The keyword 'this' refers to the field.  
    }

    // Getter
    public String getModel() {
        return this.model;
    }
}
\end{listing}
Now we can create an instance of this class in a Main file. Private means that these variables
are not available outside the class Car. We can run this class in the main method 
\begin{listing}{1}
public class Main {
    public static void main(String[] args) {
        Car porsche = new Car();
        Car holden = new Car();
        porsche.setModel("Carrera");
        System.out.println("Model is " + porsche.getModel());
    }
}
\end{listing}

\subsection{Constructors}

See the class \verb|BankAccount| in my notes. It is a lot of typing if we want to initialize instances.
Instead using setters, one can save some time using constructors. 
\begin{listing}{1}
public class BankAccount {

    private long accountNumber;
    private long balance;
    private long phoneNumber;
    private String customerName;
    private String email;

    // constructor
    public BankAccount() {
    // empty constructor
    // default values can be imposed as 
    // this(1111, 0, "default", "default", "default"); 
    }

    public BankAccount(long accountNumber, long balance, String phoneNumber, String customerName, 
        string email) {
        // We can use setters here, but it is not recommended 
        this.accountNumber = accountNumber;
        this.balance = balance;
        this.phoneNumber = phoneNumber;
        this.customerName = customerName;
        this.email = email;
    }

    . . .
}
\end{listing}
Observe that we had to include an empty constructor to be able to use the \verb|new BankAccount()| 
as before. If we do not define a constructor, Java define the empty constructor for us. 
It is simple to conclude that this is a method overloading.
Actually, it is a good practice to write several constructors to allow several different configurations, 
but it is recommended to use the \verb|this(foo, bar)| to enter the default values. 
Then, in the main class, we can pass the constructor as 
\begin{listing}{1}
BankAccount melissaAccount = new BankAccount(23579, 1000, "2345", "Melissa Araujo", "myemail@melissa.com");
\end{listing}


\subsection{Inheritance}

The idea is to create classes that inherit features of other classes. For example,
one can try to build a general class for animals, and create other classes for 
specific types of animals. Say, there are some general features that all animals 
should satisfy, but there are others thar only small classes of animals have. 
See how it works. We start defining the class animals
\begin{listing}{1}
public class Animal {
    
    private String name;
    private int brain;
    private int body;
    private int size;
    private int weight;

    public void set(String name){
        this.name = name;
    }
    public void set(int brain){
        this.brain = brain;
    }
    public void setBody(int body) {
        this.body = body;
    }
    public void setSize(int size) {
        this.size = size;
    }
    public void setWeight(int weight) {
        this.weight = weight;
    }

    public String getName() {
        return name;
    }
    public int getBrain() {
        return brain;
    }
    public int getBody() {
        return body;
    }
    public int getSize() {
        return size;
    }
    public int getWeight() {
        return weight;
    }


    public Animal(String name, int brain, int body, int size, int weight) {
        this.name = name;
        this.brain = brain;
        this.body = body;
        this.size = size;
        this.weight = weight;
    }

    public void eat() {
        System.out.println("method called...");
    }

    public void move(int speed) {
        System.out.println("Animal is moving at " + speed);
    }
}
\end{listing}
Next, we define animals, we explain the syntax later 
\begin{listing}{1}
public class Dog extends Animal { 

    private int eyes;
    private int legs;
    private int tail;
    private int teeth;
    private String coat;
    
    // public Animal(String name, int brain, int body, int size, int weight) {
        // super(name, brain, body, size, weight);
    // }
    // Actually, we do not need to use all the arguments, since some of them are 
    // obvious for dogs, for example, the number of brains, we do 

    public Dog(String name, int size, int weight, int eyes, int legs, int tail, int teeth, String coat) {
        super(name, 1, 1, size, weight);
            this.eyes = eyes;
            this.legs = legs;
            this.tail = tail;
            this.teeth = teeth;
            this.coat = coat;
    }
    
    // We can override methods for specific methods. 

    private void chew() {
       System.out.println("Dog.chew() method");
    }
    
    @Override
    public void eat() {
        System.out.println("Dog.eat() called");
        chew();
        super.eat();
    }
    
    public void walk() {
        System.out.println("Dog.walk() called");
        super.move(5);
    }
    
    public void run() {
        System.out.println("Dog.run() called");
        move(10);
    }

    private void moveLegs(int speed) {
        System.out.println("Dog.moveLegs() called");
    }

    @Override
    public void move(int speed) {
        System.out.println("Dog.move() called");
        moveLegs(speed);
        super.move(speed);
    }

}
\end{listing}
We use the keyword \verb|extends| to initialize the class \verb|Animal|. Then we 
write the constructor, and we add \verb|super| to call the Animals features.
Without the super keyword, java looks for the method inside the subclass, the 
one that has been overriden. Moreover, the methods of Animals are also available 
here. 

\subsection{Some terminology}

Here we discuss some terminology we have been using so far. 

\subsubsection{Reference \& Object \& Instance \& Class}

We have been using these keywords for a while, and now it is time to understand the differences 
with a little more detail. Use an anology of building a house. 

\emph{Classes} are blueprints for houses. Using these blueprints, one can build as many houses as 
one wants. Each house we build is an \emph{instance}, also known as an \emph{object}. Each house 
has an address, this is the \emph{reference}.
\begin{listing}{1}
    . . .
    House blueHouse = new House("blue");
    House anotherHouse = blueHouse;
    . . .
\end{listing}
In this example, we have an object of type House, and two references: \verb|blueHouse| and 
\verb|anotherHouse|.

\subsubsection{This \& Super}

The keyword \verb|super| is used to access the parent class members, that is, variables and methods.
The keyword \verb|this| is used to call the current class members (variables and methods). This is 
required when we have a parameter with the same name as an instance vatiable (field). 

The keyword \verb|this| is commonly used in setters and getters, while the keyword \verb|super| is 
commonly used with method overriding, that is, then we call a method with the same name from a
method in the parent class.
\begin{listing}{1}
    . . . // We are in a SubClass (child class 
@Override 
    public void printMethod() {
        super.printMethod(); // it class a method from SuperClass (parent)
    }
\end{listing}
Without this \verb|super| keyword, the method would be recursive. 

We also have the \verb|this()| and \verb|super()| calls - notice the brackets. We use this() to call
a constructor from another overloaded constructor in the same class. Is can only be used in a 
constructor, and in fact, it must be the first statement of this constructor. It helps us to reduce
code repetition (DRY - don't repeat yourself).

The only way to call a parent constructor is using the super(). If we don't add it, the compiler 
inserts it for us. Is also must be the first statement. We cannot use both, this() and super(), 
in the same constructor. 

Let us consider some example. First, the bad example: 
\begin{listing}{1}
class Rectangle {
    private int x;
    private int y;
    private int width;
    private int height;

    public Rectangle () {
        this.x = 0;
        this.y = 0;
        this.weight = 0; 
        this.height = 0;
    }

    public Rectangle (int width, int height) {
        this.x = 0;
        this.y = 0;
        this.weight = weight; 
        this.height = height;
    }

    public Rectangle (int x, int y, int weight, int height) {
        this.x = x;
        this.y = y;
        this.weight = weight; 
        this.height = height;
    }

}
\end{listing}
Now, the good example:
\begin{listing}{1}
class Rectangle {
    private int x;
    private int y;
    private int width;
    private int height;

    // 1st constructor
    public Rectangle () {
        this(0, 0); // calls the 2nd constructor
    }

    // 2nd constructor
    public Rectangle (int width, int height) {
        this(0, 0, width, height); // calls the 3rd constructor
    }

    // 3rd constructor
    public Rectangle (int x, int y, int weight, int height) {
        // initialize variables 
        this.x = x;
        this.y = y;
        this.weight = weight; 
        this.height = height;
    }

}
\end{listing}
No matter what constructor we call, the variables will always be initialized by the 3rd constructor.

Now, let us consider an example of the super() call. First, the parent class Shape:
\begin{listing}{1}
class Shape { 
    private int x; 
    private int y; 

    public Shape(int x, int y) {
        this.x = x;
        this.y = y; 
    }
}
\end{listing}
Now, the child class Rectangle: 
\begin{listingcont}
class Rectangle extends Shape {
    private int width; 
    private height; 

    // 1st constructor
    public Rectangle (int x, int y) {
        this(x, y, 0, 0); // calls the 2nd constructor
    }

    // 2nd constructor
    public Rectangle (int x, int y, int weight, int height) {
        // initialize variables 
        super(x, y); // calls constructor from parent class (Shape)
        this.weight = weight; 
        this.height = height;
    }

}
\end{listingcont}


\subsubsection{Method Overloading \& Method Overriding}

\paragraph{Overloading} Method overloading means providing two or more separate methods in a 
class with the same name but different parameters. Method returns type may or may not be 
different and that allows us to reuse the same method name. It allows to avoid duplication. 
It does not have anything to do with Polymorphism, but Java Devs often refer to overloading as 
Compile Time Polymorphism. It can be used inside the same class and/or in child classes, since child 
classes inherit the methods from their parent class. 

\paragraph{Overriding} It denotes the definition of a method in a child class that already exists 
in the parent class with same signature (same name and arguments). It is also called Runtime 
Polymorphism and Dynamic Method Dispatch. It is recommended to put @Override immediately above 
the method. It can be defined just in child classes.

Let consider some simple examples: 
\begin{listing}{1}
class Dog { 
    public void bark() {
        System.out.println("woof");
    }

    /Overloading
    public void bark(int number) {
        for (int i = 0; i < number, i++) {
            System.out.println("woof");
        }
    }
}

class GermanShepherd extends Dog {
    
    // Overriding 
    @Override 
    public void bark() { 
        system.out.println("woof woof woof");
    }
}
\end{listing}

The main differences are summarized in figure~\ref{fig:over}.
\begin{figure}[htb!]
	\includegraphics[width=0.7\textwidth]{over.png}
    \caption{Comparison Overloading and Overring.}
	\label{fig:over}
\end{figure}

Finally, it might be confusing the terminology \emph{covariant} above, but it 
denotes how the return type varies as we go to a SubClass. For example, consider 
two classes
\begin{listing}{1}
class Burguer {
// fields, methods, ...
}

class HealthyBurguer extends Burguer {
// fields, methods, ...
}
\end{listing}

\begin{listing}{1}
class BurguerFactory {
    public Burguer createBurguer() {
        return new Burguer();
    }
}

class HealthyBurguerFactory {
    public HealthyBurguer createBurguer() {
        return new HealthyBurguer();
    }
}
\end{listing}


\subsubsection{Static Methods \& Instance Methods}

\paragraph{Static Methods} These methods are declared using a verb|static| modifier. These 
methods can't access instance methods and instances variables directly. They are
usually used for operations that don't require any data from an instance of the 
class (from 'this'). When we see a method that does not use instance variables, that
methods should be declared as static. 
\begin{listing}{1}
class Calculator {
    public static void printSum(int a, int b) {
        System.out.println("sum = " + (a + b));
    }
}
\end{listing}
and
\begin{listing}{1}
public class Main {
    public static void main(String[] args) {
        Calculator.printSum(5,10); // call static method 
        printHello(); // call static method 
    }

    public static void printHello() {
        System.out.println("Hello");
    }
}
\end{listing}
Observe that we do not need an instance to apply the method. It is what makes this
a static class.

\paragraph{Instance Methods} These methods belong to an instance of a class. To use 
an instance method, we need to use the \verb|new| keyword.
\begin{listing}{1}
class Dog {
    public void bark() {
        System.out.println("woof");
    }
}
\end{listing}
and
\begin{listing}{1}
public class Main {
    public static void main(String[] args) {
        Dog rex = new Dog(); // create instance
        rex.bark(); // call instance method 
    }
}
\end{listing}

Should we use static or instance methods? 
\begin{figure}[htb!]
	\includegraphics[width=0.7\textwidth]{methods.png}
    \caption{Shold we use static or instance methods?}
	\label{fig:methods}
\end{figure}


\subsubsection{Static Variables \& Instance Variables}

\paragraph{Static} These are defined using the keyword static. They are also known as static member
variables. They are not used very often, but might be useful in some particular cases, for 
example, when we need an imput of the user, that is, in scanners. 

In any case, see the behaviour of these variables in the following example: 
\begin{listing}{1}
class Dog { 
    private static String name; 

    public Dog(String name) {
        Dog.name = name;
    }

    public void printName() {
        System.out.println("name = " + name);
    }
}
\end{listing}
and 
\begin{listing}{1}
public class Main {
    public static void main(String[] args) {
        Dog rex = new Dog("rex"); // create instance rex
        Dog fluffy = new Dog("fluffy"); // create instance fluffy
        rex.printName(); // prints fluffy
        fluffy.printName(); // prints fluffy
    }
}
\end{listing}
It prints \verb|fluffy| in both cases, because we have changed the variable name to fluffy when we 
create the instance fluffy. 

\paragraph{Instance} We do not use the keyword static. Moreover, instance variables are also known 
as fields or member variables. Instance variables belong to the instance of a class. It represents 
the state of a given instance. Consider the previous example, but without the keyword static. Then 
\begin{listing}{1}
class Dog { 
    private String name; 

    public Dog(String name) {
        this.name = name;
    }

    public void printName() {
        System.out.println("name = " + name);
    }
}
\end{listing}
and 
\begin{listing}{1}
public class Main {
    public static void main(String[] args) {
        Dog rex = new Dog("rex"); // create instance rex
        Dog fluffy = new Dog("fluffy"); // create instance fluffy
        rex.printName(); // prints rex
        fluffy.printName(); // prints fluffy
    }
}
\end{listing}
