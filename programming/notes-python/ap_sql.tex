\section{SQL and Python}

Here I want to understand how to handle data. SQLite3 is a simple language, so we start here. First of all, the command \verb|.help| 
is, naturally, the help.

\subsection{SQL crash course}

There are four types of data we can store, these are
\begin{itemize}
    \item Integers
    \item Reals
    \item Blobs - these data types are stored as they are. 
\end{itemize}

\subsubsection{Creating tables}

The syntax for creating tables is very straightforward. As a simple example, consider the following table
\begin{listing}{1}
CREATE TABLE dogs(
    name=TEXT,
    breed=TEXT,
    age=INTEGER
)
\end{listing}
It is a convention to have SQL commands capitalized. If we are inside the sqlite interface, this command above is 
lost when we close the window, so we need to save it. So, we save it in a file \verb|.db|. For example
\begin{listing}{1}
.open dogs_db.db -- it creates the file
CREATE TABLE dogs(
    name=TEXT,
    breed=TEXT,
    age=INTEGER
);
\end{listing}
We can check the existing tables with the command \verb|.tables|.

\subsubsection{Inserting data}

Now we need to learn how to insert data to our files. It is done with the command
\begin{listing}{1}
INSERT INTO dogs (name, breed, age) VALUES ("Blue", "Scottish Fold", 3);
SELECT * FROM cats;
\end{listing}   

\subsection{SQL + Python}

Now we would like to manage SQL tables with python. Here we can import a module to work with 
SQL. It is the \verb|sqlite3|. 
\begin{listing}{1}
import sqlite3
conn = sqlite3.connect("books.db")
# Create cursor
c = conn.cursor()
# Execute some sql
c.execute(" CREATE TABLE books (Author TEXT, Title TEXT, Publication Year INTEGER);")
# Commit changes
c.commit()

conn.close()begin{listing}{1}
import sqlite3 
conn = sqlite3.connect("books.db")

conn.close()
\end{listing}
The second command opens an existing database, or create one. The other commands are easy to understand. 
At the end, we need to close our database. Now we want to add information to this table.
\begin{listing}{1}
import sqlite3
conn = sqlite3.connect("books.db")
# Create cursor
c = conn.cursor()

# Execute some sql
# c.execute(" CREATE TABLE books (Author TEXT, Title TEXT, Publication Year INTEGER);")
data = ("Steven Weinberg", "Quantum Field Theory", 1997)
query = "INSERT INTO books VALUES (?,?,?)"
c.execute(query, data)

# Commit changes
conn.commit()
conn.close()
\end{listing}