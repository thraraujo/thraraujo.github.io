\documentclass[a4paper,11pt]{amsart}

\pdfoutput=1 

\usepackage{definitions}

\hypersetup{
	pdftitle={python},
	pdfsubject={programming},
	pdfauthor={Thiago Araujo},
	pdfkeywords={python},
	pdfsubject={programming},
	colorlinks=true,linkcolor=link,citecolor=link,urlcolor=link,linktocpage
}

\usepackage{moreverb} % This package improves the verbatim. Now we can write codes that considers 
%the TAB. Moreover, we can use numbered verbatim. Quite handy to write codes. 

\begin{document}

%%%%%%%%%%%%%%%%%%%%%%%%%%%%%%%%%%%%%%%%%%%%%%
%%%%%%%%%%%%%%%%%%%%%%%%%%%%%%%%%%%%%%%%%%%%%%
%%%%%%%%%%%%%%%%%%%%%%%%%%%%%%%%%%%%%%%%%%%%%%
%%%%%%%%%%%%%%%%%%%%%%%%%%%%%%%%%%%%%%%%%%%%%%


\title[Notes on Python]{Notes on Python 3}

\author{Thiago Araujo}

\email{\texttt{\href{thgr.araujo@gmail.com}{thgr.araujo@gmail.com}}} 

\keywords{Python}
%\subjclass[2020]{37K10, 82B20, 82B23}
%\date{\today}

\begin{abstract}
Here I collect some basic facts on python 3.

\bigskip

\noindent \textbf{Keywords:} Python
\end{abstract}

\maketitle

\setcounter{tocdepth}{2}
\tableofcontents

%%%%%%%%%%%%%%%%%%%%%%%%%%%%%%%%%%%%%%%%%%%%%%
%%%%%%%%%%%%%%%%%%%%%%%%%%%%%%%%%%%%%%%%%%%%%%
%%%%%%%%%%%%%%%%%%%%%%%%%%%%%%%%%%%%%%%%%%%%%%
%%%%%%%%%%%%%%%%%%%%%%%%%%%%%%%%%%%%%%%%%%%%%%

\section*{Introduction}

Here I follow~\cite{pythontutorial} and the course of Colt Steele at Udemy~\cite{csteele}.

%%%%%%%%%%%%%%%%%%%%%%%%%%%%%%%%%%%%%%%%%%%%%%
%%%%%%%%%%%%%%%%%%%%%%%%%%%%%%%%%%%%%%%%%%%%%%
%%%%%%%%%%%%%%%%%%%%%%%%%%%%%%%%%%%%%%%%%%%%%%
%%%%%%%%%%%%%%%%%%%%%%%%%%%%%%%%%%%%%%%%%%%%%%

%%%%%%%%%%%%%%%%%%%%%%%%%%%%%%%%%%%%%%%%%%%%%%
%%%%%%%%%%%%%%%%%%%%%%%%%%%%%%%%%%%%%%%%%%%%%%
%%%%%%%%%%%%%%%%%%%%%%%%%%%%%%%%%%%%%%%%%%%%%%
%%%%%%%%%%%%%%%%%%%%%%%%%%%%%%%%%%%%%%%%%%%%%%

\section{Data Structures}

\subsection{Lists}

Lists have the form
\begin{verbatim}
> lst = ["a", 1, [], 2.5] 
\end{verbatim}
and you can see that it admits several types of data, including other lists \verb|[ ]|. Moreover, lists are indexed, and the index starts with zero ``0''. The most import point is the methods we can use with lists, and it is the topic I want to discuss now. 

\subsubsection{Methods}

We have the following methods:

\begin{itemize}
	\item Miscellaneous
	\begin{itemize}
		\item index
		\item count
		\item reverse
		\item sort
		\item join\footnote{That, to be precise, is a string method}
		\item slicing
	\end{itemize}
	
	\item Adding data
	\begin{itemize}
		\item append
		\item extend
		\item insert
	\end{itemize}
	
	\item Removing data
	\begin{itemize}
		\item clear
		\item remove
		\item pop\footnote{ We also have the keyword del. It is not a method (and pop and remove). 
            Given a list L, The the keyword del is used as del L[n] where n is an index}.
	\end{itemize}
\end{itemize}

\paragraph{Miscellaneous}

Consider the lis
\begin{verbatim}
> lst = ["A", "B", "C", "D", "E", "E", "E", "E", "F"]
\end{verbatim}

We want to describe the methods we mentioned before. 

\# {\bf index(x, start, end)} This method gives the first instance where the value \verb|'x'| appears in the list. In the list we mentioned above, we have
\begin{verbatim}
> lst.index("A")
0
> lst.index("C")
2
> lst.index("E")
4
\end{verbatim}
Observe that string \verb|"E"| appears in 4 because it is the first time it appears in the list. For this reason, we can add the options to restrict the interval it search for a particular string. For example:
\begin{verbatim}
> lst.index("5",5,7)
5
\end{verbatim}
it gives the first time it appears in the list, starting at the value 5.

\# {\bf count(x)} This method returns the number of times \verb|x| appears in the list. For example:
\begin{verbatim}
> lst.count("A")
1
> lst.count("E")
4
\end{verbatim}
    
\# {\bf reverse()} It reverses the order of the list. For example:
\begin{verbatim}
> lst.reverse()
> lst
['F', 'E', 'E', 'E', 'E', 'D', 'C', 'B', 'A']
\end{verbatim}

\# {\bf sort()} It sorts the elements of the list. For example:
\begin{verbatim}
> lst.reverse()
> lst
['F', 'E', 'E', 'E', 'E', 'D', 'C', 'B', 'A'] 
> lst.sort()
> lst
['A', 'B', 'C', 'D', 'E', 'E', 'E', 'E', 'F']     
\end{verbatim}

\# {\bf join()}
Given a list as above, it makes a string with the entries of the list. The syntax is a bit different now. First I want to define how the entries will be patched, and inside the parenthesis we give the list. Using the list above, we have 
\begin{verbatim}
> "".join(lst)
'ABCDEEEEF'
> " ".join(lst)
'A B C D E E E E F'
> "_".join(lst)
'A_B_C_D_E_E_E_E_F'
> " and ".join(lst)
'A and B and C and D and E and E and E and E and F'
\end{verbatim}        

\# {\bf copy()} It creates a copy of the list. 
\begin{verbatim}
> lst2 = lst.copy()
> lst
['A', 'B', 'C', 'D', 'E', 'E', 'E', 'E', 'F']   
> lst2 
['A', 'B', 'C', 'D', 'E', 'E', 'E', 'E', 'F']
\end{verbatim}

\# {\bf slicing} It has notation given by \verb|[start:end:steps]|. The notation is a but obvious. Consider the list of numbers from 1 to 100, that is 
\begin{verbatim}
> num = list(range(1,101))
\end{verbatim}
now we consider the slice
\begin{verbatim}
> num[3:20]
[4, 5, 6, 7, 8, 9, 10, 11, 12, 13, 14, 15, 16, 17, 18, 19, 20]
\end{verbatim}
Observe that it does not include the number 3. We can consider steps of 4, then
\begin{verbatim}    
> num[3:20:4]
[4, 8, 12, 16, 20]
\end{verbatim}

\paragraph{Adding data}

Consider the list
\begin{verbatim}
> colors = ["red", "purple", "cyan", "yellow", "blue"]
\end{verbatim}

\# {\bf append(x)} It add the element x to the end of the list 
\begin{verbatim}
> colors.append("green")
> colors
['red', 'purple', 'cyan', 'yellow', 'blue', 'green']
\end{verbatim}

\# {\bf extend(x)} It extends the list by appending another list 
\begin{verbatim}
> colors.extend(["brown", "black", "white"])       
> colors
['red', 'purple', 'cyan', 'yellow', 'blue', 'green', 'brown', 'black', 'white']
\end{verbatim}        
        
\# {\bf insert(i, x)} It inserts the element \verb|x| before the element currently at position labeled by \verb|i|. For example, in our case, we want to insert \verb|"pink"| before \verb|'purple'|, which is labeled by \verb|i=1|. We do the following:
\begin{verbatim}
> colors.insert(1, "pink")       
> colors
['red', 'pink', 'purple', 'cyan', 'yellow', 'blue', 'green', 'brown', 'black', 'white']
\end{verbatim}        

\paragraph{Removing data}

Consider the list
\begin{verbatim}
> lst = ['red', 'pink', 'purple', 'cyan', 'yellow', 'blue']
\end{verbatim}

\# {\bf clear()} It does what it says it does.

\# {\bf remove(x)} It removes the first with that matches the entry \verb|x|.

\# {\bf pop([i])} It removes the item at the position \verb|i|, and returns it. For example
\begin{verbatim}
> lst.pop(2)
'purple'
> lst 
['red', 'pink', 'cyan', 'yellow', 'blue']
\end{verbatim}
and we can return to the original list with insert 
\begin{verbatim}
> lst.insert(2, 'purple')
['red', 'pink', 'purple', 'cyan', 'yellow', 'blue']
\end{verbatim}


\subsubsection{List comprehension}

\paragraph{Basic definitions}

It is a shorthand way to build new lists. It has the form
\begin{verbatim}
> [ _ _ _ for _ _ _ in _ _ _ ]
\end{verbatim}
For example:
\begin{verbatim}
> nums = [1, 2, 3, 4]
> lst = [x ** 2 for x in nums ]
[1, 4, 9, 16]
\end{verbatim}

We could also do it with a loop:
\begin{verbatim}
> nums = [1, 2, 3, 4]
> lst = []
> for j in nums:
   	squ = j ** 2

        lst.append(squ)
> print(lst)
\end{verbatim}

Colt made a mistake in the next example:

\begin{verbatim}
> friends = ['ashley', 'matt', 'michael']
> lst = [friend[0].upper() for friend in friends]
['A', 'M', 'M']
\end{verbatim}
Actually, he meant
\begin{verbatim}
> friends = ['ashley', 'matt', 'michael']
> lst = [friend[0].upper() + friend[1:] for friend in friends]
\end{verbatim}
This example is nice because it shows that we can slice the strings. See slice in our previous chapter. 

\paragraph{List comprehension with conditional logic} Here I want to understand important aspects of list comprehension.

{\bf \# for:} It has the form 
\begin{verbatim}
> [ _ _ _ for _ _ _ in _ _ _ if _ _ _ ]
\end{verbatim}
For example, even and odd numbers.
\begin{verbatim}
> numbers = list(range(1,11))
> evens = [num for num in numbers if num % 2 == 0]
> odds = [num for num in numbers if num % 2 != 0]
\end{verbatim}

Let me remove vowels in my name
\begin{verbatim}
> name = "Thiago Rocha Araujo"
> without = [char for char in name if char not in "aeiou"]
> "".join(without)
'Thg Rch Arj'
\end{verbatim}

We could also do it directly inside the join
\begin{verbatim}
> without = " ".join(char for char in name if char not in "aeiou")
\end{verbatim}

Another example from the course. It removes the vowels from the word amazing
\begin{verbatim}
> answer = [letter for letter in "amazing" if letter not in "aeiou"]
\end{verbatim}

Another example
\begin{verbatim}
> numbers = list(range(1,11))
> lst = [ num ** 2 if num % 2 == 0 else num ** 3 for num in numbers]
\end{verbatim}

There is an interesting example in Colt's course. It has solution given by 
\begin{verbatim}
answer = [name[len(name):0:-1] + name[0].lower for name in ["Elie", "Tim", "Matt"]]
\end{verbatim}
Also, \verb|name[::-1].lower()| gives the same result. It takes the names in the list, reverse and changes the capital to lower case. 

{\bf \# else:}

\begin{verbatim}
> [ _ _ _ _ if _ _ _ _ else _ _ _ _ for _ _ _ _ in _ _ _ _]
\end{verbatim}        
For example, we build a list with even squared and odd cubed
\begin{verbatim}
> numbers = list(range(1,11))
> lst = [ num ** 2 if num % 2 == 0 else num ** 3 for num in numbers]
\end{verbatim}

\subsubsection{Nested lists}

These are lists inside lists.

\paragraph{Nested loops} The important point here is how it is indexed. For example:
\begin{verbatim}    
> nested_list = [[1,2,3],[4,5,6],[7,8,9]]
> nested_list[1]
[4,5,6]
> nested_list[1][2]
6
\end{verbatim}
I can also print these values using a loop. Of course, I need two loops. 
\begin{verbatim}
> for l in nested_list:
> 	for val in l:
> 		print(val)
\end{verbatim}

\paragraph{Nested list comprehension} For the list
\begin{verbatim}
> nested_list = [[1,2,3],[4,5,6],[7,8,9]]
\end{verbatim}
let us create a list with squares.
\begin{verbatim}
> [[val ** 2 for val in l] for l in nested_list]
\end{verbatim}

Colt's example: tic tac toe
\begin{verbatim}
[["X" if num % 2 != 0 else "O" for num in range(3)] for val in range(3)]
\end{verbatim}

%%%%%%%%%%%%%%%%%%%%%%%%%%%%%%%%%%%%%%%%%%%%%%
%%%%%%%%%%%%%%%%%%%%%%%%%%%%%%%%%%%%%%%%%%%%%%
%%%%%%%%%%%%%%%%%%%%%%%%%%%%%%%%%%%%%%%%%%%%%%
%%%%%%%%%%%%%%%%%%%%%%%%%%%%%%%%%%%%%%%%%%%%%%

\subsection{Dictionary}

\subsubsection{Basics}

There are some limitations in the lists, so we need to introduce the dictionary. One particular type of limitation is a shopping card, where we need to give the name of the object we want to buy, the amount and price. So, we have three types of data, and we do't want to put 'em all together. For example:
\begin{verbatim}
> instructor = {
"name" : "Colt", 
"owns_dog" : True,
"num_courses" : 4, 
44 : "favorite_number"
}
\end{verbatim}
We can also build a dictionary with the \verb|dict()| function. 
\begin{verbatim}
> cat = dict(name="kitty", age=0.5)
\end{verbatim}
We can also retrieve the information with the brackets. For example, we can write \verb|instructor["name"]| gives \verb|"Colt"|.

\subsubsection{Looping} Consider the dictionary
\begin{verbatim}
> instructor = {
"name" : "Colt", 
"owns_dog" : True,
"num_courses" : 4, 
"favorite_language": "Python,
44 : "favorite_number"
}
\end{verbatim}
The function \verb|.values()| give the values we need, in this case, \verb|"Colt", True, "Python", "favorite_number"|, but it returns as a list (or tuples, I still don't know), that is 
\begin{verbatim}
> instructor.values()
dict_values(["Colt", True, 4, "Python", "favorite_number"])
\end{verbatim}

The keys are similar, but with \verb|.keys()|
\begin{verbatim}
> instructor.keys()
dict_keys(["name", "owns_dog", "num_courses", "favorite_language, 44])
\end{verbatim}

We can access both with \verb|.items()| and we get the tuples. If we want to loop over it, we'll need to give two entries, for example
\begin{verbatim}
> for k,v in instructor.items()
> 	print("{} and {}".format(k,v))
\end{verbatim}

One example of the course:
\begin{verbatim}
> donations = dict(sam=25.0, lena=88.99, chuck=13.0, 
... linus=99.5, stan=150.0, lisa=50.25, harrison=10.0)
> total_donations = 0
> for don in donations.values():
> 	total_donations += don
\end{verbatim}

Advanced solution 1: This solution uses a built-in function called \verb|sum()|. 
\begin{verbatim}
total_donations = sum(donation for donation in donations.values())
\end{verbatim}

Advanced solution 2: An even better solution using the same sum built-in function is just this nice little line:
\begin{verbatim}
total_donations = sum(donations.values())
\end{verbatim}

We can test if a dictionary has a key, we can use:
\begin{verbatim}
> "name" in instructor
True
> "phone" in instructor
False
\end{verbatim}


\subsubsection{Methods}

Fortunately, we don't have as many as in the lists. 
\begin{itemize}
	\item clear
	\item copy
	\item fromkeys
	\item get
	\item pop
	\item update
\end{itemize}

The first two methods are easy, so we start with the third. 

\# {\bf fromkeys} It is a way to create initial dictionaries where we can give entries later. For example, 
\begin{verbatim}
> new_user = {}.fromkeys(['name', 'score', 'email', 'profile bio'], 'unkown')
> new_user
{'name': 'unkown', 'score': 'unkown', 'email': 'unkown', 'profile bio': 'unkown'}
\end{verbatim}

\# {\bf get} It retrieves a key in an object and returns \verb|None| instead of a \verb|keyerror| if the key does not exist.
\begin{verbatim}
> d = dict(a=1, b=2, c=3) (this is equivalent to d = {"a":1, "b":2, "c":3})
> d["a"]
1
> d["b"]
2
> d.get("a")
1
> d.get("b") 
2
> d["teste"]
error
> print(d.get("teste"))
None
\end{verbatim}

\# {\bf pop} Contrary to the lists, now we need to provide an argument, the key we want to remove. 

\# {\bf popitem} It removes an arbitrary entry.

\# {\bf update} It updates the keys and values with another set of keys and values. For example:
\begin{verbatim}
> first = dict(a=1, b=2, c=3, d=4, e=5)
> second = {}
> second.update(first)
> second 
{'a': 1, 'b': 2, 'c': 3, 'd': 4, 'e': 5}
\end{verbatim}

We can also overwrite the data
\begin{verbatim}
> second["a"] = "AMAZING"
> second 
{'a': 'AMAZING', 'b': 2, 'c': 3, 'd': 4, 'e': 5}
\end{verbatim}

It does not erase the dictionary, but it adds the list. for example,
\begin{verbatim}
> first 
{'a': 1, 'b': 2, 'c': 3, 'd': 4, 'e': 5}
> third = dict(A="oranges", B="bananas")
> third.update(first)
> third
{'A': 'oranges', 'B': 'bananas', 'a': 1, 'b': 2, 'c': 3, 'd': 4, 'e': 5}
\end{verbatim}

But if I change one value, I can return to the original list with the update. For example:
\begin{verbatim}
> third["a"] = "apples"
> third
{'A': 'oranges', 'B': 'bananas', 'a': 'apples', 'b': 2, 'c': 3, 'd': 4, 'e': 5}
> third.update(first)
> third
{'A': 'oranges', 'B': 'bananas', 'a': 1, 'b': 2, 'c': 3, 'd': 4, 'e': 5}
\end{verbatim}


\subsubsection{Dictionary comprehension}

\paragraph{Loops}

The syntax is similar to the list syntax, but we have some tweaks.
\begin{verbatim}
> { _ _ _ : _ _ _ for _ _ _ in _ _ _ }
\end{verbatim}
For example
\begin{verbatim}
> numbers = dict(first=1, second=2, third=3)
> squared_numbers = {key: value ** 2 for key, value in numbers.items()}
> squared
{'first': 1, 'second': 4, 'third': 9}
\end{verbatim}

A second example
\begin{verbatim}
> squared_num = {num: num ** 2 for num in [1,2,3,4,5]}
> squared_num
{1: 1, 2: 4, 3: 9, 4: 16, 5: 25}
\end{verbatim}

Another example
\begin{verbatim}
> str1 = "ABCDE"
> str2 = "123"
> combo = {str1[i]:str2[i] for i in range(0,len(str2))}
> combo
{'A': '1', 'B': '2', 'C': '3'}
\end{verbatim}

\paragraph{Conditional logic} The syntax is given by
\begin{verbatim}
> { _ _ _ : ( _ _ _ if _ _ _ else _ _ _ ) for _ _ _ in _ _ _ }
> { ( _ _ _ if _ _ _ else _ _ _ ) : _ _ _ for _ _ _ in _ _ _ }
\end{verbatim}
For example
\begin{verbatim}
> num_list = list(range(0,21))
> parity = {num:("even" if num % 2==0 else "odd") for num in num_list}
> parity 
\end{verbatim}

There is a interesting function for list comprehension, the function \verb|zip()| that works as follows. Suppose we have two lists 
\begin{verbatim}
> list1 = ["CA", "NJ", "RI"]
> list2 = ["California", "New Jersey", "Rhode Island"]
> answer = {list1[i] : list2[i] for i in range(3)}
> answer2 = dict(zip(list1,list2))  
> answer == answer2
True
\end{verbatim}

Another example for the function \verb|zip()|
\begin{verbatim}
> person = [["name", "Jared"], ["job", "Musician"], ["city", "Bern"]]
> keys = [person[i][0] for i in range(3)]
> values = [person[i][1] for i in range(3)]
> answer = dict(zip(keys, values))
> answer
{'name': 'Jared', 'job': 'Musician', 'city': 'Bern'}
\end{verbatim}

Another example with \verb|dict|
\begin{verbatim}
> ans1 = {char:0 for char in 'aeiou'}	
> ans2 = dict.fromkeys("aeiou", 0)
> ans1 == ans2
True
\end{verbatim}

%%%%%%%%%%%%%%%%%%%%%%%%%%%%%%%%%%%%%%%%%%%%%%
%%%%%%%%%%%%%%%%%%%%%%%%%%%%%%%%%%%%%%%%%%%%%%
%%%%%%%%%%%%%%%%%%%%%%%%%%%%%%%%%%%%%%%%%%%%%%
%%%%%%%%%%%%%%%%%%%%%%%%%%%%%%%%%%%%%%%%%%%%%%

\subsection{Tuples \& Sets}

\subsubsection{Tuples}

It is an ordered collection or groups and items. It seems like lists but there are a few diffrences. First, we cannot change it after defined, and the syntax is different.
\begin{verbatim}
> x = (1, 2, 3)	
> x[0] = "change me"
ERROR
\end{verbatim}
We use them because they are lighter than lists. The code is safer from bugs. In the function \verb|items()|, the return is a tuple. We access their data with the square bracket as well, similar to the list. We can create elements with the function \verb|tuples|.

Tuples can be used as keys in dictionaries. For example:
\begin{verbatim}
> locations = { 
... (latitude1, longitude1) : "Sao Paulo",
... (latitude2, longitude2) : "Pohang",
... (latitude3, longitude3) : "Bern",
}
\end{verbatim}
Then we can access this data with \verb|locations[(latitude1, longitude1)] = "Sao Paulo"|. 

Looping over tuples is equal to looping over lists. That is all. Moreover, we also have \verb|count()| and \verb|index()| as methods. We can also slice the lists, so things are similar. 

%%%%%%%%%%%% 

\subsubsection{Sets}

There are like formal mathematical sets. So, it is a collection of data without any structure and the elements are unique. Since there are no order, we cannot use indices to access the elements. The syntax uses curly brackets. 
\begin{verbatim}
> s = set({1,2,3,4,5,5})
> r = {1,2,3,4,5}	
> s
{1,2,3,4,5}
> s == r
True
> 5 in s
True
\end{verbatim}

\paragraph{Methods}

There are some important methods to discuss with sets. 

\# {\bf add(`x').} It adds a new element \verb|x|. 

\# {\bf remove(`x').} It removes the element \verb|x|. 

\# {\bf discard(`x').} It removes the element \verb|x|, but does not give an error if the element \verb|x| is not present in the set.  

\# {\bf copy().} It creates a copy of the set. 

\# {\bf clear().} It is obvious as well

\paragraph{Mathematical Sets}

These are the data I am interested in. Now we can consider union, intersection and other mathematical operations. 

Consider two classes:
\begin{verbatim}
> math_students = {"Bashir", "Odo", "Sisko", "Worf", "Picard", "Quark"}
> biology_students = {"The Doctor", "Sisko", "Worf", "Seven of Nine", "Data"}
> union_students = math_students | biology_students
{'Seven of Nine', 'Sisko', 'Worf', 'Quark', 'Data', 
'The Doctor', 'Picard', 'Odo', 'Bashir'}
> intersection_students = math_students & biology_students
{'Worf', 'Sisko'}
\end{verbatim}

\paragraph{Set comprehension}

It has the form 
\begin{verbatim}
> { _ _ _ for _ _ _ in _ _ _ }
\end{verbatim}

For example, 
\begin{verbatim}
> { x ** 2 for x in range(10) }
{0, 1, 64, 4, 36, 9, 16, 49, 81, 25}
\end{verbatim}


%%%%%%%%%%%%%%%%%%%%%%%%%%%%%%%%%%%%%%%%%%%%%%
%%%%%%%%%%%%%%%%%%%%%%%%%%%%%%%%%%%%%%%%%%%%%%
%%%%%%%%%%%%%%%%%%%%%%%%%%%%%%%%%%%%%%%%%%%%%%
%%%%%%%%%%%%%%%%%%%%%%%%%%%%%%%%%%%%%%%%%%%%%%

\section{Functions}

It is process for executing a task. It is useful for executing similar procedures over and over. It is helpful to avoid repetition. DRY = Don't Repeat Yourself. 

\subsubsection{Defining a function}

Here is the syntax 
\begin{verbatim}
> def name_of_the_function():
>     # block of code
\end{verbatim}
For example, 
\begin{verbatim}
> def say_hi():
>     print("Hi")
\end{verbatim}

We can return values with \verb|return|, that is
\begin{verbatim}
> def say_hi():
>     return "Hi"
\end{verbatim}
The \verb|return| exists the function. With return we can use the result in a variable. 

Let us write a function that will toss a coin for us. 
\begin{verbatim}
> from random import random
> def flip_coin():
>     # generate a random number
>     r = random()
>     if r > 0.5:
>         return "Heads"
>     else:
>         return "Tails"	
\end{verbatim}

\subsubsection{Parameters}

Now we need to understand how to define functions which accepts inputs. It is very easy. 
\begin{verbatim}
> def square(var):
>     return f(var)
\end{verbatim}

For example, with the Happy Birthday:
\begin{verbatim}
> def happy(name):
>     print("Happy birthday to you")
>     print("Happy birthday to you")
>     print(f"Happy birthday dear {name}")
>     print("Happy birthday to you")
\end{verbatim}

One can also consider more than 1 argument:
\begin{verbatim}
> def multiply(a,b):
>     return a * b
\end{verbatim}

We would also like to have default parameter. For example, if we do not insert a parameter, we execute some pre-defined parameter. The example that Colt gives is interesting, we want a default value for the exponent. For example, if we do not specify, it takes the square. It is defined as follows
\begin{verbatim}
> def exp(num, power=2):
>     return num ** power
> print(exp(2,3))
8
> print(exp(2))
4
\end{verbatim}
If we know the parameters, we can specify the arguments and the order does not matter anymore, in the example above, we can do
\begin{verbatim}
> def exp(num, power=2):
>     return num ** power
> print(exp(2,3))
8
> print(exp(power = 3, num = 2))
8
\end{verbatim}

A nice example of a function. Find the intersection {\bf list}
\begin{verbatim}
> def intersection(lst1, lst2):
>     return list(set(lst1) & set(lst2))
\end{verbatim}


\paragraph{Exercise} (Colt's course): The three codes below are equivalent:

\# Code 1: 
\begin{verbatim}
> def speak(animal="dog"):
>     if animal == "pig":
>         return "oink"
>     elif animal == "duck":
>          return "quack"
>     elif animal == "cat":
>         return "meow"
>     elif animal == "dog":
>         return "woof"
>     else:
>         return "?"
\end{verbatim}

\# Code2: 
\begin{verbatim}
> def speak(animal="dog"):
>     noises = {"dog": "woof", "pig": "oink", "duck": "quack", "cat": "meow"}
>     noise = noises.get(animal)
>     if noise:
>         return noise
>     return "?"
\end{verbatim}

\# Code3:
\begin{verbatim}
> def speak(animal='dog'):
>     noises = {'pig':'oink', 'duck':'quack', 'cat':'meow', 'dog':'woof'}
>     return noises.get(animal, '?')
\end{verbatim}
The third code uses the default value of \verb|get()|.

\subsubsection{Scope} 

Parameters are now always defined everywhere in the code, it is a topic called scope, and we can see more about in lecture 160 of Colt~\cite{csteele}. See also here~\cite{rpython}.

\subsubsection{Documentation} 

We can explain what our functions do with the triple quotation marks, \verb|""" message """|. 
The nice thing about it is that we can access this message with the syntax \verb|my_function.__doc__|. For example, 
\begin{verbatim}
> def my_function():
>     """Silly function that explains something"""
>     print("hello")
>
> my_function.__doc__
'Silly function that explains something'
\end{verbatim}

\subsubsection{\(\ast\) args} It is special operator. 
\begin{verbatim}
> def sum_all_nums(num1, num2, num3):
>      return num1 + num2 + num3
\end{verbatim}
But we would like to do this sum for an arbitrary number of values, but hard coding is silly. We can do it with \verb|*-args| operator. The arguments define a tuple. 

\begin{verbatim}
> def sum_all_nums(*args):
>      total = 0
>      for num in args:
>            total += num
>      return total
\end{verbatim}
Observe that the \verb|*| needs to be present, and the \verb|args| is just a name. That is
\begin{verbatim}
> def sum_all_nums(*nums):
>      total = 0
>      for num in nums:
>            total += num
>      return total
\end{verbatim}
also works. 


\subsubsection{\(\ast\ast\)-kwargs} It gathers remaining arguments as keys in dictionaries. For example
\begin{verbatim}
> def favorite_food(**kwargs):
>     for person, food in kwargs.items():
>         print(f"{person}'s favorite food is {food}")	
\end{verbatim}
Then I can run this function with some examples:
\begin{verbatim}
> favorite_food("thiago"="hamburger", "aline"="cake")
\end{verbatim}

A code that combines words, see exercise 58.
\begin{verbatim}
> def combine_words(name, **kwargs):
>     if "prefix" in kwargs.keys():
>         return "{}{}".format(kwargs["prefix"],name)
>     elif "suffix" in kwargs.keys():
>         return "{}{}".format(name,kwargs["suffix"])
>     return name
\end{verbatim}

\subsubsection{Ordering parameters} Remember that this order is important
\begin{enumerate}
	\item parameters
	\item \(*\)-args
	\item default parameters	
	\item \(*\)-kwargs
\end{enumerate}

\subsubsection{Lists and Tuples unpacking} We want to execute a function that takes its values in a list or tuple. 
\begin{verbatim}
> def sum_all_nums(*nums):
>      total = 0
>      for num in nums:
>            total += num
>      return total
\end{verbatim}
But now, suppose we want to use the following list and tuple in the function above
\begin{verbatim}
> num1 = [1, 2, 3, 4, 5, 6]
> num2 = (1, 2, 3, 4, 5, 6)
\end{verbatim}
If we try to apply the function on \verb|num1| and \verb|num2|, we get an error because they are a list and a tuple respectively. 
We we need to do is to unpack these objects. 

\subsubsection{Dictionary unpacking} 
\begin{verbatim}
> def display_names(first, second):
>     print(f"{first} says hello to {second}")
\end{verbatim}
We get an error if we try to use this function with a dictionary, for example
\begin{verbatim}
> names = {"first" : "Thiago", "second" : "Aline"}
> display_name(names)
> Error
> display_name(**names)
\end{verbatim}

See this example
\begin{verbatim}
> def add_and_multiply_numbers(a,b,c, **kwargs)
>     print(a + b * c)
>     print("OTHER STUFF. . .")
>     print(kwargs)
\end{verbatim}
Now we can apply to the dictionary
\begin{verbatim}
> data = dict(a=1, b=2, c=3, d=55, name="Tony")
> add_and_multiply_numbers(**data, cat="Blue")
\end{verbatim}


%%%%%%%%%%%%%%%%%%%%%%%%%%%%%%%%%%%%%%%%%%%%%%
%%%%%%%%%%%%%%%%%%%%%%%%%%%%%%%%%%%%%%%%%%%%%%
%%%%%%%%%%%%%%%%%%%%%%%%%%%%%%%%%%%%%%%%%%%%%%
%%%%%%%%%%%%%%%%%%%%%%%%%%%%%%%%%%%%%%%%%%%%%%

\subsection{Lambdas}

These are special functions. Let us start with the usual definition of functions. 
\begin{verbatim}
> def square(num):
>     return num * num
\end{verbatim}
The syntax for the \verb|lambda| is given by
\begin{verbatim}
> square2 = lambda num: num * num
> add = lambda a,b: a + b
\end{verbatim}
The idea of this function is that we want to use a function that is applied in a specific place. 

\subsubsection{map}

It is a standard function that accepts at least two arguments, a function and an iterable (something that can be iterated, strings, lists, dictionaries, sets, tuples). It runs a \verb|lambda| for each value in the iterable. 
\begin{verbatim}
> num = [2, 4, 6, 8, 10]
> doubles = map(lambda x: x * 2, nums) # it is a ____
> doubles2 = list(map(lambda x: x * 2, nums))
\end{verbatim}
Or we can define the function separately
\begin{verbatim}
> def double(x): return x*2
> doubles = map(double, nums)
> doubles2 = list(map(lambda x: x * 2, nums))
\end{verbatim}

Another example
\begin{verbatim}
> names = [{'first':'Thiago', 'last':'Araujo'}, {'first':'Vanessa', 'last':'Araujo'}, 
...{'first':'Julia', 'last':'Araujo'}]
> firstnames = list(map(lambda x : x['first'], names ))
\end{verbatim}

Another example from Colt's course
\begin{verbatim}
> def extract_full_name(names):
>     return list(map(lambda x : "{} {}".format(x['first'], x['last']), names))
\end{verbatim}

\subsubsection{filters}

We can filter out some values, that is, there is a lambda for each item in the iterable. 
\begin{verbatim}
> lst = [1, 2, 3, 4]
> evens = list(filter(lambda x: x % 2 ==0, lst ))
\end{verbatim}

Colt's example of users is interesting. Suppose we want to collect the inactive users of a social media. 
\begin{verbatim}
> users = [
>     {"username": "samuel", "tweets": ["I love cake", "I love pie", "hello world!"]},
>     {"username": "katie", "tweets": ["I love my cat"]},
>     {"username": "jeff", "tweets": []},
>     {"username": "bob123", "tweets": []},
>     {"username": "doggo_luvr", "tweets": ["dogs are the best", "I'm hungry"]},
>     {"username": "guitar_gal", "tweets": []}
> ]
> #extract inactive users using filter:
> inactive_users = list(filter(lambda u: not u['tweets'], users))
> inactive_users
> [{'username': 'jeff', 'tweets': []}, {'username': 'bob123', 'tweets': []}, 
{'username': 'guitar_gal', 'tweets': []}]
\end{verbatim}

But now, suppose we want to create a list of the usernames only, then we need to combine this filter with a map. 
\begin{verbatim}
> inactive_users_names = list( map(lambda user: user['username'].upper(), 
... filter(lambda u: not u['tweets'], users)))
\end{verbatim}

We could use list comprehension to solve both problems above. 
\begin{verbatim}
> #extract inactive users using list comprehension:
> inactive_users2= [user for user in users if not user["tweets"]]
> extract usernames of inactive users w/ list comprehension
> usernames2 = [user["username"].upper() for user in users if not user["tweets"]]
\end{verbatim}


%%%%%%%%%%%%%%%%%%%%%%%%%%%%%%%%%%%%%%%%%%%%%%
%%%%%%%%%%%%%%%%%%%%%%%%%%%%%%%%%%%%%%%%%%%%%%
%%%%%%%%%%%%%%%%%%%%%%%%%%%%%%%%%%%%%%%%%%%%%%
%%%%%%%%%%%%%%%%%%%%%%%%%%%%%%%%%%%%%%%%%%%%%%

\subsection{Built-in functions}

Some important built-in functions.

\subsubsection{all} It returns True if all elements of the iterable are truthy or if the iterable is empty. 

\subsubsection{any} It returns True if any element of the iterable is truthy. If the iterable is empty, it returns false. 

\subsubsection{Generator expressions \& .getsizeof} These functions do not need the bracket. For example
\begin{verbatim}
> names = [Thiago", "Thomas", "Theo", "Thales"]
> all([name[0] == 'T' for name in names])
True
> all(name[0] == 'T' for name in names)
True
\end{verbatim}
It makes the code lighter. So, if we do not need to build a list, we should use this generator expression. Run the following code to see the size
\begin{verbatim}
> import sys
> list_comp = sys.getsizeof([x * 10 for x in range(1000)])
> gen_exp = sys.getsizeof(x * 10 for x in range(1000))
> 
> print("To do the same thing, it takes. . . ")
> print(f"List comprehension: {list_comp} bytes")
> print(f"Generator expression: {gen_exp} bytes") 
\end{verbatim}

\subsubsection{sorted} It is a built-in function that works for more general iterables, and not only in lists. It does not change the iterable .
\begin{verbatim}
> nums = [4, 6, 1, 30, 155]
> sorted(nums)
[1, 4, 6, 30, 155]
> nums
[4, 6, 1, 30, 155]
> nums.sort()
> nums
[1, 4, 6, 30, 155]
\end{verbatim}
We can also give arguments
\begin{verbatim}
> nums = [4, 6, 1, 30, 155]
> sorted(nums, reverse=True)
[155, 30, 6, 4, 1]
\end{verbatim}

But it is stronger than that. Suppose we want to organize the social media. 
\begin{verbatim}
> users = [
>     {"username": "samuel", "tweets": ["I love cake", "I love pie", "hello world!"]},
>     {"username": "katie", "tweets": ["I love my cat"]},
>     {"username": "jeff", "tweets": []},
>     {"username": "bob123", "tweets": []},
>     {"username": "doggo_luvr", "tweets": ["dogs are the best", "I'm hungry"]},
>     {"username": "guitar_gal", "tweets": []}
> ]
>
> sorted(users, key= lambda user: user['username']) # alphabetically
> sorted(users, key= lambda user: len(user['tweets'])) # activity
\end{verbatim}

Songs, by play count
\begin{verbatim}
> songs = [
> {'title': 'happy birthday', 'playcount': 1},
> {'title': 'Survive', 'playcount': 6},
> {'title': 'YMCA', 'playcount':99},
> {'title': 'Toxic', 'playcount':31}
> ]
>
> sorted(songs, key = lambda s: s['playcount'])
\end{verbatim}

\subsubsection{Max \& Min}

\paragraph{Max} It returns the maximum value.

\paragraph{Min} It returns the minimum value.

These functions are a bit obvious, but we can also give other arguments. 
\begin{verbatim}
> names = ['Arya', 'Samson', 'Dora', 'Tim', 'Ollivander']
> max(names, key=lambda n: len(n))
'Ollivander'
> max(names)
'Tim'
\end{verbatim}

\subsubsection{Reversed}

It returns the reverse of the iterable object. For example
\begin{verbatim}
> for x in reversed(range(1,10)):
>	print(x)
\end{verbatim}
If we want to use reversed with strings using the join function
\begin{verbatim}
> str[::-1]
> ''.join(list(reversed(str)))
\end{verbatim}

\subsubsection{len} We already know how it works. But it works calling the \verb|__len__()| method. 

\subsubsection{abs} Returns the absolute value of a number. There is also the function \verb|fabs| that we need to import with \verb|math| that does the same thing, but it converts to a float first. 

\subsubsection{sum} It is a sum of the elements of the iterable. We can also say how it starts. 
\begin{verbatim}
> sum([1,2,3])
6 
> sum([1,2,3], 10)
16
\end{verbatim}

\subsubsection{round} It approximates the number and gives the precision. 
\begin{verbatim}
> round(3.5122, 2)
3.51
> round(3.5122)
4
> round(3.4122)
3
\end{verbatim}

\subsubsection{zip} This function makes an iterator that aggregates elements from each of the iterables. Imagine that we have some lists. This function pairs elements of these lists. 
\begin{verbatim}
> nums1 = [1,2,3,4,5]
> nums2 = [6, 7, 8, 9, 10]
> z = zip(nums1, nums2)
\end{verbatim}

Complex example. Consider the lists
\begin{verbatim}
> midterms = [80, 91, 78]
> final = [98, 89, 53]
> students = ['dan', 'ang', 'kate']
> grades = {t[0]: max(t[1], t[2]) for t in zip(students, midterms, final)}
\end{verbatim}

Another solution using \verb|lambdas|
\begin{verbatim}
> final_grades = dict( zip( students, map( lambda 
... pair: max(pair), zip(midterms, finals))))
\end{verbatim}



%%%%%%%%%%%%%%%%%%%%%%%%%%%%%%%%%%%%%%%%%%%%%%
%%%%%%%%%%%%%%%%%%%%%%%%%%%%%%%%%%%%%%%%%%%%%%
%%%%%%%%%%%%%%%%%%%%%%%%%%%%%%%%%%%%%%%%%%%%%%
%%%%%%%%%%%%%%%%%%%%%%%%%%%%%%%%%%%%%%%%%%%%%%

\section{Modules}

Modules is a way to use other python files in other files. 

\subsection{Built-in modules}

Python has several build-in modules, see here in the Python Module Index~\cite{python:index}. We can use them with the syntax 
\begin{verbatim}
> import MODULE
\end{verbatim}
For example, the \verb|random| package can be used as
\begin{verbatim}
> import random 
> my_list = ["apple", "banana", "durian", "pear"]
> random.choice(my_list)
> random.randint(1,100)
> random.shuffle(my_list)
> my_list
\end{verbatim}

We can change how we call the module using the \verb|as|
\begin{verbatim}
> import MODULE as NEW_NAME
\end{verbatim}
For example
\begin{verbatim}
> import random as rand
> rand.choice(["apple", "banana", "durian", "pear"])
> rand.randint(1,100)
> my_list = ["apple", "banana", "durian", "pear"]
> rand.shuffle(my_list)
> my_list
\end{verbatim}

We can also call just the function in the module with the syntax
\begin{verbatim}
> from MODULE import *, **
\end{verbatim}
For example
\begin{verbatim}
> from random import choice, randint
> my_list = ["apple", "banana", "durian", "pear"]
> choice(my_list)
> randint(1,100)
\end{verbatim}

We can also change how we call the module using the \verb|as|
\begin{verbatim}
> from MODULE import * as NEW_NAME_1, ** as NEW_NAME_2
\end{verbatim}
For example
\begin{verbatim}
> from random import randint as rdi, choice as cho
> rdi(1,100)
> cho([1,2,3,4,5,6,7])
\end{verbatim}

There is a nice exercise in the lectures of Colt. It checks if a given string is a python key. That is
\begin{verbatim}
> from keyword import iskeyword
> def contains_keyword(*args):
>      """Define a function that check if any word in the arguments 
... is a python keywords"""
>      # First, I can transform the arguments in a tuple 
... (because it is lighter than a list) of True and False. It is a tuple comprehension 
... with conditional logic
>      entries = tuple(True if iskeyword(ent) else False for ent in args)
>      return any(entries)
\end{verbatim}

%%%%%%%%%%%%%%%%%%%%%%%%%%%%%%%%%%%%%%%%
%%%%%%%%%%%%%%%%%%%%%%%%%%%%%%%%%%%%%%%%

\subsection{Custom modules}

It is similar to the previous case, but now I can write the module. We use two python files, and we need to call it. Given the files \verb|file1| and \verb|file2|, we have define our functions in \verb|file1| and we use these definitions in \verb|file2| with the syntax 
\begin{verbatim}
> import file1
> file1.my_function()
\end{verbatim}

We can ignore the code with the \verb|__name__| variable. in particular, I need to add 
\begin{verbatim}
if __name__ == "__main__":
\end{verbatim}
Suppose that we have two python files \verb|file1| and \verb|file2|. In \verb|first1.py|, we write
\begin{verbatim}
>>> import file2
>>> print(__name__)
>>> print(file2.__name__)
\end{verbatim}
and the result is 
\begin{verbatim}
__main__
file2
\end{verbatim}

If we write the code in \verb|file2|
\begin{verbatim}
>>> print(__name__)
\end{verbatim}
Executing the \verb|file1| now gives
\begin{verbatim}
file2
__main__
file2
\end{verbatim}
that is, the code executes \verb|file2| and then executes \verb|file2|. We don't want to execute \verb|file2|, we just want to use some of the predefined functions, so we can avoid this with the following line in \verb|file2|
\begin{verbatim}
>>> if __name__ == "__main__":
>>>     print(__name__)
\end{verbatim}

\subsection{External modules}

We install packages with pip. We can search for packages here in the website \href{https://pypi.org/search/}{pypi.org}. The syntax is 
\begin{verbatim}
> python3 -m pip install NAME_OF_THE_PACKAGE
\end{verbatim}

As an interesting example, use the module \verb|termcolor|
\begin{verbatim}
> from termcolor import colored
> print(dir(termcolor)) # it gives the attributes of the package
> help(termcolor) # it explains the package
> print(colored("Hi THERE!", color="magenta", on_color="on_cyan", attrs=["blink"]))
\end{verbatim}

%%%%%%%%%%%%%%%%%%%%%%%%%%%%%%%%%%%%%%%%
%%%%%%%%%%%%%%%%%%%%%%%%%%%%%%%%%%%%%%%%
%%%%%%%%%%%%%%%%%%%%%%%%%%%%%%%%%%%%%%%%
%%%%%%%%%%%%%%%%%%%%%%%%%%%%%%%%%%%%%%%%

\section{Object Oriented Programming}

Object Oriented Programming (OOP) is about creating representation of things in the real world using codes. We do it using {\bf classes} and {\bf objects}.

\paragraph{\(\#\) \bf class} Blueprints for objects. Classes can contain methods (functions) and attributes (similar to keys in a dictionary). 

\paragraph{ \(\#\) \bf instance} - objects that are constructed from a class blueprint that contain their class's methods and properties.

\subsection{Definitions}

With OOP we can classify and organize our codes. 

\subsubsection{Example} Suppose we want to make a poker game. We could have the following entities: \(\bullet\) Game, \(\bullet\) Player, \(\bullet\) Card, \(\bullet\) Deck, \(\bullet\) Hand, \(\bullet\) Chip, \(\bullet\) Bet and so on. We could hard code all these entities, but it is oftentimes convenient to define classes and work with methods associated to these classes. 

I can consider a different explanation, aimed for mathematicians. Suppose I want to work with matrices. We could hard code these objects and hard code all operations associated to them, but it is better to define the class of matrices, and methods associated to them. These methods can be the transposition, trace and so on. 

\subsubsection{Defining a class}

Suppose we want to create a game. We want to build a class for users. The syntax is the following:
\begin{verbatim}
> class User: # Classes are oftentimes capitalized
>    def __init__(self, first, last, age):
>        self.name = first
>        self.last = last
>        self.age = age
\end{verbatim} 

We always use the syntax \verb|__init__| in the definition of the class. The \verb|self| is a dummy variable, but it is traditionally written as \verb|self| and it reffers to the objects (instances) themselves. Python calls the \verb|__init__| method whenever we create an instance of class (instatiate).

Now we can build these users using the methods, that is 
\begin{verbatim}
> user1 = User('Joe', 'Smith', 68)
> user2 = User('Blanka', 'Lopez', 41)
\end{verbatim}

And finally, we can retried the information of the users using the methods associated to them. For example, the code
\begin{verbatim}
> print(user1.first, user1.last)
> print(user2.first, user2.last)
\end{verbatim}
prints the names. Observe that the method does not use the brackets \verb|()|.

Another example is the following. Consider the class \verb|Comments| in a social network. That is 
\begin{verbatim}
> class Comments:
>      def __init__(self, username, text, likes=0) # likes default value is 0. 
>      self.username = username	
>      self.text = text
>      self.likes = likes
\end{verbatim}
Then, we run this code as
\begin{verbatim}
> c = Comment("davey123", 'lol you\'re so silly', 3)
> print(c.username)
> print(c.text)
> print(c.likes)
\end{verbatim}

\begin{shaded}
Here I need to comment something important: The underscores have meaning. In particular, we have things in the form: 
\begin{itemize}
	\item \verb|_name| : just one underscore in front of a method is a message to other developers. It says that this method whould be private, although python does not have full fledged secrete methods. It is basically a method to be used inside the definition of the class. 
	\item \verb|__name__|: These are special methods of python. Leave them alone. 
	\item \verb|__name| : Name mangling. This is something I will eventually learn somewhere. Google name mangling to understant a bit before any formal definition.
\end{itemize}
\end{shaded}	



\subsubsection{Instance Methods}

Here I want to consider some instance methods. It basically defines the methods inside these classes, and they act on the objects we define. 

In this case, we use them with the syntax
\begin{verbatim}
> object.my_method(argument)
\end{verbatim}

For example, let us keep with the \verb|User| class we defined above. Let us define some methods. 
\begin{verbatim}
> class User: # Classes are oftentimes capitalized
>    def __init__(self, first, last, age):
>        self.name = first
>        self.last = last
>        self.age = age
>
>    def full_name(self): # full_name method
>        return f"{self.first} {self.last}"
>
>    def initials(self): # initials method
>        return f"{self.first[0]} {self.last[0]}"
>
>    def likes(self, thing): # likes method. This methods has an argument
>        return f"{self.first} likes {thing}"
>
>    def is_senior(self): # Is senhor method has conditionals
>        return self.age >= 65    
>
>    def birthday(self):
>        self.age +=1
>        return f"Happy {self.age}th, {self.first}"
\end{verbatim}

Now, we can run these methods. First we need to define the users. 
\begin{verbatim}
> user1 = User("Joe", "Smith", 68)        
> user2 = User("Blanka", "Lopez", 41)
\end{verbatim}
Then, 
\begin{verbatim}
> print(user2.full_name())
> print(user1.likes("Ice Cream"))
>
> print(user1.initials())
> print(user2.initials())
>
> print(user2.is_senior())
> print(user1.age)
> print(user1.birthday())
> print(user1.age)	
\end{verbatim}

Another example is the bank account. See
\begin{verbatim}
> class BankAccount:
>     def __init__(self, owner, balance=0.0):
>         self.owner = owner
>         self.balance = balance
> 	
>     def deposit(self, add):
>         self.balance += add
>         return self.balance 
>
>     def withdraw(self, rem):
>         self.balance -= rem
>         return self.balance
> 
> acct = BankAccount("Darcy")
> 
> print(acct.owner)
> print(acct.balance)
> print(acct.deposit(10))
> print(acct.withdraw(3))
> print(acct.balance)	
\end{verbatim}


%%%%%%%%%%%%%%%%%%%%%%%%%%%%%%%%%%%%%%%%%%
%%%%%%%%%%%%%%%%%%%%%%%%%%%%%%%%%%%%%%%%%%

\subsubsection{Class attributes}

We now want to define attributes for the classes themselves, and not only for the instances. Let's come back to the User example. 
\begin{listing}{1}
class User:

    active_users = 0 # We define a class attribute

    def __init__(self,first,last,age):
        self.first = first
        self.last = last
        self.age = age
        User.active_users += 1

    def logout(self): 
        User.active_users -= 1 
        return f"{self.first} has logged out"  

    def full_name(self):
        return f"{self.first} {self.last}"

    def initials(self):
        return f"{self.first[0]}.{self.last[0]}."

    def likes(self, thing):
        return f"{self.first} likes {thing}"

    def is_senior(self):
        return self.age >= 65    

    def birthday(self):
        self.age +=1
        return f"Happy {self.age}th, {self.first}"
\end{listing}

In the case above, \verb|active|users| is a class attribute. Whenever a new user logs in or logs out we need to update the value of this attribute. We can test the code above with 

\begin{listingcont}{1}
print(User.active_users)

user1 = User("Joe", "Smith", 68)        
user2 = User("Blanka", "Lopez", 41)

print(User.active_users)

print(user2.logout())

print(User.active_users)
\end{listingcont}

Class attributes can be used as validation. For example, suppose we have a class for pets, and we want to forbid certain animals, for example, an alligator as a pet, then we can do the following:
\begin{listing}{1}
class Pet:

allowed = ['cat', 'dog', 'fish', 'rat']

def __init__(self, name, species):
	if species not in Pet.allowed:
		raise ValueError(f"You can't have a {species} pet")
	self.name = name
	self.species = species

def set_species(self, species):
	if species not in Pet.allowed:
		raise ValueError(f"You can't have a {species} pet")
	self.species = species	
\end{listing}

The second method, \verb|set_species| changes the species of our pet, but within the allowed class. If we deactivate the conditional part, we could change it the species of our pet to an animal that is not in the list of allowed pets. 

\begin{listingcont}
tonny = Pet("Tonny", "cat")

print(Pet.allowed)
print(tonny.allowed)

print(tonny.species)
new_species = tonny.set_species('dog')
print(tonny.species)

print(tonny.allowed == Pet.allowed)	
\end{listingcont}

Another example is the following:
\begin{listing}{1}
class Chicken:

    species = ['Partridge Silkie', 'Speckled Sussex']
    total_eggs = 0

    def __init__(self, name, species, eggs=0):
        # if species not in Chicken.allowed:
        #     raise ValueError(f"This is not a known {species}")
        self.name = name
        self.species = species
        self.eggs = eggs

    def lay_eggs(self):
        self.eggs += 1
        Chicken.total_eggs += 1

c1 = Chicken("Alice", 'Partridge Silkie')
c2 = Chicken("Amelia", 'Speckled Sussex') 

print(Chicken.total_eggs)
c1.lay_eggs()
print(Chicken.total_eggs)
c1.lay_eggs()
c1.lay_eggs()
print(Chicken.total_eggs)	
\end{listing}

%%%%%%%%%%%%%%%%%%%%%%%%%%%%%%%%%%%%%%%%%%
%%%%%%%%%%%%%%%%%%%%%%%%%%%%%%%%%%%%%%%%%%

\subsubsection{Class methods}

Now we want to define methods which are concerned with the class as as whole, and not with the instances themselves. We have encountered class methods before. The \verb|fromkeys| method is an example. 

Let us return to our old friend class of Users. Then 
\begin{listing}{1}
class User: 

    active_users = 0 # We define a class attribute

    @classmethod
    def display_active_users(cls):
        return f"There are currently {cls.active_users} active users."

    @classmethod
    def from_string(cls, data_str):
        first,last,age = data_str.split(",")        
        return cls(first,last,int(age))

    def __init__(self,first,last,age):
        self.first = first
        self.last = last
        self.age = age
        User.active_users += 1
		
	def __repr__(self):
		return f"{self.first}"	

    def logout(self): 
        User.active_users -= 1 
        return f"{self.first} has logged out"  

...	
\end{listing}
where we have omitted the unnecessary methods. The class methods are defined with \verb|@classmethod|. We have also defined an instance method to logout users and method with some mysterious \verb|__repr__| that we will explain below. Then, run the following code 
\begin{listingcont}
user1 = User("Joe", "Smith", 68)        
user2 = User("Blanka", "Lopez", 41)
print(User.display_active_users())
print(user2.logout())
print(User.display_active_users())	
\end{listingcont}
We see that initially there were 2 active users, then one logged out, and remains just 1 active user. 

The second class method is more interesting. Suppose we have a string that is in the format of comma separated values (csv) and we want to build a new user from it. We basically have to separate this string in 3 entries, the first name, last name and age. This method does it for us. Run 
\begin{listingcont}
tom = User.from_string("Tom, Jones, 89")
print(tom.first) 
print(tom.full_name(), ",", tom.age, "years old")
\end{listingcont}

There is one final piece of information in the code above we need to know. The \verb|__repr__| method is quite useful if we want to have some control over the output. First, comment this part of the code and try to run the following code
\begin{listingcont}
print(tom)
\end{listingcont}
we get something of the form \verb|<__main__.User object at XXX>|. On the other hand, we use the \verb|__repr__| method to control this outcome. In that case we obtain the first name as an output. 


\subsubsection{Exercise: Cards \& Deck Classes}

Here I want to consider the problem of section \# 25 of Colt's course. It is basically the definition of two classes with the following properties:

\textbf{-- Card class}
\begin{itemize}
    \item Each instance of Card  should have a suit ("Hearts", "Diamonds", "Clubs", or "Spades").
    \item Each instance of Card  should have a value ("A", "2", "3", "4", "5", "6", "7", "8", "9", "10", "J", "Q", "K").
    \item Card 's \verb|__repr__| method should return the card's value and suit (e.g. "A of Clubs", "J of Diamonds", etc.)
\end{itemize}


\textbf{-- Deck class}
\begin{itemize}
    \item Each instance of Deck should have a cards attribute with all 52 possible instances of Card.
    \item Deck  should have an instance method called count  which returns a count of how many cards remain in the deck.
    \item Deck 's \verb|__repr__| method should return information on how many cards are in the deck (e.g. "Deck of 52 cards", "Deck of 12 cards", etc.)
    \item Deck  should have an instance method called \verb|_deal|  which accepts a number and removes at most that many cards from the end of the deck (it may need to remove fewer if you request more cards than are currently in the deck!). If there are no cards left, this method should return a ValueError  with the message "All cards have been dealt".
    \item Deck  should have an instance method called shuffle  which will shuffle a full deck of cards. If there are cards missing from the deck, this method should raise a \verb|ValueError| with the message "Only full decks can be shuffled". shuffle should return the shuffled deck.
    \item Deck  should have an instance method called \verb|deal_card| which uses the \verb|_deal| method to deal a single card from the deck and return that single card.
    \item Deck  should have an instance method called \verb|deal_hand| which accepts a number and uses the \verb|_deal| method to deal a list of cards from the deck and return that list of cards.
\end{itemize}

\paragraph{My Solution}

Here is my solution
\begin{listing}{1}
from random import shuffle

class Card:
    allowed_suits = ('Hearts', 'Diamonds', 'Clubs', 'Spades')
    allowed_values =  ('A', '2', '3', '4', '5', '6', 
    '7', '8', '9', '10', 'J', 'Q', 'K')

    def __init__(self, value, suit):
        if value not in Card.allowed_values:            
            raise ValueError(f"Value needs to be 'A', '2', '3', '4', '5', 
            '6', '7', '8', '9', '10', 'J', 'Q', 'K'")
        if suit not in Card.allowed_suits:
            raise ValueError(f"Suit needs to be 'Hearts', 
            'Diamonds', 'Clubs' or 'Spades'")
        self.value = value
        self.suit = suit

    def __repr__(self):
        return f'{self.value} of {self.suit}'      
\end{listing}
and 
\begin{listingcont}
class Deck:
    def __init__(self):
        allowed_suits = ('Hearts', 'Diamonds', 'Clubs', 'Spades')
        allowed_values =  ('A', '2', '3', '4', '5', '6', '7', '8', '9',
        '10', 'J', 'Q', 'K')        
        self.cards = [Card(v,s) for v in allowed_values for s in allowed_suits]

    def __repr__(self):
        return f"Deck of {self.count()} cards"
        # I should favor the .format() format.
                    
    def count(self):
        return len(self.cards)

    def _deal(self, num):        
        cards = self.cards 
        hand = []
        count = 0
        if len(cards) == 0:
            raise ValueError("All cards have been dealt")
        while count < num:
            hand.append(cards.pop(-1))
            if len(cards) == 0:
                break
            count +=1
        return hand

    def shuffle(self):    
        if self.count() < 52:
            raise ValueError("Only full decks can be shuffled")

        shuffle(self.cards)
        return self            

    def deal_card(self):
        return self._deal(1)[0]

    def deal_hand(self, n):
        return self._deal(n)    
\end{listingcont}

I can also execute some tests
\begin{listingcont}

card = Card('A', 'Spades')
print(card)

my_deck = Deck()
print(my_deck.cards)
print(10 * '*')
my_deck.shuffle()
print(my_deck.cards)

card = my_deck.deal_card()
print(card)

print(my_deck.deal_hand(50))
print(my_deck.count())
print(my_deck.deal_hand(5))
print(my_deck.deal_hand(10))    
\end{listingcont}


\paragraph{Colt's solution}

Here I should compare with Colt's solution. 
\begin{listing}{1}
from random import shuffle

class Card:
    def __init__(self, value, suit):
        self.value = value
        self.suit = suit

    def __repr__(self):
        # return "{} of {}".format(self.value, self.suit)
        return f"{self.value} of {self.suit}"    
\end{listing}
and 
\begin{listingcont}
class Deck:
	def __init__(self):
		suits = ["Hearts", "Diamonds", "Clubs", "Spades"]
		values = ['A','2','3','4','5','6','7','8','9','10','J','Q','K']
		self.cards = [Card(value, suit) for suit in suits for value in values]

	def __repr__(self):
		return f"Deck of {self.count()} cards"

	def count(self):
		return len(self.cards)

	def _deal(self, num):
		count = self.count()
		actual = min([count,num])
		if count == 0:
			raise ValueError("All cards have been dealt")
		cards = self.cards[-actual:]
		self.cards = self.cards[:-actual]
		return cards

	def deal_card(self):
		return self._deal(1)[0]

	def deal_hand(self, hand_size):
		return self._deal(hand_size)

	def shuffle(self):
		if self.count() < 52:
			raise ValueError("Only full decks can be shuffled")

		shuffle(self.cards)
		return self    
\end{listingcont}
with tests
\begin{listingcont}

d = Deck()
d.shuffle()
card = d.deal_card()
print(card)
hand = d.deal_hand(50)
card2 = d.deal_card()
print(card2)
print(d.cards)
card2 = d.deal_card()

# print(d.cards)
\end{listingcont}



\subsection{Inheritance}

Suppose we want to build users with different functionalities. For example, ordinary users and moderators. These two share a lot in common, but moderators have additional functionalities. We could define two distinct classes, but it is easier to use this idea of \emph{inheritance}.

As a simpler example 
\begin{listing}{1}
class Animal:
    cool = True # This is a class attribute

    def make_sound(self, sound):
        print(sound)
    
class Cat(Animal):
    pass 

gandalf = Cat()
gandalf.make_sound
gandalf.cool
\end{listing}
I could use the \verb|isinstance(instance, object)| function that verifies that the instance belongs to both classes. 
\begin{listingcont}
print(isinstance(gandalf, Animal))
> True
print(isinstance(gandalf, Cat))
> True
print(isinstance(gandalf, list))
> False
\end{listingcont}


\subsubsection{Properties}

Consider the class Human:
\begin{listing}{1}
class Human: 
    def __init__(self, first, last, age):
        self.first = first 
        self.last = last
        if age >= 0:
            self.age = age
        else:
            self.age = 0
\end{listing}
where the conditionals avoid, for example, negative ages. The solution above does not prevent, on the other hand, that we change the age afterwords, for example if we replace \verb|jane.age = -100|. 
\begin{listingcont}
jane = Human("Jane", "Goodall", 50)
print(jane.age)
jane.age = -100
print(jane.age)
\end{listingcont}
Then, we need to define some properties. First of all, we need to do some modifications
\begin{listing}{1}
class Human: 
    def __init__(self, first, last, age):
        self.first = first 
        self.last = last
        if _age >= 0:
            self._age = age
        else:
            self.age = 0

# Not the best way of doing that:

#       def get_age(self):
#          return self._age
#       def set_age(self, new_age):
#           if _age >= 0:
#               self._age = age
#           else:
#               self.age = 0
 
# The best way of doing that:

    @property
    def age(self):
        return self._age

    @age.setter 
    def age(self, value):
        if value >=0:
            self._age =value
        else:
            raise ValueError("age can't be negative")
\end{listing}
then we run
\begin{listingcont}
jane = Human("Jane", "Goodall", 50)
print(jane.age)
\end{listingcont}

\subsubsection{Introduction to Super()}

The Cat class that we have considered above is useless as it was. We want to give more functionalities to it. 
\begin{listing}{1}
    class Animal:
    def __init__(self, name, species):
        self.name = name 
        self.species = species

    def __repr__(self):
        return f"{self.name} is a {self.species}"
        
    def make_sound(self, sound):
        print(sound)
    
class Cat(Animal):
    def __init__(self, name, breed, toy): 
# Repetition we want to avoid:
#       self.name = name 
#       self.species = "Cat" 
# This is one possibility to avoid repetition:
#       Animal.__init__(self, name, species="Cat") 
# This is the pythonic way:
        super().__init__(name, species="Cat")
        self.breed = breed
        self.toy = toy

        def play(self):
            print(f"{self.name} plays with {self.toy}")

gandalf = Cat("Gandalf", "Cat", "Scottish Fold", "String")
print(gandalf)
gandalf.play() # Cat method
gandalf.make_sound("Meow") # Animal method
\end{listing}

\subsubsection{Example of Inheritance: Users \& Moderators}

Let me come back to the original \verb|Users| class:
\begin{listing}{1}
class User: 

    active_users = 0 # We define a class attribute

    @classmethod
    def display_active_users(cls):
        return f"There are currently {cls.active_users} active users."

    @classmethod
    def from_string(cls, data_str):
        first,last,age = data_str.split(",")        
        return cls(first,last,int(age))

    def __init__(self,first,last,age):
        self.first = first
        self.last = last
        self.age = age
        User.active_users += 1
        
    def __repr__(self):
        return f"{self.first}"	

    def logout(self): 
        User.active_users -= 1 
        return f"{self.first} has logged out"  

    def full_name(self): # full_name method
        return f"{self.first} {self.last}"

    def initials(self): # initials method
        return f"{self.first[0]} {self.last[0]}"

    def likes(self, thing): # likes method. This methods has an argument
        return f"{self.first} likes {thing}"

    def is_senior(self): # Is senhor method has conditionals
        return self.age >= 65    

    def birthday(self):
        self.age +=1
        return f"Happy {self.age}th, {self.first}"
\end{listing}

We can now define the class of \verb|Moderators| and print some results:
\begin{listingcont}
class Moderator(User):
    total_mods = 0
    def __init__(self, first, last, age, community):
        super().__init__(first, last, age)
        self.community = community
        Moderator.total_mods += 1

    @classmethod
    def display_active_mods(cls):
        return f"There are currently {cls.total_mods} active mods."

    def remove_post(self):
        return f"{self.full__name()} removed a 
        post from the {self.community} community"

`user1 = User('Tom', 'Morello', 57)
user2 = User('Adam', 'Jones', 57)
user3 = User('Danny', 'Carey', 60)
jasmine = Moderator('Thiago', "Araujo", 34, 'Bands')      
jasmine = Moderator('Aline', "Lima", 35, 'Piano')      
print(User.display_active_users())
print(Moderator.display_active_mods())'  
\end{listingcont}


\subsubsection{Multiple Inheritance}

It is not recommended to use multiple inheritance. It is better to organize the ideas a bit better. Consider the example
\begin{listing}{1}
class Aquatic:
    def __init__(self, name):
        self.name = name 

    def swim(self):
        return f"{self.name} is swimming"

    def greet(self):
        return f"I am {self.name} of the sea!"        

class Ambulatory:
    def __init__(self, name):
        self.name = name 

    def walk(self):
        return f"{self.name} is walking"

    def greet(self):
        return f"I am {self.name} of the land"

class Penguin(Ambulatory, Aquatic):
    def __init__(self, name):
        # super().__init__(name = name) In this particular case, 
        #it is better to be explicit
        Ambulatory.__init__(self, name=name)
        Aquatic.__init__(self, name=name)
\end{listing}
and we can try the code 
\begin{listingcont}
jaws = Aquatic('Jaws')
lassie = Ambulatory('Lassie')
captain_cook = Penguin('Captain Cook')

print(captain_cook.swim())
print(captain_cook.walk())    
\end{listingcont}

\subsubsection{Method Resolution Order (MRO)}

Whenever we create a class, Python sets a MRO for that class which is the order in which Python will loop for methods on instances of that class.

We can see it in three ways
\begin{itemize}
    \item \verb|__mro__| attribute on the class 
    \item Use the method \verb|mro()| on the class 
    \item Use the builtin \verb|help(cls)| method
\end{itemize}

Using the Penguin example above, we can run
\begin{listing}{1}
Penguin.__mro__    
Penguin.mro()
help(Penguin)
\end{listing}

Another example
\begin{shaded}
\begin{listing}{1}
class A:
    def do_something(self):
        print("Method defined in: A")    

class B(A):
    def do_something(self):
        print("Method defined in: B")        

class C(A):
    def do_something(self):
        print("Method defined in: C")

class D(B, C):
    def do_something(self):
        print("Method defined in: D")        

thing = D()        
thing.do_something()

help(D)
\end{listing}    
The methods are inherited from \verb|D|, \verb|B|, \verb|C|, \verb|A|. We can comment the methods to see as it happens.
\end{shaded}


\subsection{Polymorphism}

There are two types of polymorphism.

\subsubsection{Polymorphism \& Inheritance}

\# 1. When the class method works in a similar way for different classes. For example, when we have a method in a parent class and that is overridden by a subclass. This is called \emph{method overriding}. For example:
\begin{listing}{1}
class Animal():
    def speak(self):
        raise NotImplementedError('Subclass needs to implement this method'
        
class Dog(Animal):
    def speak(self):
        return 'woof'        

class Cat(Animal):
    def speak(self):
        return 'meow'  
\end{listing}
It is an example of polymorphism.

\subsubsection{Special methods}

2. It also is when the same operation works for different kinds of objects. The \verb|len| method for example. For Example
\begin{listing}{1}
8 + 2
> 10
'8' + '2'
> '82'    
\end{listing}
In the first case, it is a sum of integers, and in the second example it is a concatenation of strings. 

\paragraph{Example}

Let us now define our own special method. These are examples of dunder methods: \verb|__something__|, such as \verb|__init__|, \verb|__repr__|, \verb|__len__| and so forth.
\begin{listing}{1}
from copy import copy

class Human:
    def __init__(self, first, last, age):
        self.first = first
        self.last = last
        self.age = age

    def __repr__(self):
        return f"Human named {self.first} {self.last}"

    def __len__(self):
        return self.age      

    def __add__(self, other):
        if isinstance(other, Human):
            return Human(first = 'Newborn', last = self.last, age = 0)
        return "You can't add that!"

    def __mul__(self, other):
        if isinstance(other, int):
            return [copy(self) for i in range(other)]
        return "Can't multiply"

j = Human('James', 'T. Kirk', 35) 
k = Human('Nyota', 'Uhura', 30) 
print(j)
print(len(j))

print(j + k)
print(j * 2)
\end{listing}


\section{Iterators \& Generators}

\subsection{Iterables \& Iterator} Let us understand the differences between these two classes of objects.

{\bf \# Iterable} is an object which will return an iterator when \verb|iter()| is called on it.

For example, the string \verb|'HELLO'| is an iterable, but is not an iterator. A list is also an example of iterable. 

{\bf \# Iterator} is an object that can be iterated upon. An object which returns data, one element at a time when \verb|next()| is called on it. 

When we call \verb|next()| on an iterator, the iterator returns the next item. It keeps doing so until it raises a \verb|StopIteraction| error.

For example:   
\begin{listing}{1}
name = 'melissa' # melissa is a iterable.
it_name = iter(name) # now we define the iterator it_name 
next(it_name)
\end{listing}
This is what the \verb|for|-loops do behind the scene. It calls the \verb|next()| over and over again.

\subsubsection{Creating iterators}

Here we want to use a version of \verb|for|-loops. It will help us to understand what is happening behind the curtains. 
\begin{listing}{1}
def my_for(iterable):
iterator = iter(iterable)
while True:
    try:
        print(next(iterator))
    except StopIteration:
        break

my_for('melissa')
my_for([1,2,3,4,5]) 
\end{listing}

I can also do more interesting things with this function. For example:
\begin{listing}{1}
def my_for(iterable, func):
    iterator = iter(iterable)
    while True:
        try:
            thing = next(iterator)
        except StopIteration:
            break
        else:
            func(thing)

def square(num):            
    print(num ** 2)

my_for('melissa', print)
my_for([1,2,3,4,5], square)  
\end{listing}

\subsubsection{Writing a custom iterator}

Here we want to create our version of \verb|range|. Here we call it \verb|Counter|.
\begin{listing}{1}
class Counter:
    def __init__(self, low, high):
        self.current = low
        self.high = high 

    def __iter__(self):
        return self 

    def __next__(self):
        if self.current < self.high:
            num = self.current
            self.current += 1
            return num 
        raise StopIteration

for x in Counter(50,70):
    print(x)
\end{listing}

\subsubsection{Making the Deck class iterable}

Remember the exercise we have done on the Deck class. We want to make it iterable. Starting from that class we have defined, we want to do something of the form 
\begin{listing}{1}
my_deck = Deck()

for card in my_deck:
    print(card)
\end{listing}
We need to add the following line to the \verb|Deck()| class
\begin{listing}{1}
    def __iter__(self):
        return iter(self.cards)
\end{listing}

\subsection{Generators}

These are also iterators. It is an easy and quicker way to create iterators. It also needs less memory. 

\subsubsection{Generator functions}

While functions use \verb|return| once and they return the return value. Generator functions, on the other hand, use \verb|yield| and can be used multiple times. Also, they return generators. An an example
\begin{listing}{1}
def count_up(max):
    count = 1
    while count <= max:
        yield count
        count += 1    
\end{listing}
When the program executes the \verb|yield|, it stops there waiting the \verb|next|. We can execute it as follows
\begin{listingcont}
count = count_up(5)
print(next(count))
print(next(count))
print(next(count))    
print(list(count))
\end{listingcont}

Another funny example 
\begin{listing}{1}
def week():    
    days_week = ['Monday', 'Tuesday', 'Wednesday',
    'Thursday', 'Friday', 'Saturday', 'Sunday']
    count = 'Monday'
    for day in days_week:
        yield day    
\end{listing}

\subsubsection{Infinite generator}

We can create an infinite generator. Let us create a function that we want to create a function that return \(1,2,3,4\) and \(1,2,3,4\) and so on and so forth. It is like the \(4/4\) beat in music. 

The brute force way is basically the following
\begin{listing}{1}
def current_beat(max):    
    # max = 100
    nums = [1,2,3,4]
    i = 0
    result = []
    while len(result) < max:
        if i >= len(nums): i =0
        result.append(nums[i])
        i += 1
    return result

print(current_beat(100))    
\end{listing}
But this is not what we want. We want one thing at a time. Moreover, we do not want to collect all results in a giant list. Then 
\begin{listing}{1}
def current_beat():
    i = 0
    while True:
        if i >= len(nums): i = 0
        yield nums[i]
        i += 1
\end{listing}

Another funny problem 
\begin{listing}{1}
def yes_or_no():
    ans = 'yes'
    while True:
        yield ans
        ans = 'no' if ans == 'yes' else 'yes'    
\end{listing}

\subsubsection{Example: Song}

Another funny exercise. My solution
\begin{listing}{1}
def make_song(count = 99, beverage = 'soda'):
    n = 0
    while True: 
        if count - n == 1:
            yield f'Only 1 bottle of {beverage} left!'
        elif count - n == 0:
            yield f'No more {beverage}!'
            raise StopIteration
        else: 
            yield f'{count - n} bottles of {beverage} on the wall.'
        n += 1            

song = make_song(5,'kombucha')

print(next(song))
print(next(song))
print(next(song))
print(next(song))
print(next(song))
print(next(song))
print(next(song))    
\end{listing}

Colt's solution
\begin{listing}{1}
def make_song(verses=99, beverage="soda"):
    for num in range(verses, -1, -1):
        if num > 1:
            yield "{} bottles of {} on the wall.".format(num, beverage)
        elif num == 1:
            yield "Only 1 bottle of {} left!".format(beverage)
        else:
            yield "No more {}!".format(beverage)    
\end{listing}


\subsubsection{Fibonacci numbers}

Let us see some advantages of generators with the nice example of the Fibonacci numbers. The code with ordinary functions is heavy
\begin{listing}{1}
def fib_list(max):
    nums = []
    a, b = 0, 1
    while len(nums) < max:
        nums.append(b)
        a, b = b, a + b
    return nums

print(fib_list(10))    
\end{listing}

Here with generator
\begin{listing}{1}
def fib_gen(max):
    x = 0
    y = 1
    count = 0
    while count < max: 
        x, y = y, x + y
        yield x
        count += 1

for n in fib_gen(50):
    print(n)      
\end{listing}


\subsubsection{Getting multiples}

First example for Getting multiples:
\begin{listing}{1}
# MY CODE: 
def get_multiples(number = 1, count = 10):
    default_numbers = list(range(1,count+1))
    for n in default_numbers:
        yield number * n
        
# COLT'S CODE 
def get_multiples(num=1, count=10):
    next_num = num
    while count > 0:
        yield next_num
        count -= 1
        next_num += num        
\end{listing}

For infinite multiples:
\begin{listing}{1}
# MY CODE 
def get_unlimited_multiples(num = 1):
    n = 1
    while True:
        yield n * num
        n += 1


sevens = get_unlimited_multiples(7)
A = [next(sevens) for i in range(15)] 
print(A)

ones = get_unlimited_multiples()
B = [next(ones) for i in range(20)] 
print(B)    
\end{listing}
and 
\begin{listingcont}
# COLT'S CODE     
def get_unlimited_multiples(num=1):
    next_num = num
    while True:
        yield next_num
        next_num += num
\end{listingcont}


\subsubsection{Generator expressions}

This is the equivalent of list comprehensions. It is an easy way to define generators. The syntax is 
\begin{listing}{1}
g1 = (num for num in range(1,10)) 

type(g1)
> <class 'generator'>
\end{listing}
This is equivalent to 
\begin{listingcont}
def nums():
    for num in range(1,10):
        yield num 

g2 = nums()
type(g2)
> <class 'generator'>
\end{listingcont}

\section{Decorators}

Before starting \verb|Decorators|, we need to review some concepts on functions. 

\subsection{Higher Order functions} For example, higher order functions, that is, functions that accept other functions as arguments. For example
\begin{listing}{1}
def sum(n, func):
    total = 0
    for num in range(n):
        total += func(num)
    return total

def square(x):
    return x ** 2        
\end{listing}

We can nest functions. 
\begin{listing}{1}
from random import choice
def greet(person):
    def get_mood():
        msg = choice(('Hello there', 'Go away', 'I love you'))
        return msg
    result = get_mood() + ' ' + person 
    return result

print(greet('Toby'))
\end{listing}
In this case, we have returned the results of nesting functions one inside the other. But we can also return the functions 
\begin{listing}{1}
from random import choice
def make_laugh():
    def get_laugh():
        l = choice(('HAHAHAHA', 'lol', 'hehehe'))
        return l
    
    return get_laugh
\end{listing}

We can also add an extra layer of complexity if we allow arguments. For example, 
\begin{listing}{1}
from random import choice
def make_laugh(person):
    def get_laugh():
        laugh = choice(('HAHAHAHA', 'lol', 'hehehe'))
        return f'{laugh} {person}'
    
    return get_laugh

laugh_at = make_laugh('Linda')

print(laugh_at())
print(laugh_at())
\end{listing}


\subsection{Decorators}

These are functions that wrap other functions and enhance their behavious. They are examples of higher order functions. Consider the example
\begin{listing}{1}
def be_polite(fn):
    def wrapper():
        print('What a pleasure to meet you!)
        fn()
        print('Have a good day')
    return wrapper 

def greet():
    print('My name is Thiago')    

def rage():
    print('I hate you')
\end{listing}
Observe that the function \verb|be_polite| wraps the functions \verb|greet| and |rage|. Then we can call them as follows
\begin{listingcont}
greet = be_polite(greet)    
polite_rage = be_polite(rage)

greet()
polite_rage()
\end{listingcont}
Observe that I had to define new functions \verb|greet| and \verb|polite_rage| to call the be polite function. 

The decorators help us with these definitions. We can improve the previous example as follows
\begin{listing}{1}
def be_polite(fn):
    def wrapper():
        print('What a pleasure to meet you!)
        fn()
        print('Have a good day')
    return wrapper 

@be_polite    
def greet():
    print('My name is Thiago')    

@be_polite
def rage():
    print('I hate you')
\end{listing}
And now, we can call the function as 
\begin{listingcont}
greet()
polite_rage()
\end{listingcont}

\subsubsection{Decorators with different signatures.}

Now, suppose that we want to consider different arguments to our functions. The syntas is easy, we just need to use the \verb|*args| and \verb|**kwargs|. 

Consider the following example
\begin{listing}{1}
def shout(fn):
    def wrapper(*args, **kwargs):
        return fn(*args, **kwargs).upper()
    return wrapper

@shout 
def greet(name):
    return f"Hi, I am {name}"    

@shout 
def order(main, side):
    return f"Hi, I'd like the {main}, with a side of {side}, please"
    
@shout 
def lol():
    return 'lol'
\end{listing}
and we call them 
\begin{listingcont}
print(greet('todd'))
print(order('burger','fries'))
print(lol())
\end{listingcont}

\subsection{Using decorators, several examples}

\subsubsection{Using decorators to preserve metadata}

Consider the following code 
\begin{listing}{1}
def log_function_data(fn):
    def wrapped(*args, **kwargs):
        print(f"you are about to call {fn.__name__}")
        print(f"here's the documentation: {fn.__doc__}")
        return fn(*args, **kwargs)
    return wrapper 

@log_function_data
def add(x, y):
    '''Adds two numbers together'''
    return x + y
\end{listing}
but the metadata is lost with the wrapper. See the odd behaviour if we call the documentation
\begin{listingcont}
print(add.__doc__)
print(add.__name__)
help(help)    
\end{listingcont}

We can solve this problem using the module \verb|functools|. In particular, 
\begin{listing}{1}
from functions import wraps

def log_function_data(fn):
    @wraps(fn)
    def wrapped(*args, **kwargs):
        print(f"you are about to call {fn.__name__}")
        print(f"here's the documentation: {fn.__doc__}")
        return fn(*args, **kwargs)
    return wrapper 

@log_function_data
def add(x, y):
    '''Adds two numbers together'''
    return x + y
\end{listing}
Now, things work correctly
\begin{listingcont}
print(add.__doc__)
print(add.__name__)
help(help)    
\end{listingcont}

\subsubsection{Building a speed-test with decorators}

Using decorators, we can build a simple spped-test, for example 
\begin{listing}{1}
from time import time 
from functools import wraps 

def speed_test(fn):
    def wrapper(*args, **kwargs):
        start_time = time()
        result = fn(*args, **kwargs)
        end_time = time()
        print(f"Executing {fn.__name__}")
        print(f"Time elapsed: {end_time - start_time}")
        return result 
    return wrapper 

@speed_test 
def sum_nums():
    return sum(x for x in range(10000000))

print(sum_nums())    
\end{listing}

\paragraph{Exercises} In this exercise, Colt defines a function that shows the arguments before executing the function: 
\begin{listing}{1}
from functools import wraps    

def show_args(fn):
    def wrapper(*args, **kwargs):
        _args = tuple([arg for arg in args])
        _kwargs = {item : key for item, key in kwargs.items()}
        print(f'Here are the args: {_args}')
        print(f'Here are the kwargs: {_kwargs}')
        return fn(*args, **kwargs)
    return wrapper
        
@show_args
def do_nothing(*args, **kwargs):
    pass

do_nothing(1, 2, 3,a="hi",b="bye")    
\end{listing}

His solution is simpler 
\begin{listing}{1}
from functools import wraps

def show_args(fn):
    @wraps(fn)
    def wrapper(*args, **kwargs):
        print("Here are the args:", args)
        print("Here are the kwargs:", kwargs)
        return fn(*args, **kwargs)
    return wrapper    
\end{listing}

\subsubsection{Ensuring Args with a decorator}

One can also use decorators to prevent misuse of functions. For example, avoid that someone uses \verb|**kwargs| when we (the developers) expected \verb|*args|. For example 
\begin{listing}{1}
from functools import wraps

def ensure_no_kwargs(fn):
    @wraps(fn)
    def wrapper(*args, **kwargs):
        if kwargs:
            raise ValueError("No kwargs allowed! sorry :(")
        return fn(*args, **kwargs)
    return wrapper

@ensure_no_kwargs
def greet(name):
    print(f"hi there {name}")

greet(name="Tony")    
\end{listing}


\paragraph{Exercise} Another problem is the following boilerplate code 
\begin{listing}{1}
from functools import wraps 

def double_return(fn):
    @wraps(fn)
    def wrapper(*args, **kwargs):
        return [fn(*args, **kwargs), fn(*args, **kwargs)]
    return wrapper

@double_return 
def add(x, y):
    return x + y
    
@double_return
def greet(name):
    return "Hi, I'm " + name

print(add(1, 2)) # [3, 3]
print(greet("Colt")) # ["Hi, I'm Colt", "Hi, I'm Colt"]    
\end{listing}

\paragraph{Ensuring a limited number of arguments} Suppose now that we want a function that accepts a maximum of \(3\) arguments.
\begin{listing}{1}
from functools import wraps

def ensure_fewer_than_three_args(fn):
    def wrapper(*args, **kwargs):
        if kwargs:
            raise ValueError("No kwargs allowed! sorry :(")
        elif len(args) >= 3:
            return "Too many arguments!"
        return fn(*args, **kwargs)
    return wrapper

@ensure_fewer_than_three_args
def add_all(*nums):
    return sum(nums)

print(add_all()) # 0
print(add_all(1)) # 1
print(add_all(1,2)) # 3
print(add_all(1,2,3)) # "Too many arguments!"
print(add_all(1,2,3,4,5,6)) # "Too many arguments!"    
\end{listing}

\paragraph{Example} Another example with arguments with constraits.
\begin{listing}{1}
def only_ints(fn):
    def wrapper(*args, **kwargs):
        truth = [True if type(arg) == int else False for arg in args]
        if all(truth) == False:
            return "Please only invoke with integers."
        return fn(*args, **kwargs)
    return wrapper


@only_ints 
def add(x, y):
    return x + y
    
print(add(1, 2)) # 3
print(add("1", "2")) # "Please only invoke with integers."    
\end{listing}

Colt uses \verb|any| in this case 
\begin{listing}{1}
from functools import wraps

def only_ints(fn):
    @wraps(fn)
    def inner(*args, **kwargs):
        if any([arg for arg in args if type(arg) != int]):
            return "Please only invoke with integers."
        return fn(*args, **kwargs)
    return inner    
\end{listing}


\subsubsection{Ensuring users with decorators}

Now, we write a code to ensure authorized users. 
\begin{listing}{1}
def ensure_authorized(fn):
    def wrapper(*args, **kwargs):
        if ('role', 'admin') not in kwargs.items():
            return "Unauthorized"
        return fn(*args, **kwargs)
    return wrapper 

@ensure_authorized
def show_secrets(*args, **kwargs):
    return "Shh! Don't tell anybody!"

print(show_secrets(role="admin")) # "Shh! Don't tell anybody!"
print(show_secrets(role="nobody")) # "Unauthorized    
\end{listing}
Colt's solution 
\begin{listing}{1}
def ensure_authorized(fn):
    @wraps(fn)
    def wrapper(*args, **kwargs):
        if kwargs.get("role") == "admin":
            return fn(*args, **kwargs)
        return "Unauthorized"
    return wrapper    
\end{listing}


\subsection{Decorators that accept arguments} For example, we want to ensure that the first arguments in a function always take a chosen value. We need an extra layer of function, pay attention to the following 
\begin{listing}{1}
# When we write:
@decorator
def func(*args, **kwargs):
    pass
# We're really doing:
func = decorator(func)


# When we write:
@decorator_with_args(arg)
def func(*args, **kwargs):
    pass
# We're really doing:
func = decorator_with_args(arg)(func)    
\end{listing}

See the working example below.
\begin{listing}{1}
from functools import wraps

def ensure_first_arg_is(val):
    def inner(fn):
        @wraps(fn)
        def wrapper(*args, **kwargs):
            if args and args[0] != val:
                return f"First arg needs to be {val}"
            return fn(*args, **kwargs)
        return wrapper
    return inner

@ensure_first_arg_is("burrito")
def fav_foods(*foods):
    print(foods)

print(fav_foods("burrito", "ice cream")) # ('burrito', 'ice cream')
print(fav_foods("ice cream", "burrito")) # 'Invalid! First argument must be burrito'

@ensure_first_arg_is(10)
def add_to_ten(num1, num2):
    return num1 + num2

print(add_to_ten(10, 12)) # 12
print(add_to_ten(1, 2)) # 'Invalid! First argument must be 10'    
\end{listing}

\subsubsection{Enforcing types with decorators}

See the example below 
\begin{listing}{1}
def enforce(*types):
    def decorator(f):
        def new_func(*args, **kwargs):
            #convert args into something mutable   
            newargs = []        
            for (arg, type) in zip(args, types):
               newargs.append( type(arg)) 
            return f(*newargs, **kwargs)
        return new_func
    return decorator

@enforce(str, int)
def repeat_msg(msg, times):
	for time in range(times):
		print(msg)

@enforce(float, float)
def divide(a,b):
	print(a/b)
# repeat_msg("hello", '5')
divide('1', '4')    
\end{listing}


\subsubsection{Delay function}

This is another example that uses the module \verb|time|. Here we want to wait a bit before executing the function. Here is my solution
\begin{listing}{1}
from time import sleep

def delay(interval):
    def inner(fn):
        def wrapper(*args, **kwargs):
            print("Waiting {}s before running say_hi".format(interval))
            sleep(interval)
            return fn(*args, **kwargs)
        return wrapper
    return inner

@delay(3)
def say_hi():
    return "hi"    

print(say_hi(3))       
\end{listing}

Here is Colt's solution
\begin{listing}{1}
from functools import wraps
from time import sleep
    
def delay(timer):
    def inner(fn):
        @wraps(fn)
        def wrapper(*args, **kwargs):
            print("Waiting {}s before running {}".format(timer, fn.__name__))
            sleep(timer)
            return fn(*args, **kwargs)
        return wrapper
    return inner    
\end{listing}

\section{Nested functions}

Here I would like to understand a bit better some important details of nested functions. The main reason is that I want to understand two examples of exercises that Colt's proposed and I do not completely understand his solutions. In particular, let me start with the problems, my solution and his solution. 

These problems can be found in the last of the course. I have solved them in the \verb|exercises.py|.

\paragraph{\bf Problem 01:} Running the average

In this example, we need to define a function that runs the average of the previous values.

Here is my solution
\begin{listing}{1}
    def running_average():
        values = []
        def average(num):
                values.append(num)
                return round(sum(values) / len(values),1)
        return average
fn = running_average()        
print(fn(10))
print(fn(11))
print(fn(12))
fn = running_average()        
print(fn(1))
print(fn(3))
\end{listing}	

We need to compare it with Colt's solution 
\begin{listing}{1}
def running_average():
    running_average.accumulator = 0
    running_average.size = 0

    def inner(number):
        running_average.accumulator += number
        running_average.size += 1
        return running_average.accumulator / running_average.size

    return inner    
\end{listing}

Observe that we have two methods that I do not understand very well, namely \verb|accumulator| and \verb|size|.

\paragraph{\bf Problem 02:} Running the average

The second problem is 
\begin{listing}{1}
def once(fn):
    n = []
    def test(*args):
            if len(n) < 1:
                    n.append(0)
                    return fn(*args)
            else:
                    return None
    return test

def add(a,b):
    return a + b

oneAddition = once(add)

print(oneAddition(2,2)) # 4
print(oneAddition(2,2)) # None
print(oneAddition(12,200)) # None     
\end{listing}

With Colt's solution given by
\begin{listing}{1}
def once(fn):
    fn.is_called = False
    def inner(*args):
        if not(fn.is_called):
            fn.is_called = True
            return fn(*args)
    return inner    
\end{listing}

\subsection{Closures}

What is annoying in my solutions is that I had to define an empty list as a counter. In the first example, the list is reasonable and it can be considered an okay solution. In the second example, although the list works, it is an ugly workaround. It should be better to use the counter as an \emph{scalar} (it is not a pythonic terminologt). This has to do with scopes. 

There are some good articles explaining the situations above. Some examples articles (clickable links):
\begin{itemize}
    \item \href{https://zetcode.com/python/python-closures/}{Python closures}
    \item \href{https://www.linkedin.com/pulse/5-essential-aspects-python-closures-sagar-an/}{5 Essential Aspects of Python Closures}
    \item \href{https://www.geeksforgeeks.org/python-inner-functions/}{Python Inner Functions}
    \item \href{https://stackoverflow.com/questions/21959985/why-cant-python-increment-variable-in-closure}{Why can't Python increment variable in closure?}
    \item \href{https://stackoverflow.com/questions/38693236/python-counter-with-closure}{python counter with closure}
    \item \href{https://stackoverflow.com/questions/36901798/pythons-closure-local-variable-referenced-before-assignment}{Python's closure - local variable referenced before assignment}
    \item \href{https://stackoverflow.com/questions/64946483/dot-notation-in-pythons-function-definition}{dot notation in Python's function definition}
\end{itemize}

The problem we have above is the \verb|scope|. We have two keywords \verb|global| and \verb|nonlocal|. See the examples of how they work. 

\paragraph{\bf Global} It is used when we want to use a global variable in a function. For example: 
\begin{listing}{1}
# EXAMPLE OF A SCOPING PROBLEM:
total = 0

def increment():
    total += 1
    return total

print(increment()) # Error! 
# "I can't find a variable named total in this function"
\end{listing}    
but we can solve it as 
\begin{listing}{1}
total = 0

def increment():
    global total #use the global variable total
    total += 1
    return total

print(increment()) # 1
print(increment()) # 2
print(increment()) # 3    
\end{listing}
and now it works. 

\paragraph{\bf Nonlocal} Nonlocal variables can be used in nested functions (as the examples we have above). Actually, it allows us the change a parent variable in a child function. For example 
\begin{listing}{1}
def outer():
    count = 0 
    def inner():
        nonlocal count
        count += 1
        return count 
    return inner()
\end{listing}

\clearpage

\appendix 

\section{Handling Errors}

Here I need to talk about details on the errors. I need to observe this section, because it is useful. References are~\cite{geeks,errors, errors:types}.

\subsection{Raising errors}

Suppose that we want to color a text. Then we can raise some errors
\begin{listing}{1}
def colorize(text, color):
    colors = ("cyan", "yellow", "blue", "green", "magenta")
    if type(text) is not str:
        raise TypeError("text must be instance of str")
    if color not in colors:
        raise ValueError("color is invalid color")
    print(f"Printed {text} in {color}"
colorize([], 'cyan')
colorize(34, "red")    
\end{listing}
There are different types of Errors, see references above. I need to understand the subtleties related to these Errors.

\subsection{Try, except, else and finally}

Sometimes we expect some error, for example in an input. So, instead breaking the code, we can use the Try and Expect. The code tries something, if it works that is okay, if it does not work, it returns something else that we can use to correct our code.
\begin{listing}{1}
def get(d,key):
    try:
        return d[key]
    except KeyError:
        return None
d = {"name": "Ricky"}
print(get(d, "city"))
d["city"]    
\end{listing}

It can be complemented with \verb|else| and \verb|finally|. The syntax is 
\begin{listing}{1}
while True: 
    try: 
        "This block will test the excepted error to occur"
    except
        "Here you can handle the error"
    else: 
        "If there is no exception then this block will be executed"
    finally:
        "Finally block always gets executed either exception is generated or not"
\end{listing}

As an example
\begin{listing}{1}
while True:
    try:
        num = int(input("please enter a number: "))
    except ValueError:
    	print("That's not a number!")
    else:
    	print("Good job, you entered a number!")
    	break
    finally:
    	print("RUNS NO MATTER WHAT!")
print("REST OF GAME LOGIC RUNS!")        
\end{listing}

Another example
\begin{listing}{1}
def divide(a,b):
    try:
    	result = a/b
    except ZeroDivisionError:
    	print("don't divide by zero please!")
    except TypeError as err:
    	print("a and b must be ints or floats")
    	print(err)
    else:
    	print(f"{a} divided by {b} is {result}")    
\end{listing}
we can also collect the \verb|except| blocks, but we cannot be specific about the \verb|Error|, that is
\begin{listing}{1}
def divide(a,b):
    try:
    	result = a/b
    except (ZeroDivisionError, TypeError) as err:
    	print("Something went wrong!")
    	print(err)
    else:
    	print(f"{a} divided by {b} is {result}")    
\end{listing}

\subsection{Debugging}

Here we can debug with the module \verb|pdb|. The basic commands are
\begin{verbatim}
# l (list)
# n (next line)
# p (print)
# c (continue - finishes debugging)        
\end{verbatim}
We use this code as
\begin{verbatim}
import pdb
pdb.set_trace() # This part says where the debugging starts
# or in one line
import pdb; pdb.set_trace()    
\end{verbatim}

See an example 
\begin{listing}{1}
import pdb

first = "First"
second = "Second"
pdb.set_trace()
result = first + second
third = "Third"
result += third
print(result)

def add_numbers(a, b, c, d):
    import pdb; pdb.set_trace() 
    return a + b + c + d
add_numbers(1,2,3,4)        
\end{listing}

\section{File I/O}

Here we will learn how to work with other files which are not python files. 

\subsection{Reading files}

We use the \verb|open| function. For example, we use the syntax
\begin{listing}{1}
f = open('path')
f.read()
\end{listing}    
When files reads a file, it uses a cursor. So, when we read the file, the cursor is at the end, so if we try to read the file again, it shows an empty string. We need to learn how to work with cursors.
\begin{listing}{1}
from time import sleep 

file = open('text.txt')
print(file.read())

sleep(3)

print(file.read())    
\end{listing}

\subsubsection{seek()} It is a method to manipulate the cursor. 
\begin{listing}{1}
from time import sleep 

file = open('text.txt')
print(file.read())

sleep(3)

file.seek(6) # The cursor moves to the sixth character.

print(file.read())    
\end{listing}

\subsubsection{readline()} This reads the file line by line 
\begin{listing}{1}
file = open('text.txt')
print(file.readline())
print(file.readline())
print(file.readline())
print(file.readline())
\end{listing}

\subsubsection{readlines()} Now it gives a list with the lines
\begin{listing}{1}
file = open('text.txt')
print(file.readlines())
\end{listing}

\subsubsection{close()} Finally, we need to close the file when we are done. 
\begin{listing}{1}
file = open('text.txt')
print(file.closed)
file.close()
print(file.closed)    
\end{listing}

\subsection{With statements}

We can do the same thing with a different syntax. It is the following
\begin{listing}{1}
with open('file') as file:
    data = file.read()

file.closed # True
data
\end{listing}
In this case, the file is closed automatically. 

\subsection{Writing to files} Now we need to learn how we can modify our files. Here is a good example 
\begin{listing}{1}
with open('file', 'w') as file:
    file.write("Lorem ipsum dolor sit amet\n")
    file.write("Lorem ipsum dolor sit amet")
\end{listing}
we need to specify the argument \verb|'w'|. 

This method overwrites existing files or creates new files. In order to add information to existing files, we need to understand modes. 

\subsection{Modes for opening files} Here are the common modes

\begin{itemize}
    \item \verb|r| - Read a file (no writing). This is the default mode. 
    \item \verb|w| - Write to the file. It overwrites the file.
    \item \verb|a| - Append content to the end of the file. 
    \item \verb|r+| - Read and write to a file. It only works with existing files. It does not create a new one. 
\end{itemize}

\subsection{Examples}

Here some examples to practice a bit. 

\subsubsection{Example 1: Copy}

A function that copies the content of a file and pastes to another. 
\begin{listing}{1}
def copy(file_source, file_target):
    with open(file_source) as file:
        text = file.read()

    with open(file_target, 'w') as file:
        file.write(text)

copy('story.txt', 'story_copy.txt')    
\end{listing}

\subsubsection{Example 2: Reverserd copy}

A function that copies the content of a file and pastes to another but in a reversed order. 
\begin{listing}{1}
def copy_and_reverse(file_source, file_target):
    with open(file_source) as file:
        text = file.read()

    with open(file_target, 'w') as file:
        file.write(text[::-1])

copy_and_reverse('story.txt', 'story_copy.txt')     
\end{listing}

\subsubsection{Example 3: Statistics}

A code that gives the number of words, characters and lines in a text. My code does not give the correct answer. The difference is that I counted over the whole text and Colt counted over lines. My code misses some words because it joins the last word with the first of some paragraphs
\begin{listing}{1}
def statistics(file_source):
    _keys = ['lines', 'words', 'characters']
    with open(file_source) as file:
        text = file.read()
        file.seek(0)
        text_list = file.readlines()
        _values = [len(text_list), len(text.split(' ')), len(text)]
        # _values.append(len(text_list)) # len(text_list)
        # _values.append(len(text.split(' ')))
        # _values.append(len(text))
    return dict(zip(_keys,_values))

print(statistics('story.txt'))    
\end{listing}
Actually, observe that the difference between my answer \(1974\) and the correct answer \(2145\) is the number of lines minus one, that is \(171\). This is because the last words combines with the fist word of the paragraphs, except in the last line. 

A better code is the following 
\begin{listing}{1}
def statistics(file_source):
    _keys = ['lines', 'words', 'characters']
    with open(file_source) as file:
        text = file.read()
        file.seek(0)
        text_list = file.readlines()

        lines = len(text_list)

        words_per_line = tuple(len(text_line.split(' ')) for text_line in text_list)

        words = sum(words_per_line)

        characters = len(text) 

        _values = [lines, words, characters]

    return dict(zip(_keys,_values))

print(statistics('story.txt'))    
\end{listing}

Colt's solution is very short
\begin{listing}{1}
def statistics(file_name):
    with open(file_name) as file:
        lines = file.readlines()
 
    return { "lines": len(lines),
             "words": sum(len(line.split(" ")) for line in lines),
             "characters": sum(len(line) for line in lines) }    
\end{listing}


\subsubsection{Example 4: Find and replace}

Here I want to write a code that search in a file, and we then replace a work with another. There is a \verb|replace()| method for strings, see documentation \verb|help(str)| and see also~\cite{strings:replace}. 

My solution is more naive
\begin{listing}{1}
def find_and_replace(source_file, name_search, name_replace):
    with open(source_file) as file:
        text = file.read()
        replace_text = text.replace(name_search, name_replace)

    with open(source_file, 'w') as file:        
        file.write(replace_text)    
\end{listing}

Colt's solution 
\begin{listing}{1}
def find_and_replace(file_name, old_word, new_word):
    with open(file_name, "r+") as file:
        text = file.read()
        new_text = text.replace(old_word, new_word)
        file.seek(0)
        file.write(new_text)
        file.truncate()    
\end{listing}

\section{Extras}

Here are notes of some extra material that I am not particularly keen to investigate now, but that might be useful in the future. 

\begin{itemize}
    \item See pep8. See Colt's section on this topic. See section 11, lesson 96.
    \item \verb|Debugging|. See section 21 of the course.
    \item Tests: \verb|Assert|, \verb|doctest| and \verb|unittest|. See section 29.
\end{itemize}

\section{Comma Separated Values - CSV}

It a popular format for formular data. We can use the \verb|csv| module to work with this format. 

\subsection{Reading}

Here we have 
\begin{itemize}
    \item \verb|reader| : it lets us to iterate over rows of the CSV as lists.
    \item \verb|DictReader| : it lets us to iterate over rows of the CSV as OrderedDicts.
\end{itemize}

\subsubsection{reader}

Here we have the following example
\begin{listing}{1}
from csv import reader
with open('fighters.csv') as file:
    csv_reader = reader(file)
    for row in csv_reader:
        print(row)    
\end{listing}

We can work with this objects as follows, 
\begin{listing}{1}
from csv import reader
with open('fighters.csv') as file:
    csv_reader = reader(file)
    next(csv_reader)
    for fighter in csv_reader:
        print(f'{fighter[0]} is from {fighter[1]}')    
\end{listing}
We add the line \verb|next(csv_reader)| because we do not want to print the headers, otherwise the header would also be printed. 

We could also convert this data into a list (the \verb|csv_reader| is an iterator)
\begin{listing}{1}
from csv import reader
with open('fighters.csv') as file:
    csv_reader = reader(file)
    data = list(csv_reader)
    print(data)    
\end{listing}

\subsubsection{DictReader}

We use the syntax 
\begin{listing}{1}
from csv import DictReader
with open('fighters.csv') as file:
    csv_reader = DictReader(file)
    for row in csv_reader:
        print(row)    
\end{listing}
The rows are now an ordered dictionary.

\begin{shaded}{Notes}
It is important to mention that the delimeter does not need to be a \emph{comma}, it can be other symbols, as long as it is consistent all along the file. For example 
\begin{listing}{1}
from csv import reader
with open('foobar.csv') as file:
    csv_reader = reader(file, delimiter = "|")
    for row in csv_reader:
        print(row)
\end{listing}
\end{shaded}


\subsection{Writing}

Here we use 
\begin{itemize}
    \item \verb|writer| : creates a writer object for writing CSV
    \item \verb|DictWriter| : it creates a writer object for writing using dictionaries 
\end{itemize}

\subsubsection{writer}

For example 
\begin{listing}{1}
from csv import writer
with open('fighters.csv', 'w') as file:
    csv_writer = writer(file)    
    csv_writer.writerow(["character", "Move"])
    csv_writer.writerow(["Ryu", "Hadouken"])
\end{listing}

\paragraph{Example} Let us copy the information of a CSV file and create another file with everything capitalized. 
\begin{listing}{1}
from csv import reader, writer

with open('fighters.csv') as file:
    csv_reader = reader(file)
    fighters = [[s.upper() for s in row] for row in csv_reader]
    # for row in fighters:
    #     print(row)

with open('screaming_fighters.csv', 'w') as file:
    csv_writer = writer(file)
    for fighter in fighters:
        csv_writer.writerow(fighter)    
\end{listing}

We can also nest everything 
\begin{listing}{1}
from csv import reader, writer

with open('fighters.csv') as file:
    csv_reader = reader(file)
    with open('screaming_fighters.csv', 'w') as file:
        csv_writer = writer(file)
        for fighter in csv_reader:
            csv_writer.writerow([s.upper() for s in fighter])    
\end{listing}


\subsubsection{DictWriter}
 
This case is more convoluted since we need more structures. For example
\begin{listing}{1}
from csv import DictWriter

with open("more_fighters.csv", "w") as file:
    headers = ["Character" , "Move"]
    csv_writer = DictWriter(file, fieldnames=headers)
    csv_writer.writeheader()
    csv_writer.writerow({"Character": "Ryu", "Move": "Hadouken"})
\end{listing}

\paragraph{Example}

Now we want to convert cm to inches in the list of fighters. See Colt's solution. 
\begin{listing}{1}
from csv import DictReader, DictWriter

def cm_to_in(cm):
    return float(cm) * 0.393701

with open("fighters.csv") as file:
    csv_reader = DictReader(file)
    fighters = list(csv_reader)

with open("inches_fighters.csv", "w") as file:
    headers = ("Name","Country","Height")
    csv_writer = DictWriter(file, fieldnames=headers)
    csv_writer.writeheader()
    for f in fighters:
        csv_writer.writerow({
            "Name": f["Name"],
            "Country": f["Country"],
            "Height": cm_to_in(f["Height (in cm)"])
        })    
\end{listing}

\paragraph{Exercise} In this exercise, we take add information to a existing CSV. My solution
\begin{listing}{1}
from csv import reader, writer

def add_user(first, last):
    with open("users.csv", "r") as file:
        csv_file = list(reader(file))
        csv_file.append([first, last])
    
    with open("users.csv", "w") as file:
        csv_write = writer(file)
        for row in csv_file:
            csv_write.writerow(row)    
\end{listing}

Colt's solution
\begin{listing}{1}
import csv

def add_user(first_name, last_name):
    with open("users.csv", "a") as csvfile:
        csv_writer = csv.writer(csvfile)
        csv_writer.writerow([first_name, last_name])    
\end{listing}

\paragraph{Exercise} Here is another exercise that prints the full name of the users. 
\begin{listing}{1}
from csv import DictReader

def print_users():
    with open("users.csv", "r") as file:
        csv_reader = DictReader(file)
        for user in csv_reader:
            print(f"{user['First Name']} {user['Last Name']}")

print_users()    
\end{listing}

\paragraph{Example}

Here, another example that finds the arguments and tells the index
\begin{listing}{1}
from csv import reader

def find_user(first, last):
    with open("users.csv") as file:
        csv_list = list(reader(file))
        for row in csv_list:
            if row[0] == first and row[1] == last:
                result = csv_list.index(row)
                break
            else:
                result = "{} {} not found.".format(first, last)
    return result    
\end{listing}

Colt's solution gives
\begin{listing}{1}
import csv

def find_user(first_name, last_name):
    with open("users.csv") as csvfile:
        csv_reader = csv.reader(csvfile)
        for (index, row) in enumerate(csv_reader):
            first_name_match = first_name == row[0]
            last_name_match = last_name == row[1]
            if first_name_match and last_name_match:
                return index
        return "{} {} not found.".format(first_name, last_name)    
\end{listing}

\paragraph{Example} Another example, this exercise changes the entries and counts how many entries we have changed. 
\begin{listing}{1}
def update_users(old_first, old_last, new_first, new_last):
    counter = 0
    with open("users.csv") as csvfile:
        csv_list = list(csv.reader(csvfile))
        for row in csv_list:
            if row[0] == old_first and row[1] == old_last:
                counter += 1
                row[0] = new_first
                row[1] = new_last    

        with open('users.csv', 'w') as csvfile:
            csv_writer = csv.writer(csvfile)
            for row in csv_list:
                csv_writer.writerow(row)

    print(counter)    
\end{listing}

Colt's solution 
\begin{listing}{1}
import csv

def update_users(old_first, old_last, new_first, new_last):
    with open("users.csv") as csvfile:
        csv_reader = csv.reader(csvfile)
        rows = list(csv_reader)
    
    count = 0
    with open("users.csv", "w") as csvfile:
        csv_writer = csv.writer(csvfile)
        for row in rows:
            if row[0] == old_first and row[1] == old_last:
                csv_writer.writerow([new_first, new_last])
                count += 1
            else:
                csv_writer.writerow(row)
    
    return "Users updated: {}.".format(count)    
\end{listing}


\paragraph{Example} Finally, this example deletes the user. 

\begin{listing}{1}
import csv

def delete_users(first, last):
    counter = 0
    with open("users.csv") as csvfile:
        csv_list = list(csv.reader(csvfile))
        for row in csv_list:
            if row[0] == first and row[1] == last:
                j = csv_list.count(row)
                counter = j
                while j > 0:
                    j -= 1
                    csv_list.remove(row)

        with open('users.csv', 'w') as csvfile:
            csv_writer = csv.writer(csvfile)
            for row in csv_list:
                csv_writer.writerow(row)

    return "Users deleted: {}.".format(counter)    
\end{listing}

Colt's solution
\begin{listing}{1}
import csv

def delete_users(first_name, last_name):
    with open("users.csv") as csvfile:
        csv_reader = csv.reader(csvfile)
        rows = list(csv_reader)
    
    count = 0
    with open("users.csv", "w") as csvfile:
        csv_writer = csv.writer(csvfile)
        for row in rows:
            if row[0] == first_name and row[1] == last_name:
                count += 1
            else:
                csv_writer.writerow(row)
    
    return "Users deleted: {}.".format(count)    
\end{listing}



\subsection{Pickling}

Thie idea here is that we can store (pickle) a data in a serialized manner. See example 
\begin{listing}{1}
import pickle
class Animal:
    def __init__(self, name, species):
        self.name = name
        self.species = species

    def __repr__(self):
        return f"{self.name} is a {self.species}"

    def make_sound(self, sound):
        print(f"this animal says {sound}")


class Cat(Animal):
    def __init__(self, name, breed, toy):
        super().__init__(name, species="Cat") # Call init on parent class
        self.breed = breed
        self.toy = toy

    def play(self):
        print(f"{self.name} plays with {self.toy}")


blue = Cat("Blue", "Scottish Fold", "String")

# To pickle an object:
with open("pets.pickle", "wb") as file:
    pickle.dump(blue, file)

# To unpickle something:
# with open("pets.pickle", "rb") as file:
# 	zombie_blue = pickle.load(file)
# 	print(zombie_blue)
# 	print(zombie_blue.play())    
\end{listing}

See also \verb|jsonpickling|.

\section{SQL and Python}

Here I want to understand how to handle data. SQLite3 is a simple language, so we start here. First of all, the command \verb|.help| 
is, naturally, the help.

\subsection{SQL crash course}

There are four types of data we can store, these are
\begin{itemize}
    \item Integers
    \item Reals
    \item Blobs - these data types are stored as they are. 
\end{itemize}

\subsubsection{Creating tables}

The syntax for creating tables is very straightforward. As a simple example, consider the following table
\begin{listing}{1}
CREATE TABLE dogs(
    name=TEXT,
    breed=TEXT,
    age=INTEGER
)
\end{listing}
It is a convention to have SQL commands capitalized. If we are inside the sqlite interface, this command above is 
lost when we close the window, so we need to save it. So, we save it in a file \verb|.db|. For example
\begin{listing}{1}
.open dogs_db.db -- it creates the file
CREATE TABLE dogs(
    name=TEXT,
    breed=TEXT,
    age=INTEGER
);
\end{listing}
We can check the existing tables with the command \verb|.tables|.

\subsubsection{Inserting data}

Now we need to learn how to insert data to our files. It is done with the command
\begin{listing}{1}
INSERT INTO dogs (name, breed, age) VALUES ("Blue", "Scottish Fold", 3);
SELECT * FROM cats;
\end{listing}   

\subsection{SQL + Python}

Now we would like to manage SQL tables with python. Here we can import a module to work with 
SQL. It is the \verb|sqlite3|. 
\begin{listing}{1}
import sqlite3
conn = sqlite3.connect("books.db")
# Create cursor
c = conn.cursor()
# Execute some sql
c.execute(" CREATE TABLE books (Author TEXT, Title TEXT, Publication Year INTEGER);")
# Commit changes
c.commit()

conn.close()begin{listing}{1}
import sqlite3 
conn = sqlite3.connect("books.db")

conn.close()
\end{listing}
The second command opens an existing database, or create one. The other commands are easy to understand. 
At the end, we need to close our database. Now we want to add information to this table.
\begin{listing}{1}
import sqlite3
conn = sqlite3.connect("books.db")
# Create cursor
c = conn.cursor()

# Execute some sql
# c.execute(" CREATE TABLE books (Author TEXT, Title TEXT, Publication Year INTEGER);")
data = ("Steven Weinberg", "Quantum Field Theory", 1997)
query = "INSERT INTO books VALUES (?,?,?)"
c.execute(query, data)

# Commit changes
conn.commit()
conn.close()
\end{listing}

\clearpage

\bibliographystyle{utphys}
\bibliography{library.bib}

\end{document}
