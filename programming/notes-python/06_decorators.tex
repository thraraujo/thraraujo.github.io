\section{Decorators}

Before starting \verb|Decorators|, we need to review some concepts on functions. 

\subsection{Higher Order functions} For example, higher order functions, that is, functions that accept other functions as arguments. For example
\begin{listing}{1}
def sum(n, func):
    total = 0
    for num in range(n):
        total += func(num)
    return total

def square(x):
    return x ** 2        
\end{listing}

We can nest functions. 
\begin{listing}{1}
from random import choice
def greet(person):
    def get_mood():
        msg = choice(('Hello there', 'Go away', 'I love you'))
        return msg
    result = get_mood() + ' ' + person 
    return result

print(greet('Toby'))
\end{listing}
In this case, we have returned the results of nesting functions one inside the other. But we can also return the functions 
\begin{listing}{1}
from random import choice
def make_laugh():
    def get_laugh():
        l = choice(('HAHAHAHA', 'lol', 'hehehe'))
        return l
    
    return get_laugh
\end{listing}

We can also add an extra layer of complexity if we allow arguments. For example, 
\begin{listing}{1}
from random import choice
def make_laugh(person):
    def get_laugh():
        laugh = choice(('HAHAHAHA', 'lol', 'hehehe'))
        return f'{laugh} {person}'
    
    return get_laugh

laugh_at = make_laugh('Linda')

print(laugh_at())
print(laugh_at())
\end{listing}


\subsection{Decorators}

These are functions that wrap other functions and enhance their behavious. They are examples of higher order functions. Consider the example
\begin{listing}{1}
def be_polite(fn):
    def wrapper():
        print('What a pleasure to meet you!)
        fn()
        print('Have a good day')
    return wrapper 

def greet():
    print('My name is Thiago')    

def rage():
    print('I hate you')
\end{listing}
Observe that the function \verb|be_polite| wraps the functions \verb|greet| and |rage|. Then we can call them as follows
\begin{listingcont}
greet = be_polite(greet)    
polite_rage = be_polite(rage)

greet()
polite_rage()
\end{listingcont}
Observe that I had to define new functions \verb|greet| and \verb|polite_rage| to call the be polite function. 

The decorators help us with these definitions. We can improve the previous example as follows
\begin{listing}{1}
def be_polite(fn):
    def wrapper():
        print('What a pleasure to meet you!)
        fn()
        print('Have a good day')
    return wrapper 

@be_polite    
def greet():
    print('My name is Thiago')    

@be_polite
def rage():
    print('I hate you')
\end{listing}
And now, we can call the function as 
\begin{listingcont}
greet()
polite_rage()
\end{listingcont}

\subsubsection{Decorators with different signatures.}

Now, suppose that we want to consider different arguments to our functions. The syntas is easy, we just need to use the \verb|*args| and \verb|**kwargs|. 

Consider the following example
\begin{listing}{1}
def shout(fn):
    def wrapper(*args, **kwargs):
        return fn(*args, **kwargs).upper()
    return wrapper

@shout 
def greet(name):
    return f"Hi, I am {name}"    

@shout 
def order(main, side):
    return f"Hi, I'd like the {main}, with a side of {side}, please"
    
@shout 
def lol():
    return 'lol'
\end{listing}
and we call them 
\begin{listingcont}
print(greet('todd'))
print(order('burger','fries'))
print(lol())
\end{listingcont}

\subsection{Using decorators, several examples}

\subsubsection{Using decorators to preserve metadata}

Consider the following code 
\begin{listing}{1}
def log_function_data(fn):
    def wrapped(*args, **kwargs):
        print(f"you are about to call {fn.__name__}")
        print(f"here's the documentation: {fn.__doc__}")
        return fn(*args, **kwargs)
    return wrapper 

@log_function_data
def add(x, y):
    '''Adds two numbers together'''
    return x + y
\end{listing}
but the metadata is lost with the wrapper. See the odd behaviour if we call the documentation
\begin{listingcont}
print(add.__doc__)
print(add.__name__)
help(help)    
\end{listingcont}

We can solve this problem using the module \verb|functools|. In particular, 
\begin{listing}{1}
from functions import wraps

def log_function_data(fn):
    @wraps(fn)
    def wrapped(*args, **kwargs):
        print(f"you are about to call {fn.__name__}")
        print(f"here's the documentation: {fn.__doc__}")
        return fn(*args, **kwargs)
    return wrapper 

@log_function_data
def add(x, y):
    '''Adds two numbers together'''
    return x + y
\end{listing}
Now, things work correctly
\begin{listingcont}
print(add.__doc__)
print(add.__name__)
help(help)    
\end{listingcont}

\subsubsection{Building a speed-test with decorators}

Using decorators, we can build a simple spped-test, for example 
\begin{listing}{1}
from time import time 
from functools import wraps 

def speed_test(fn):
    def wrapper(*args, **kwargs):
        start_time = time()
        result = fn(*args, **kwargs)
        end_time = time()
        print(f"Executing {fn.__name__}")
        print(f"Time elapsed: {end_time - start_time}")
        return result 
    return wrapper 

@speed_test 
def sum_nums():
    return sum(x for x in range(10000000))

print(sum_nums())    
\end{listing}

\paragraph{Exercises} In this exercise, Colt defines a function that shows the arguments before executing the function: 
\begin{listing}{1}
from functools import wraps    

def show_args(fn):
    def wrapper(*args, **kwargs):
        _args = tuple([arg for arg in args])
        _kwargs = {item : key for item, key in kwargs.items()}
        print(f'Here are the args: {_args}')
        print(f'Here are the kwargs: {_kwargs}')
        return fn(*args, **kwargs)
    return wrapper
        
@show_args
def do_nothing(*args, **kwargs):
    pass

do_nothing(1, 2, 3,a="hi",b="bye")    
\end{listing}

His solution is simpler 
\begin{listing}{1}
from functools import wraps

def show_args(fn):
    @wraps(fn)
    def wrapper(*args, **kwargs):
        print("Here are the args:", args)
        print("Here are the kwargs:", kwargs)
        return fn(*args, **kwargs)
    return wrapper    
\end{listing}

\subsubsection{Ensuring Args with a decorator}

One can also use decorators to prevent misuse of functions. For example, avoid that someone uses \verb|**kwargs| when we (the developers) expected \verb|*args|. For example 
\begin{listing}{1}
from functools import wraps

def ensure_no_kwargs(fn):
    @wraps(fn)
    def wrapper(*args, **kwargs):
        if kwargs:
            raise ValueError("No kwargs allowed! sorry :(")
        return fn(*args, **kwargs)
    return wrapper

@ensure_no_kwargs
def greet(name):
    print(f"hi there {name}")

greet(name="Tony")    
\end{listing}


\paragraph{Exercise} Another problem is the following boilerplate code 
\begin{listing}{1}
from functools import wraps 

def double_return(fn):
    @wraps(fn)
    def wrapper(*args, **kwargs):
        return [fn(*args, **kwargs), fn(*args, **kwargs)]
    return wrapper

@double_return 
def add(x, y):
    return x + y
    
@double_return
def greet(name):
    return "Hi, I'm " + name

print(add(1, 2)) # [3, 3]
print(greet("Colt")) # ["Hi, I'm Colt", "Hi, I'm Colt"]    
\end{listing}

\paragraph{Ensuring a limited number of arguments} Suppose now that we want a function that accepts a maximum of \(3\) arguments.
\begin{listing}{1}
from functools import wraps

def ensure_fewer_than_three_args(fn):
    def wrapper(*args, **kwargs):
        if kwargs:
            raise ValueError("No kwargs allowed! sorry :(")
        elif len(args) >= 3:
            return "Too many arguments!"
        return fn(*args, **kwargs)
    return wrapper

@ensure_fewer_than_three_args
def add_all(*nums):
    return sum(nums)

print(add_all()) # 0
print(add_all(1)) # 1
print(add_all(1,2)) # 3
print(add_all(1,2,3)) # "Too many arguments!"
print(add_all(1,2,3,4,5,6)) # "Too many arguments!"    
\end{listing}

\paragraph{Example} Another example with arguments with constraits.
\begin{listing}{1}
def only_ints(fn):
    def wrapper(*args, **kwargs):
        truth = [True if type(arg) == int else False for arg in args]
        if all(truth) == False:
            return "Please only invoke with integers."
        return fn(*args, **kwargs)
    return wrapper


@only_ints 
def add(x, y):
    return x + y
    
print(add(1, 2)) # 3
print(add("1", "2")) # "Please only invoke with integers."    
\end{listing}

Colt uses \verb|any| in this case 
\begin{listing}{1}
from functools import wraps

def only_ints(fn):
    @wraps(fn)
    def inner(*args, **kwargs):
        if any([arg for arg in args if type(arg) != int]):
            return "Please only invoke with integers."
        return fn(*args, **kwargs)
    return inner    
\end{listing}


\subsubsection{Ensuring users with decorators}

Now, we write a code to ensure authorized users. 
\begin{listing}{1}
def ensure_authorized(fn):
    def wrapper(*args, **kwargs):
        if ('role', 'admin') not in kwargs.items():
            return "Unauthorized"
        return fn(*args, **kwargs)
    return wrapper 

@ensure_authorized
def show_secrets(*args, **kwargs):
    return "Shh! Don't tell anybody!"

print(show_secrets(role="admin")) # "Shh! Don't tell anybody!"
print(show_secrets(role="nobody")) # "Unauthorized    
\end{listing}
Colt's solution 
\begin{listing}{1}
def ensure_authorized(fn):
    @wraps(fn)
    def wrapper(*args, **kwargs):
        if kwargs.get("role") == "admin":
            return fn(*args, **kwargs)
        return "Unauthorized"
    return wrapper    
\end{listing}


\subsection{Decorators that accept arguments} For example, we want to ensure that the first arguments in a function always take a chosen value. We need an extra layer of function, pay attention to the following 
\begin{listing}{1}
# When we write:
@decorator
def func(*args, **kwargs):
    pass
# We're really doing:
func = decorator(func)


# When we write:
@decorator_with_args(arg)
def func(*args, **kwargs):
    pass
# We're really doing:
func = decorator_with_args(arg)(func)    
\end{listing}

See the working example below.
\begin{listing}{1}
from functools import wraps

def ensure_first_arg_is(val):
    def inner(fn):
        @wraps(fn)
        def wrapper(*args, **kwargs):
            if args and args[0] != val:
                return f"First arg needs to be {val}"
            return fn(*args, **kwargs)
        return wrapper
    return inner

@ensure_first_arg_is("burrito")
def fav_foods(*foods):
    print(foods)

print(fav_foods("burrito", "ice cream")) # ('burrito', 'ice cream')
print(fav_foods("ice cream", "burrito")) # 'Invalid! First argument must be burrito'

@ensure_first_arg_is(10)
def add_to_ten(num1, num2):
    return num1 + num2

print(add_to_ten(10, 12)) # 12
print(add_to_ten(1, 2)) # 'Invalid! First argument must be 10'    
\end{listing}

\subsubsection{Enforcing types with decorators}

See the example below 
\begin{listing}{1}
def enforce(*types):
    def decorator(f):
        def new_func(*args, **kwargs):
            #convert args into something mutable   
            newargs = []        
            for (arg, type) in zip(args, types):
               newargs.append( type(arg)) 
            return f(*newargs, **kwargs)
        return new_func
    return decorator

@enforce(str, int)
def repeat_msg(msg, times):
	for time in range(times):
		print(msg)

@enforce(float, float)
def divide(a,b):
	print(a/b)
# repeat_msg("hello", '5')
divide('1', '4')    
\end{listing}


\subsubsection{Delay function}

This is another example that uses the module \verb|time|. Here we want to wait a bit before executing the function. Here is my solution
\begin{listing}{1}
from time import sleep

def delay(interval):
    def inner(fn):
        def wrapper(*args, **kwargs):
            print("Waiting {}s before running say_hi".format(interval))
            sleep(interval)
            return fn(*args, **kwargs)
        return wrapper
    return inner

@delay(3)
def say_hi():
    return "hi"    

print(say_hi(3))       
\end{listing}

Here is Colt's solution
\begin{listing}{1}
from functools import wraps
from time import sleep
    
def delay(timer):
    def inner(fn):
        @wraps(fn)
        def wrapper(*args, **kwargs):
            print("Waiting {}s before running {}".format(timer, fn.__name__))
            sleep(timer)
            return fn(*args, **kwargs)
        return wrapper
    return inner    
\end{listing}