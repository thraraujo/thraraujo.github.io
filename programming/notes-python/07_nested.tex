\section{Nested functions}

Here I would like to understand a bit better some important details of nested functions. The main reason is that I want to understand two examples of exercises that Colt's proposed and I do not completely understand his solutions. In particular, let me start with the problems, my solution and his solution. 

These problems can be found in the last of the course. I have solved them in the \verb|exercises.py|.

\paragraph{\bf Problem 01:} Running the average

In this example, we need to define a function that runs the average of the previous values.

Here is my solution
\begin{listing}{1}
    def running_average():
        values = []
        def average(num):
                values.append(num)
                return round(sum(values) / len(values),1)
        return average
fn = running_average()        
print(fn(10))
print(fn(11))
print(fn(12))
fn = running_average()        
print(fn(1))
print(fn(3))
\end{listing}	

We need to compare it with Colt's solution 
\begin{listing}{1}
def running_average():
    running_average.accumulator = 0
    running_average.size = 0

    def inner(number):
        running_average.accumulator += number
        running_average.size += 1
        return running_average.accumulator / running_average.size

    return inner    
\end{listing}

Observe that we have two methods that I do not understand very well, namely \verb|accumulator| and \verb|size|.

\paragraph{\bf Problem 02:} Running the average

The second problem is 
\begin{listing}{1}
def once(fn):
    n = []
    def test(*args):
            if len(n) < 1:
                    n.append(0)
                    return fn(*args)
            else:
                    return None
    return test

def add(a,b):
    return a + b

oneAddition = once(add)

print(oneAddition(2,2)) # 4
print(oneAddition(2,2)) # None
print(oneAddition(12,200)) # None     
\end{listing}

With Colt's solution given by
\begin{listing}{1}
def once(fn):
    fn.is_called = False
    def inner(*args):
        if not(fn.is_called):
            fn.is_called = True
            return fn(*args)
    return inner    
\end{listing}

\subsection{Closures}

What is annoying in my solutions is that I had to define an empty list as a counter. In the first example, the list is reasonable and it can be considered an okay solution. In the second example, although the list works, it is an ugly workaround. It should be better to use the counter as an \emph{scalar} (it is not a pythonic terminologt). This has to do with scopes. 

There are some good articles explaining the situations above. Some examples articles (clickable links):
\begin{itemize}
    \item \href{https://zetcode.com/python/python-closures/}{Python closures}
    \item \href{https://www.linkedin.com/pulse/5-essential-aspects-python-closures-sagar-an/}{5 Essential Aspects of Python Closures}
    \item \href{https://www.geeksforgeeks.org/python-inner-functions/}{Python Inner Functions}
    \item \href{https://stackoverflow.com/questions/21959985/why-cant-python-increment-variable-in-closure}{Why can't Python increment variable in closure?}
    \item \href{https://stackoverflow.com/questions/38693236/python-counter-with-closure}{python counter with closure}
    \item \href{https://stackoverflow.com/questions/36901798/pythons-closure-local-variable-referenced-before-assignment}{Python's closure - local variable referenced before assignment}
    \item \href{https://stackoverflow.com/questions/64946483/dot-notation-in-pythons-function-definition}{dot notation in Python's function definition}
\end{itemize}

The problem we have above is the \verb|scope|. We have two keywords \verb|global| and \verb|nonlocal|. See the examples of how they work. 

\paragraph{\bf Global} It is used when we want to use a global variable in a function. For example: 
\begin{listing}{1}
# EXAMPLE OF A SCOPING PROBLEM:
total = 0

def increment():
    total += 1
    return total

print(increment()) # Error! 
# "I can't find a variable named total in this function"
\end{listing}    
but we can solve it as 
\begin{listing}{1}
total = 0

def increment():
    global total #use the global variable total
    total += 1
    return total

print(increment()) # 1
print(increment()) # 2
print(increment()) # 3    
\end{listing}
and now it works. 

\paragraph{\bf Nonlocal} Nonlocal variables can be used in nested functions (as the examples we have above). Actually, it allows us the change a parent variable in a child function. For example 
\begin{listing}{1}
def outer():
    count = 0 
    def inner():
        nonlocal count
        count += 1
        return count 
    return inner()
\end{listing}