\section{Comma Separated Values - CSV}

It a popular format for formular data. We can use the \verb|csv| module to work with this format. 

\subsection{Reading}

Here we have 
\begin{itemize}
    \item \verb|reader| : it lets us to iterate over rows of the CSV as lists.
    \item \verb|DictReader| : it lets us to iterate over rows of the CSV as OrderedDicts.
\end{itemize}

\subsubsection{reader}

Here we have the following example
\begin{listing}{1}
from csv import reader
with open('fighters.csv') as file:
    csv_reader = reader(file)
    for row in csv_reader:
        print(row)    
\end{listing}

We can work with this objects as follows, 
\begin{listing}{1}
from csv import reader
with open('fighters.csv') as file:
    csv_reader = reader(file)
    next(csv_reader)
    for fighter in csv_reader:
        print(f'{fighter[0]} is from {fighter[1]}')    
\end{listing}
We add the line \verb|next(csv_reader)| because we do not want to print the headers, otherwise the header would also be printed. 

We could also convert this data into a list (the \verb|csv_reader| is an iterator)
\begin{listing}{1}
from csv import reader
with open('fighters.csv') as file:
    csv_reader = reader(file)
    data = list(csv_reader)
    print(data)    
\end{listing}

\subsubsection{DictReader}

We use the syntax 
\begin{listing}{1}
from csv import DictReader
with open('fighters.csv') as file:
    csv_reader = DictReader(file)
    for row in csv_reader:
        print(row)    
\end{listing}
The rows are now an ordered dictionary.

\begin{shaded}{Notes}
It is important to mention that the delimeter does not need to be a \emph{comma}, it can be other symbols, as long as it is consistent all along the file. For example 
\begin{listing}{1}
from csv import reader
with open('foobar.csv') as file:
    csv_reader = reader(file, delimiter = "|")
    for row in csv_reader:
        print(row)
\end{listing}
\end{shaded}


\subsection{Writing}

Here we use 
\begin{itemize}
    \item \verb|writer| : creates a writer object for writing CSV
    \item \verb|DictWriter| : it creates a writer object for writing using dictionaries 
\end{itemize}

\subsubsection{writer}

For example 
\begin{listing}{1}
from csv import writer
with open('fighters.csv', 'w') as file:
    csv_writer = writer(file)    
    csv_writer.writerow(["character", "Move"])
    csv_writer.writerow(["Ryu", "Hadouken"])
\end{listing}

\paragraph{Example} Let us copy the information of a CSV file and create another file with everything capitalized. 
\begin{listing}{1}
from csv import reader, writer

with open('fighters.csv') as file:
    csv_reader = reader(file)
    fighters = [[s.upper() for s in row] for row in csv_reader]
    # for row in fighters:
    #     print(row)

with open('screaming_fighters.csv', 'w') as file:
    csv_writer = writer(file)
    for fighter in fighters:
        csv_writer.writerow(fighter)    
\end{listing}

We can also nest everything 
\begin{listing}{1}
from csv import reader, writer

with open('fighters.csv') as file:
    csv_reader = reader(file)
    with open('screaming_fighters.csv', 'w') as file:
        csv_writer = writer(file)
        for fighter in csv_reader:
            csv_writer.writerow([s.upper() for s in fighter])    
\end{listing}


\subsubsection{DictWriter}
 
This case is more convoluted since we need more structures. For example
\begin{listing}{1}
from csv import DictWriter

with open("more_fighters.csv", "w") as file:
    headers = ["Character" , "Move"]
    csv_writer = DictWriter(file, fieldnames=headers)
    csv_writer.writeheader()
    csv_writer.writerow({"Character": "Ryu", "Move": "Hadouken"})
\end{listing}

\paragraph{Example}

Now we want to convert cm to inches in the list of fighters. See Colt's solution. 
\begin{listing}{1}
from csv import DictReader, DictWriter

def cm_to_in(cm):
    return float(cm) * 0.393701

with open("fighters.csv") as file:
    csv_reader = DictReader(file)
    fighters = list(csv_reader)

with open("inches_fighters.csv", "w") as file:
    headers = ("Name","Country","Height")
    csv_writer = DictWriter(file, fieldnames=headers)
    csv_writer.writeheader()
    for f in fighters:
        csv_writer.writerow({
            "Name": f["Name"],
            "Country": f["Country"],
            "Height": cm_to_in(f["Height (in cm)"])
        })    
\end{listing}

\paragraph{Exercise} In this exercise, we take add information to a existing CSV. My solution
\begin{listing}{1}
from csv import reader, writer

def add_user(first, last):
    with open("users.csv", "r") as file:
        csv_file = list(reader(file))
        csv_file.append([first, last])
    
    with open("users.csv", "w") as file:
        csv_write = writer(file)
        for row in csv_file:
            csv_write.writerow(row)    
\end{listing}

Colt's solution
\begin{listing}{1}
import csv

def add_user(first_name, last_name):
    with open("users.csv", "a") as csvfile:
        csv_writer = csv.writer(csvfile)
        csv_writer.writerow([first_name, last_name])    
\end{listing}

\paragraph{Exercise} Here is another exercise that prints the full name of the users. 
\begin{listing}{1}
from csv import DictReader

def print_users():
    with open("users.csv", "r") as file:
        csv_reader = DictReader(file)
        for user in csv_reader:
            print(f"{user['First Name']} {user['Last Name']}")

print_users()    
\end{listing}

\paragraph{Example}

Here, another example that finds the arguments and tells the index
\begin{listing}{1}
from csv import reader

def find_user(first, last):
    with open("users.csv") as file:
        csv_list = list(reader(file))
        for row in csv_list:
            if row[0] == first and row[1] == last:
                result = csv_list.index(row)
                break
            else:
                result = "{} {} not found.".format(first, last)
    return result    
\end{listing}

Colt's solution gives
\begin{listing}{1}
import csv

def find_user(first_name, last_name):
    with open("users.csv") as csvfile:
        csv_reader = csv.reader(csvfile)
        for (index, row) in enumerate(csv_reader):
            first_name_match = first_name == row[0]
            last_name_match = last_name == row[1]
            if first_name_match and last_name_match:
                return index
        return "{} {} not found.".format(first_name, last_name)    
\end{listing}

\paragraph{Example} Another example, this exercise changes the entries and counts how many entries we have changed. 
\begin{listing}{1}
def update_users(old_first, old_last, new_first, new_last):
    counter = 0
    with open("users.csv") as csvfile:
        csv_list = list(csv.reader(csvfile))
        for row in csv_list:
            if row[0] == old_first and row[1] == old_last:
                counter += 1
                row[0] = new_first
                row[1] = new_last    

        with open('users.csv', 'w') as csvfile:
            csv_writer = csv.writer(csvfile)
            for row in csv_list:
                csv_writer.writerow(row)

    print(counter)    
\end{listing}

Colt's solution 
\begin{listing}{1}
import csv

def update_users(old_first, old_last, new_first, new_last):
    with open("users.csv") as csvfile:
        csv_reader = csv.reader(csvfile)
        rows = list(csv_reader)
    
    count = 0
    with open("users.csv", "w") as csvfile:
        csv_writer = csv.writer(csvfile)
        for row in rows:
            if row[0] == old_first and row[1] == old_last:
                csv_writer.writerow([new_first, new_last])
                count += 1
            else:
                csv_writer.writerow(row)
    
    return "Users updated: {}.".format(count)    
\end{listing}


\paragraph{Example} Finally, this example deletes the user. 

\begin{listing}{1}
import csv

def delete_users(first, last):
    counter = 0
    with open("users.csv") as csvfile:
        csv_list = list(csv.reader(csvfile))
        for row in csv_list:
            if row[0] == first and row[1] == last:
                j = csv_list.count(row)
                counter = j
                while j > 0:
                    j -= 1
                    csv_list.remove(row)

        with open('users.csv', 'w') as csvfile:
            csv_writer = csv.writer(csvfile)
            for row in csv_list:
                csv_writer.writerow(row)

    return "Users deleted: {}.".format(counter)    
\end{listing}

Colt's solution
\begin{listing}{1}
import csv

def delete_users(first_name, last_name):
    with open("users.csv") as csvfile:
        csv_reader = csv.reader(csvfile)
        rows = list(csv_reader)
    
    count = 0
    with open("users.csv", "w") as csvfile:
        csv_writer = csv.writer(csvfile)
        for row in rows:
            if row[0] == first_name and row[1] == last_name:
                count += 1
            else:
                csv_writer.writerow(row)
    
    return "Users deleted: {}.".format(count)    
\end{listing}



\subsection{Pickling}

Thie idea here is that we can store (pickle) a data in a serialized manner. See example 
\begin{listing}{1}
import pickle
class Animal:
    def __init__(self, name, species):
        self.name = name
        self.species = species

    def __repr__(self):
        return f"{self.name} is a {self.species}"

    def make_sound(self, sound):
        print(f"this animal says {sound}")


class Cat(Animal):
    def __init__(self, name, breed, toy):
        super().__init__(name, species="Cat") # Call init on parent class
        self.breed = breed
        self.toy = toy

    def play(self):
        print(f"{self.name} plays with {self.toy}")


blue = Cat("Blue", "Scottish Fold", "String")

# To pickle an object:
with open("pets.pickle", "wb") as file:
    pickle.dump(blue, file)

# To unpickle something:
# with open("pets.pickle", "rb") as file:
# 	zombie_blue = pickle.load(file)
# 	print(zombie_blue)
# 	print(zombie_blue.play())    
\end{listing}

See also \verb|jsonpickling|.