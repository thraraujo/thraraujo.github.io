%%%%%%%%%%%%%%%%%%%%%%%%%%%%%%%%%%%%%%%%%%%%%%
%%%%%%%%%%%%%%%%%%%%%%%%%%%%%%%%%%%%%%%%%%%%%%
%%%%%%%%%%%%%%%%%%%%%%%%%%%%%%%%%%%%%%%%%%%%%%
%%%%%%%%%%%%%%%%%%%%%%%%%%%%%%%%%%%%%%%%%%%%%%

\section{Functions}

It is process for executing a task. It is useful for executing similar procedures over and over. It is helpful to avoid repetition. DRY = Don't Repeat Yourself. 

\subsubsection{Defining a function}

Here is the syntax 
\begin{verbatim}
> def name_of_the_function():
>     # block of code
\end{verbatim}
For example, 
\begin{verbatim}
> def say_hi():
>     print("Hi")
\end{verbatim}

We can return values with \verb|return|, that is
\begin{verbatim}
> def say_hi():
>     return "Hi"
\end{verbatim}
The \verb|return| exists the function. With return we can use the result in a variable. 

Let us write a function that will toss a coin for us. 
\begin{verbatim}
> from random import random
> def flip_coin():
>     # generate a random number
>     r = random()
>     if r > 0.5:
>         return "Heads"
>     else:
>         return "Tails"	
\end{verbatim}

\subsubsection{Parameters}

Now we need to understand how to define functions which accepts inputs. It is very easy. 
\begin{verbatim}
> def square(var):
>     return f(var)
\end{verbatim}

For example, with the Happy Birthday:
\begin{verbatim}
> def happy(name):
>     print("Happy birthday to you")
>     print("Happy birthday to you")
>     print(f"Happy birthday dear {name}")
>     print("Happy birthday to you")
\end{verbatim}

One can also consider more than 1 argument:
\begin{verbatim}
> def multiply(a,b):
>     return a * b
\end{verbatim}

We would also like to have default parameter. For example, if we do not insert a parameter, we execute some pre-defined parameter. The example that Colt gives is interesting, we want a default value for the exponent. For example, if we do not specify, it takes the square. It is defined as follows
\begin{verbatim}
> def exp(num, power=2):
>     return num ** power
> print(exp(2,3))
8
> print(exp(2))
4
\end{verbatim}
If we know the parameters, we can specify the arguments and the order does not matter anymore, in the example above, we can do
\begin{verbatim}
> def exp(num, power=2):
>     return num ** power
> print(exp(2,3))
8
> print(exp(power = 3, num = 2))
8
\end{verbatim}

A nice example of a function. Find the intersection {\bf list}
\begin{verbatim}
> def intersection(lst1, lst2):
>     return list(set(lst1) & set(lst2))
\end{verbatim}


\paragraph{Exercise} (Colt's course): The three codes below are equivalent:

\# Code 1: 
\begin{verbatim}
> def speak(animal="dog"):
>     if animal == "pig":
>         return "oink"
>     elif animal == "duck":
>          return "quack"
>     elif animal == "cat":
>         return "meow"
>     elif animal == "dog":
>         return "woof"
>     else:
>         return "?"
\end{verbatim}

\# Code2: 
\begin{verbatim}
> def speak(animal="dog"):
>     noises = {"dog": "woof", "pig": "oink", "duck": "quack", "cat": "meow"}
>     noise = noises.get(animal)
>     if noise:
>         return noise
>     return "?"
\end{verbatim}

\# Code3:
\begin{verbatim}
> def speak(animal='dog'):
>     noises = {'pig':'oink', 'duck':'quack', 'cat':'meow', 'dog':'woof'}
>     return noises.get(animal, '?')
\end{verbatim}
The third code uses the default value of \verb|get()|.

\subsubsection{Scope} 

Parameters are now always defined everywhere in the code, it is a topic called scope, and we can see more about in lecture 160 of Colt~\cite{csteele}. See also here~\cite{rpython}.

\subsubsection{Documentation} 

We can explain what our functions do with the triple quotation marks, \verb|""" message """|. 
The nice thing about it is that we can access this message with the syntax \verb|my_function.__doc__|. For example, 
\begin{verbatim}
> def my_function():
>     """Silly function that explains something"""
>     print("hello")
>
> my_function.__doc__
'Silly function that explains something'
\end{verbatim}

\subsubsection{\(\ast\) args} It is special operator. 
\begin{verbatim}
> def sum_all_nums(num1, num2, num3):
>      return num1 + num2 + num3
\end{verbatim}
But we would like to do this sum for an arbitrary number of values, but hard coding is silly. We can do it with \verb|*-args| operator. The arguments define a tuple. 

\begin{verbatim}
> def sum_all_nums(*args):
>      total = 0
>      for num in args:
>            total += num
>      return total
\end{verbatim}
Observe that the \verb|*| needs to be present, and the \verb|args| is just a name. That is
\begin{verbatim}
> def sum_all_nums(*nums):
>      total = 0
>      for num in nums:
>            total += num
>      return total
\end{verbatim}
also works. 


\subsubsection{\(\ast\ast\)-kwargs} It gathers remaining arguments as keys in dictionaries. For example
\begin{verbatim}
> def favorite_food(**kwargs):
>     for person, food in kwargs.items():
>         print(f"{person}'s favorite food is {food}")	
\end{verbatim}
Then I can run this function with some examples:
\begin{verbatim}
> favorite_food("thiago"="hamburger", "aline"="cake")
\end{verbatim}

A code that combines words, see exercise 58.
\begin{verbatim}
> def combine_words(name, **kwargs):
>     if "prefix" in kwargs.keys():
>         return "{}{}".format(kwargs["prefix"],name)
>     elif "suffix" in kwargs.keys():
>         return "{}{}".format(name,kwargs["suffix"])
>     return name
\end{verbatim}

\subsubsection{Ordering parameters} Remember that this order is important
\begin{enumerate}
	\item parameters
	\item \(*\)-args
	\item default parameters	
	\item \(*\)-kwargs
\end{enumerate}

\subsubsection{Lists and Tuples unpacking} We want to execute a function that takes its values in a list or tuple. 
\begin{verbatim}
> def sum_all_nums(*nums):
>      total = 0
>      for num in nums:
>            total += num
>      return total
\end{verbatim}
But now, suppose we want to use the following list and tuple in the function above
\begin{verbatim}
> num1 = [1, 2, 3, 4, 5, 6]
> num2 = (1, 2, 3, 4, 5, 6)
\end{verbatim}
If we try to apply the function on \verb|num1| and \verb|num2|, we get an error because they are a list and a tuple respectively. 
We we need to do is to unpack these objects. 

\subsubsection{Dictionary unpacking} 
\begin{verbatim}
> def display_names(first, second):
>     print(f"{first} says hello to {second}")
\end{verbatim}
We get an error if we try to use this function with a dictionary, for example
\begin{verbatim}
> names = {"first" : "Thiago", "second" : "Aline"}
> display_name(names)
> Error
> display_name(**names)
\end{verbatim}

See this example
\begin{verbatim}
> def add_and_multiply_numbers(a,b,c, **kwargs)
>     print(a + b * c)
>     print("OTHER STUFF. . .")
>     print(kwargs)
\end{verbatim}
Now we can apply to the dictionary
\begin{verbatim}
> data = dict(a=1, b=2, c=3, d=55, name="Tony")
> add_and_multiply_numbers(**data, cat="Blue")
\end{verbatim}


%%%%%%%%%%%%%%%%%%%%%%%%%%%%%%%%%%%%%%%%%%%%%%
%%%%%%%%%%%%%%%%%%%%%%%%%%%%%%%%%%%%%%%%%%%%%%
%%%%%%%%%%%%%%%%%%%%%%%%%%%%%%%%%%%%%%%%%%%%%%
%%%%%%%%%%%%%%%%%%%%%%%%%%%%%%%%%%%%%%%%%%%%%%

\subsection{Lambdas}

These are special functions. Let us start with the usual definition of functions. 
\begin{verbatim}
> def square(num):
>     return num * num
\end{verbatim}
The syntax for the \verb|lambda| is given by
\begin{verbatim}
> square2 = lambda num: num * num
> add = lambda a,b: a + b
\end{verbatim}
The idea of this function is that we want to use a function that is applied in a specific place. 

\subsubsection{map}

It is a standard function that accepts at least two arguments, a function and an iterable (something that can be iterated, strings, lists, dictionaries, sets, tuples). It runs a \verb|lambda| for each value in the iterable. 
\begin{verbatim}
> num = [2, 4, 6, 8, 10]
> doubles = map(lambda x: x * 2, nums) # it is a ____
> doubles2 = list(map(lambda x: x * 2, nums))
\end{verbatim}
Or we can define the function separately
\begin{verbatim}
> def double(x): return x*2
> doubles = map(double, nums)
> doubles2 = list(map(lambda x: x * 2, nums))
\end{verbatim}

Another example
\begin{verbatim}
> names = [{'first':'Thiago', 'last':'Araujo'}, {'first':'Vanessa', 'last':'Araujo'}, 
...{'first':'Julia', 'last':'Araujo'}]
> firstnames = list(map(lambda x : x['first'], names ))
\end{verbatim}

Another example from Colt's course
\begin{verbatim}
> def extract_full_name(names):
>     return list(map(lambda x : "{} {}".format(x['first'], x['last']), names))
\end{verbatim}

\subsubsection{filters}

We can filter out some values, that is, there is a lambda for each item in the iterable. 
\begin{verbatim}
> lst = [1, 2, 3, 4]
> evens = list(filter(lambda x: x % 2 ==0, lst ))
\end{verbatim}

Colt's example of users is interesting. Suppose we want to collect the inactive users of a social media. 
\begin{verbatim}
> users = [
>     {"username": "samuel", "tweets": ["I love cake", "I love pie", "hello world!"]},
>     {"username": "katie", "tweets": ["I love my cat"]},
>     {"username": "jeff", "tweets": []},
>     {"username": "bob123", "tweets": []},
>     {"username": "doggo_luvr", "tweets": ["dogs are the best", "I'm hungry"]},
>     {"username": "guitar_gal", "tweets": []}
> ]
> #extract inactive users using filter:
> inactive_users = list(filter(lambda u: not u['tweets'], users))
> inactive_users
> [{'username': 'jeff', 'tweets': []}, {'username': 'bob123', 'tweets': []}, 
{'username': 'guitar_gal', 'tweets': []}]
\end{verbatim}

But now, suppose we want to create a list of the usernames only, then we need to combine this filter with a map. 
\begin{verbatim}
> inactive_users_names = list( map(lambda user: user['username'].upper(), 
... filter(lambda u: not u['tweets'], users)))
\end{verbatim}

We could use list comprehension to solve both problems above. 
\begin{verbatim}
> #extract inactive users using list comprehension:
> inactive_users2= [user for user in users if not user["tweets"]]
> extract usernames of inactive users w/ list comprehension
> usernames2 = [user["username"].upper() for user in users if not user["tweets"]]
\end{verbatim}


%%%%%%%%%%%%%%%%%%%%%%%%%%%%%%%%%%%%%%%%%%%%%%
%%%%%%%%%%%%%%%%%%%%%%%%%%%%%%%%%%%%%%%%%%%%%%
%%%%%%%%%%%%%%%%%%%%%%%%%%%%%%%%%%%%%%%%%%%%%%
%%%%%%%%%%%%%%%%%%%%%%%%%%%%%%%%%%%%%%%%%%%%%%

\subsection{Built-in functions}

Some important built-in functions.

\subsubsection{all} It returns True if all elements of the iterable are truthy or if the iterable is empty. 

\subsubsection{any} It returns True if any element of the iterable is truthy. If the iterable is empty, it returns false. 

\subsubsection{Generator expressions \& .getsizeof} These functions do not need the bracket. For example
\begin{verbatim}
> names = [Thiago", "Thomas", "Theo", "Thales"]
> all([name[0] == 'T' for name in names])
True
> all(name[0] == 'T' for name in names)
True
\end{verbatim}
It makes the code lighter. So, if we do not need to build a list, we should use this generator expression. Run the following code to see the size
\begin{verbatim}
> import sys
> list_comp = sys.getsizeof([x * 10 for x in range(1000)])
> gen_exp = sys.getsizeof(x * 10 for x in range(1000))
> 
> print("To do the same thing, it takes. . . ")
> print(f"List comprehension: {list_comp} bytes")
> print(f"Generator expression: {gen_exp} bytes") 
\end{verbatim}

\subsubsection{sorted} It is a built-in function that works for more general iterables, and not only in lists. It does not change the iterable .
\begin{verbatim}
> nums = [4, 6, 1, 30, 155]
> sorted(nums)
[1, 4, 6, 30, 155]
> nums
[4, 6, 1, 30, 155]
> nums.sort()
> nums
[1, 4, 6, 30, 155]
\end{verbatim}
We can also give arguments
\begin{verbatim}
> nums = [4, 6, 1, 30, 155]
> sorted(nums, reverse=True)
[155, 30, 6, 4, 1]
\end{verbatim}

But it is stronger than that. Suppose we want to organize the social media. 
\begin{verbatim}
> users = [
>     {"username": "samuel", "tweets": ["I love cake", "I love pie", "hello world!"]},
>     {"username": "katie", "tweets": ["I love my cat"]},
>     {"username": "jeff", "tweets": []},
>     {"username": "bob123", "tweets": []},
>     {"username": "doggo_luvr", "tweets": ["dogs are the best", "I'm hungry"]},
>     {"username": "guitar_gal", "tweets": []}
> ]
>
> sorted(users, key= lambda user: user['username']) # alphabetically
> sorted(users, key= lambda user: len(user['tweets'])) # activity
\end{verbatim}

Songs, by play count
\begin{verbatim}
> songs = [
> {'title': 'happy birthday', 'playcount': 1},
> {'title': 'Survive', 'playcount': 6},
> {'title': 'YMCA', 'playcount':99},
> {'title': 'Toxic', 'playcount':31}
> ]
>
> sorted(songs, key = lambda s: s['playcount'])
\end{verbatim}

\subsubsection{Max \& Min}

\paragraph{Max} It returns the maximum value.

\paragraph{Min} It returns the minimum value.

These functions are a bit obvious, but we can also give other arguments. 
\begin{verbatim}
> names = ['Arya', 'Samson', 'Dora', 'Tim', 'Ollivander']
> max(names, key=lambda n: len(n))
'Ollivander'
> max(names)
'Tim'
\end{verbatim}

\subsubsection{Reversed}

It returns the reverse of the iterable object. For example
\begin{verbatim}
> for x in reversed(range(1,10)):
>	print(x)
\end{verbatim}
If we want to use reversed with strings using the join function
\begin{verbatim}
> str[::-1]
> ''.join(list(reversed(str)))
\end{verbatim}

\subsubsection{len} We already know how it works. But it works calling the \verb|__len__()| method. 

\subsubsection{abs} Returns the absolute value of a number. There is also the function \verb|fabs| that we need to import with \verb|math| that does the same thing, but it converts to a float first. 

\subsubsection{sum} It is a sum of the elements of the iterable. We can also say how it starts. 
\begin{verbatim}
> sum([1,2,3])
6 
> sum([1,2,3], 10)
16
\end{verbatim}

\subsubsection{round} It approximates the number and gives the precision. 
\begin{verbatim}
> round(3.5122, 2)
3.51
> round(3.5122)
4
> round(3.4122)
3
\end{verbatim}

\subsubsection{zip} This function makes an iterator that aggregates elements from each of the iterables. Imagine that we have some lists. This function pairs elements of these lists. 
\begin{verbatim}
> nums1 = [1,2,3,4,5]
> nums2 = [6, 7, 8, 9, 10]
> z = zip(nums1, nums2)
\end{verbatim}

Complex example. Consider the lists
\begin{verbatim}
> midterms = [80, 91, 78]
> final = [98, 89, 53]
> students = ['dan', 'ang', 'kate']
> grades = {t[0]: max(t[1], t[2]) for t in zip(students, midterms, final)}
\end{verbatim}

Another solution using \verb|lambdas|
\begin{verbatim}
> final_grades = dict( zip( students, map( lambda 
... pair: max(pair), zip(midterms, finals))))
\end{verbatim}

