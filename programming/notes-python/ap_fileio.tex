\section{File I/O}

Here we will learn how to work with other files which are not python files. 

\subsection{Reading files}

We use the \verb|open| function. For example, we use the syntax
\begin{listing}{1}
f = open('path')
f.read()
\end{listing}    
When files reads a file, it uses a cursor. So, when we read the file, the cursor is at the end, so if we try to read the file again, it shows an empty string. We need to learn how to work with cursors.
\begin{listing}{1}
from time import sleep 

file = open('text.txt')
print(file.read())

sleep(3)

print(file.read())    
\end{listing}

\subsubsection{seek()} It is a method to manipulate the cursor. 
\begin{listing}{1}
from time import sleep 

file = open('text.txt')
print(file.read())

sleep(3)

file.seek(6) # The cursor moves to the sixth character.

print(file.read())    
\end{listing}

\subsubsection{readline()} This reads the file line by line 
\begin{listing}{1}
file = open('text.txt')
print(file.readline())
print(file.readline())
print(file.readline())
print(file.readline())
\end{listing}

\subsubsection{readlines()} Now it gives a list with the lines
\begin{listing}{1}
file = open('text.txt')
print(file.readlines())
\end{listing}

\subsubsection{close()} Finally, we need to close the file when we are done. 
\begin{listing}{1}
file = open('text.txt')
print(file.closed)
file.close()
print(file.closed)    
\end{listing}

\subsection{With statements}

We can do the same thing with a different syntax. It is the following
\begin{listing}{1}
with open('file') as file:
    data = file.read()

file.closed # True
data
\end{listing}
In this case, the file is closed automatically. 

\subsection{Writing to files} Now we need to learn how we can modify our files. Here is a good example 
\begin{listing}{1}
with open('file', 'w') as file:
    file.write("Lorem ipsum dolor sit amet\n")
    file.write("Lorem ipsum dolor sit amet")
\end{listing}
we need to specify the argument \verb|'w'|. 

This method overwrites existing files or creates new files. In order to add information to existing files, we need to understand modes. 

\subsection{Modes for opening files} Here are the common modes

\begin{itemize}
    \item \verb|r| - Read a file (no writing). This is the default mode. 
    \item \verb|w| - Write to the file. It overwrites the file.
    \item \verb|a| - Append content to the end of the file. 
    \item \verb|r+| - Read and write to a file. It only works with existing files. It does not create a new one. 
\end{itemize}

\subsection{Examples}

Here some examples to practice a bit. 

\subsubsection{Example 1: Copy}

A function that copies the content of a file and pastes to another. 
\begin{listing}{1}
def copy(file_source, file_target):
    with open(file_source) as file:
        text = file.read()

    with open(file_target, 'w') as file:
        file.write(text)

copy('story.txt', 'story_copy.txt')    
\end{listing}

\subsubsection{Example 2: Reverserd copy}

A function that copies the content of a file and pastes to another but in a reversed order. 
\begin{listing}{1}
def copy_and_reverse(file_source, file_target):
    with open(file_source) as file:
        text = file.read()

    with open(file_target, 'w') as file:
        file.write(text[::-1])

copy_and_reverse('story.txt', 'story_copy.txt')     
\end{listing}

\subsubsection{Example 3: Statistics}

A code that gives the number of words, characters and lines in a text. My code does not give the correct answer. The difference is that I counted over the whole text and Colt counted over lines. My code misses some words because it joins the last word with the first of some paragraphs
\begin{listing}{1}
def statistics(file_source):
    _keys = ['lines', 'words', 'characters']
    with open(file_source) as file:
        text = file.read()
        file.seek(0)
        text_list = file.readlines()
        _values = [len(text_list), len(text.split(' ')), len(text)]
        # _values.append(len(text_list)) # len(text_list)
        # _values.append(len(text.split(' ')))
        # _values.append(len(text))
    return dict(zip(_keys,_values))

print(statistics('story.txt'))    
\end{listing}
Actually, observe that the difference between my answer \(1974\) and the correct answer \(2145\) is the number of lines minus one, that is \(171\). This is because the last words combines with the fist word of the paragraphs, except in the last line. 

A better code is the following 
\begin{listing}{1}
def statistics(file_source):
    _keys = ['lines', 'words', 'characters']
    with open(file_source) as file:
        text = file.read()
        file.seek(0)
        text_list = file.readlines()

        lines = len(text_list)

        words_per_line = tuple(len(text_line.split(' ')) for text_line in text_list)

        words = sum(words_per_line)

        characters = len(text) 

        _values = [lines, words, characters]

    return dict(zip(_keys,_values))

print(statistics('story.txt'))    
\end{listing}

Colt's solution is very short
\begin{listing}{1}
def statistics(file_name):
    with open(file_name) as file:
        lines = file.readlines()
 
    return { "lines": len(lines),
             "words": sum(len(line.split(" ")) for line in lines),
             "characters": sum(len(line) for line in lines) }    
\end{listing}


\subsubsection{Example 4: Find and replace}

Here I want to write a code that search in a file, and we then replace a work with another. There is a \verb|replace()| method for strings, see documentation \verb|help(str)| and see also~\cite{strings:replace}. 

My solution is more naive
\begin{listing}{1}
def find_and_replace(source_file, name_search, name_replace):
    with open(source_file) as file:
        text = file.read()
        replace_text = text.replace(name_search, name_replace)

    with open(source_file, 'w') as file:        
        file.write(replace_text)    
\end{listing}

Colt's solution 
\begin{listing}{1}
def find_and_replace(file_name, old_word, new_word):
    with open(file_name, "r+") as file:
        text = file.read()
        new_text = text.replace(old_word, new_word)
        file.seek(0)
        file.write(new_text)
        file.truncate()    
\end{listing}