%%%%%%%%%%%%%%%%%%%%%%%%%%%%%%%%%%%%%%%%%%%%%%
%%%%%%%%%%%%%%%%%%%%%%%%%%%%%%%%%%%%%%%%%%%%%%
%%%%%%%%%%%%%%%%%%%%%%%%%%%%%%%%%%%%%%%%%%%%%%
%%%%%%%%%%%%%%%%%%%%%%%%%%%%%%%%%%%%%%%%%%%%%%

\section{Modules}

Modules is a way to use other python files in other files. 

\subsection{Built-in modules}

Python has several build-in modules, see here in the Python Module Index~\cite{python:index}. We can use them with the syntax 
\begin{verbatim}
> import MODULE
\end{verbatim}
For example, the \verb|random| package can be used as
\begin{verbatim}
> import random 
> my_list = ["apple", "banana", "durian", "pear"]
> random.choice(my_list)
> random.randint(1,100)
> random.shuffle(my_list)
> my_list
\end{verbatim}

We can change how we call the module using the \verb|as|
\begin{verbatim}
> import MODULE as NEW_NAME
\end{verbatim}
For example
\begin{verbatim}
> import random as rand
> rand.choice(["apple", "banana", "durian", "pear"])
> rand.randint(1,100)
> my_list = ["apple", "banana", "durian", "pear"]
> rand.shuffle(my_list)
> my_list
\end{verbatim}

We can also call just the function in the module with the syntax
\begin{verbatim}
> from MODULE import *, **
\end{verbatim}
For example
\begin{verbatim}
> from random import choice, randint
> my_list = ["apple", "banana", "durian", "pear"]
> choice(my_list)
> randint(1,100)
\end{verbatim}

We can also change how we call the module using the \verb|as|
\begin{verbatim}
> from MODULE import * as NEW_NAME_1, ** as NEW_NAME_2
\end{verbatim}
For example
\begin{verbatim}
> from random import randint as rdi, choice as cho
> rdi(1,100)
> cho([1,2,3,4,5,6,7])
\end{verbatim}

There is a nice exercise in the lectures of Colt. It checks if a given string is a python key. That is
\begin{verbatim}
> from keyword import iskeyword
> def contains_keyword(*args):
>      """Define a function that check if any word in the arguments 
... is a python keywords"""
>      # First, I can transform the arguments in a tuple 
... (because it is lighter than a list) of True and False. It is a tuple comprehension 
... with conditional logic
>      entries = tuple(True if iskeyword(ent) else False for ent in args)
>      return any(entries)
\end{verbatim}

%%%%%%%%%%%%%%%%%%%%%%%%%%%%%%%%%%%%%%%%
%%%%%%%%%%%%%%%%%%%%%%%%%%%%%%%%%%%%%%%%

\subsection{Custom modules}

It is similar to the previous case, but now I can write the module. We use two python files, and we need to call it. Given the files \verb|file1| and \verb|file2|, we have define our functions in \verb|file1| and we use these definitions in \verb|file2| with the syntax 
\begin{verbatim}
> import file1
> file1.my_function()
\end{verbatim}

We can ignore the code with the \verb|__name__| variable. in particular, I need to add 
\begin{verbatim}
if __name__ == "__main__":
\end{verbatim}
Suppose that we have two python files \verb|file1| and \verb|file2|. In \verb|first1.py|, we write
\begin{verbatim}
>>> import file2
>>> print(__name__)
>>> print(file2.__name__)
\end{verbatim}
and the result is 
\begin{verbatim}
__main__
file2
\end{verbatim}

If we write the code in \verb|file2|
\begin{verbatim}
>>> print(__name__)
\end{verbatim}
Executing the \verb|file1| now gives
\begin{verbatim}
file2
__main__
file2
\end{verbatim}
that is, the code executes \verb|file2| and then executes \verb|file2|. We don't want to execute \verb|file2|, we just want to use some of the predefined functions, so we can avoid this with the following line in \verb|file2|
\begin{verbatim}
>>> if __name__ == "__main__":
>>>     print(__name__)
\end{verbatim}

\subsection{External modules}

We install packages with pip. We can search for packages here in the website \href{https://pypi.org/search/}{pypi.org}. The syntax is 
\begin{verbatim}
> python3 -m pip install NAME_OF_THE_PACKAGE
\end{verbatim}

As an interesting example, use the module \verb|termcolor|
\begin{verbatim}
> from termcolor import colored
> print(dir(termcolor)) # it gives the attributes of the package
> help(termcolor) # it explains the package
> print(colored("Hi THERE!", color="magenta", on_color="on_cyan", attrs=["blink"]))
\end{verbatim}