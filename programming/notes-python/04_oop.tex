%%%%%%%%%%%%%%%%%%%%%%%%%%%%%%%%%%%%%%%%
%%%%%%%%%%%%%%%%%%%%%%%%%%%%%%%%%%%%%%%%
%%%%%%%%%%%%%%%%%%%%%%%%%%%%%%%%%%%%%%%%
%%%%%%%%%%%%%%%%%%%%%%%%%%%%%%%%%%%%%%%%

\section{Object Oriented Programming}

Object Oriented Programming (OOP) is about creating representation of things in the real world using codes. We do it using {\bf classes} and {\bf objects}.

\paragraph{\(\#\) \bf class} Blueprints for objects. Classes can contain methods (functions) and attributes (similar to keys in a dictionary). 

\paragraph{ \(\#\) \bf instance} - objects that are constructed from a class blueprint that contain their class's methods and properties.

\subsection{Definitions}

With OOP we can classify and organize our codes. 

\subsubsection{Example} Suppose we want to make a poker game. We could have the following entities: \(\bullet\) Game, \(\bullet\) Player, \(\bullet\) Card, \(\bullet\) Deck, \(\bullet\) Hand, \(\bullet\) Chip, \(\bullet\) Bet and so on. We could hard code all these entities, but it is oftentimes convenient to define classes and work with methods associated to these classes. 

I can consider a different explanation, aimed for mathematicians. Suppose I want to work with matrices. We could hard code these objects and hard code all operations associated to them, but it is better to define the class of matrices, and methods associated to them. These methods can be the transposition, trace and so on. 

\subsubsection{Defining a class}

Suppose we want to create a game. We want to build a class for users. The syntax is the following:
\begin{verbatim}
> class User: # Classes are oftentimes capitalized
>    def __init__(self, first, last, age):
>        self.name = first
>        self.last = last
>        self.age = age
\end{verbatim} 

We always use the syntax \verb|__init__| in the definition of the class. The \verb|self| is a dummy variable, but it is traditionally written as \verb|self| and it reffers to the objects (instances) themselves. Python calls the \verb|__init__| method whenever we create an instance of class (instatiate).

Now we can build these users using the methods, that is 
\begin{verbatim}
> user1 = User('Joe', 'Smith', 68)
> user2 = User('Blanka', 'Lopez', 41)
\end{verbatim}

And finally, we can retried the information of the users using the methods associated to them. For example, the code
\begin{verbatim}
> print(user1.first, user1.last)
> print(user2.first, user2.last)
\end{verbatim}
prints the names. Observe that the method does not use the brackets \verb|()|.

Another example is the following. Consider the class \verb|Comments| in a social network. That is 
\begin{verbatim}
> class Comments:
>      def __init__(self, username, text, likes=0) # likes default value is 0. 
>      self.username = username	
>      self.text = text
>      self.likes = likes
\end{verbatim}
Then, we run this code as
\begin{verbatim}
> c = Comment("davey123", 'lol you\'re so silly', 3)
> print(c.username)
> print(c.text)
> print(c.likes)
\end{verbatim}

\begin{shaded}
Here I need to comment something important: The underscores have meaning. In particular, we have things in the form: 
\begin{itemize}
	\item \verb|_name| : just one underscore in front of a method is a message to other developers. It says that this method whould be private, although python does not have full fledged secrete methods. It is basically a method to be used inside the definition of the class. 
	\item \verb|__name__|: These are special methods of python. Leave them alone. 
	\item \verb|__name| : Name mangling. This is something I will eventually learn somewhere. Google name mangling to understant a bit before any formal definition.
\end{itemize}
\end{shaded}	



\subsubsection{Instance Methods}

Here I want to consider some instance methods. It basically defines the methods inside these classes, and they act on the objects we define. 

In this case, we use them with the syntax
\begin{verbatim}
> object.my_method(argument)
\end{verbatim}

For example, let us keep with the \verb|User| class we defined above. Let us define some methods. 
\begin{verbatim}
> class User: # Classes are oftentimes capitalized
>    def __init__(self, first, last, age):
>        self.name = first
>        self.last = last
>        self.age = age
>
>    def full_name(self): # full_name method
>        return f"{self.first} {self.last}"
>
>    def initials(self): # initials method
>        return f"{self.first[0]} {self.last[0]}"
>
>    def likes(self, thing): # likes method. This methods has an argument
>        return f"{self.first} likes {thing}"
>
>    def is_senior(self): # Is senhor method has conditionals
>        return self.age >= 65    
>
>    def birthday(self):
>        self.age +=1
>        return f"Happy {self.age}th, {self.first}"
\end{verbatim}

Now, we can run these methods. First we need to define the users. 
\begin{verbatim}
> user1 = User("Joe", "Smith", 68)        
> user2 = User("Blanka", "Lopez", 41)
\end{verbatim}
Then, 
\begin{verbatim}
> print(user2.full_name())
> print(user1.likes("Ice Cream"))
>
> print(user1.initials())
> print(user2.initials())
>
> print(user2.is_senior())
> print(user1.age)
> print(user1.birthday())
> print(user1.age)	
\end{verbatim}

Another example is the bank account. See
\begin{verbatim}
> class BankAccount:
>     def __init__(self, owner, balance=0.0):
>         self.owner = owner
>         self.balance = balance
> 	
>     def deposit(self, add):
>         self.balance += add
>         return self.balance 
>
>     def withdraw(self, rem):
>         self.balance -= rem
>         return self.balance
> 
> acct = BankAccount("Darcy")
> 
> print(acct.owner)
> print(acct.balance)
> print(acct.deposit(10))
> print(acct.withdraw(3))
> print(acct.balance)	
\end{verbatim}


%%%%%%%%%%%%%%%%%%%%%%%%%%%%%%%%%%%%%%%%%%
%%%%%%%%%%%%%%%%%%%%%%%%%%%%%%%%%%%%%%%%%%

\subsubsection{Class attributes}

We now want to define attributes for the classes themselves, and not only for the instances. Let's come back to the User example. 
\begin{listing}{1}
class User:

    active_users = 0 # We define a class attribute

    def __init__(self,first,last,age):
        self.first = first
        self.last = last
        self.age = age
        User.active_users += 1

    def logout(self): 
        User.active_users -= 1 
        return f"{self.first} has logged out"  

    def full_name(self):
        return f"{self.first} {self.last}"

    def initials(self):
        return f"{self.first[0]}.{self.last[0]}."

    def likes(self, thing):
        return f"{self.first} likes {thing}"

    def is_senior(self):
        return self.age >= 65    

    def birthday(self):
        self.age +=1
        return f"Happy {self.age}th, {self.first}"
\end{listing}

In the case above, \verb|active|users| is a class attribute. Whenever a new user logs in or logs out we need to update the value of this attribute. We can test the code above with 

\begin{listingcont}{1}
print(User.active_users)

user1 = User("Joe", "Smith", 68)        
user2 = User("Blanka", "Lopez", 41)

print(User.active_users)

print(user2.logout())

print(User.active_users)
\end{listingcont}

Class attributes can be used as validation. For example, suppose we have a class for pets, and we want to forbid certain animals, for example, an alligator as a pet, then we can do the following:
\begin{listing}{1}
class Pet:

allowed = ['cat', 'dog', 'fish', 'rat']

def __init__(self, name, species):
	if species not in Pet.allowed:
		raise ValueError(f"You can't have a {species} pet")
	self.name = name
	self.species = species

def set_species(self, species):
	if species not in Pet.allowed:
		raise ValueError(f"You can't have a {species} pet")
	self.species = species	
\end{listing}

The second method, \verb|set_species| changes the species of our pet, but within the allowed class. If we deactivate the conditional part, we could change it the species of our pet to an animal that is not in the list of allowed pets. 

\begin{listingcont}
tonny = Pet("Tonny", "cat")

print(Pet.allowed)
print(tonny.allowed)

print(tonny.species)
new_species = tonny.set_species('dog')
print(tonny.species)

print(tonny.allowed == Pet.allowed)	
\end{listingcont}

Another example is the following:
\begin{listing}{1}
class Chicken:

    species = ['Partridge Silkie', 'Speckled Sussex']
    total_eggs = 0

    def __init__(self, name, species, eggs=0):
        # if species not in Chicken.allowed:
        #     raise ValueError(f"This is not a known {species}")
        self.name = name
        self.species = species
        self.eggs = eggs

    def lay_eggs(self):
        self.eggs += 1
        Chicken.total_eggs += 1

c1 = Chicken("Alice", 'Partridge Silkie')
c2 = Chicken("Amelia", 'Speckled Sussex') 

print(Chicken.total_eggs)
c1.lay_eggs()
print(Chicken.total_eggs)
c1.lay_eggs()
c1.lay_eggs()
print(Chicken.total_eggs)	
\end{listing}

%%%%%%%%%%%%%%%%%%%%%%%%%%%%%%%%%%%%%%%%%%
%%%%%%%%%%%%%%%%%%%%%%%%%%%%%%%%%%%%%%%%%%

\subsubsection{Class methods}

Now we want to define methods which are concerned with the class as as whole, and not with the instances themselves. We have encountered class methods before. The \verb|fromkeys| method is an example. 

Let us return to our old friend class of Users. Then 
\begin{listing}{1}
class User: 

    active_users = 0 # We define a class attribute

    @classmethod
    def display_active_users(cls):
        return f"There are currently {cls.active_users} active users."

    @classmethod
    def from_string(cls, data_str):
        first,last,age = data_str.split(",")        
        return cls(first,last,int(age))

    def __init__(self,first,last,age):
        self.first = first
        self.last = last
        self.age = age
        User.active_users += 1
		
	def __repr__(self):
		return f"{self.first}"	

    def logout(self): 
        User.active_users -= 1 
        return f"{self.first} has logged out"  

...	
\end{listing}
where we have omitted the unnecessary methods. The class methods are defined with \verb|@classmethod|. We have also defined an instance method to logout users and method with some mysterious \verb|__repr__| that we will explain below. Then, run the following code 
\begin{listingcont}
user1 = User("Joe", "Smith", 68)        
user2 = User("Blanka", "Lopez", 41)
print(User.display_active_users())
print(user2.logout())
print(User.display_active_users())	
\end{listingcont}
We see that initially there were 2 active users, then one logged out, and remains just 1 active user. 

The second class method is more interesting. Suppose we have a string that is in the format of comma separated values (csv) and we want to build a new user from it. We basically have to separate this string in 3 entries, the first name, last name and age. This method does it for us. Run 
\begin{listingcont}
tom = User.from_string("Tom, Jones, 89")
print(tom.first) 
print(tom.full_name(), ",", tom.age, "years old")
\end{listingcont}

There is one final piece of information in the code above we need to know. The \verb|__repr__| method is quite useful if we want to have some control over the output. First, comment this part of the code and try to run the following code
\begin{listingcont}
print(tom)
\end{listingcont}
we get something of the form \verb|<__main__.User object at XXX>|. On the other hand, we use the \verb|__repr__| method to control this outcome. In that case we obtain the first name as an output. 


\subsubsection{Exercise: Cards \& Deck Classes}

Here I want to consider the problem of section \# 25 of Colt's course. It is basically the definition of two classes with the following properties:

\textbf{-- Card class}
\begin{itemize}
    \item Each instance of Card  should have a suit ("Hearts", "Diamonds", "Clubs", or "Spades").
    \item Each instance of Card  should have a value ("A", "2", "3", "4", "5", "6", "7", "8", "9", "10", "J", "Q", "K").
    \item Card 's \verb|__repr__| method should return the card's value and suit (e.g. "A of Clubs", "J of Diamonds", etc.)
\end{itemize}


\textbf{-- Deck class}
\begin{itemize}
    \item Each instance of Deck should have a cards attribute with all 52 possible instances of Card.
    \item Deck  should have an instance method called count  which returns a count of how many cards remain in the deck.
    \item Deck 's \verb|__repr__| method should return information on how many cards are in the deck (e.g. "Deck of 52 cards", "Deck of 12 cards", etc.)
    \item Deck  should have an instance method called \verb|_deal|  which accepts a number and removes at most that many cards from the end of the deck (it may need to remove fewer if you request more cards than are currently in the deck!). If there are no cards left, this method should return a ValueError  with the message "All cards have been dealt".
    \item Deck  should have an instance method called shuffle  which will shuffle a full deck of cards. If there are cards missing from the deck, this method should raise a \verb|ValueError| with the message "Only full decks can be shuffled". shuffle should return the shuffled deck.
    \item Deck  should have an instance method called \verb|deal_card| which uses the \verb|_deal| method to deal a single card from the deck and return that single card.
    \item Deck  should have an instance method called \verb|deal_hand| which accepts a number and uses the \verb|_deal| method to deal a list of cards from the deck and return that list of cards.
\end{itemize}

\paragraph{My Solution}

Here is my solution
\begin{listing}{1}
from random import shuffle

class Card:
    allowed_suits = ('Hearts', 'Diamonds', 'Clubs', 'Spades')
    allowed_values =  ('A', '2', '3', '4', '5', '6', 
    '7', '8', '9', '10', 'J', 'Q', 'K')

    def __init__(self, value, suit):
        if value not in Card.allowed_values:            
            raise ValueError(f"Value needs to be 'A', '2', '3', '4', '5', 
            '6', '7', '8', '9', '10', 'J', 'Q', 'K'")
        if suit not in Card.allowed_suits:
            raise ValueError(f"Suit needs to be 'Hearts', 
            'Diamonds', 'Clubs' or 'Spades'")
        self.value = value
        self.suit = suit

    def __repr__(self):
        return f'{self.value} of {self.suit}'      
\end{listing}
and 
\begin{listingcont}
class Deck:
    def __init__(self):
        allowed_suits = ('Hearts', 'Diamonds', 'Clubs', 'Spades')
        allowed_values =  ('A', '2', '3', '4', '5', '6', '7', '8', '9',
        '10', 'J', 'Q', 'K')        
        self.cards = [Card(v,s) for v in allowed_values for s in allowed_suits]

    def __repr__(self):
        return f"Deck of {self.count()} cards"
        # I should favor the .format() format.
                    
    def count(self):
        return len(self.cards)

    def _deal(self, num):        
        cards = self.cards 
        hand = []
        count = 0
        if len(cards) == 0:
            raise ValueError("All cards have been dealt")
        while count < num:
            hand.append(cards.pop(-1))
            if len(cards) == 0:
                break
            count +=1
        return hand

    def shuffle(self):    
        if self.count() < 52:
            raise ValueError("Only full decks can be shuffled")

        shuffle(self.cards)
        return self            

    def deal_card(self):
        return self._deal(1)[0]

    def deal_hand(self, n):
        return self._deal(n)    
\end{listingcont}

I can also execute some tests
\begin{listingcont}

card = Card('A', 'Spades')
print(card)

my_deck = Deck()
print(my_deck.cards)
print(10 * '*')
my_deck.shuffle()
print(my_deck.cards)

card = my_deck.deal_card()
print(card)

print(my_deck.deal_hand(50))
print(my_deck.count())
print(my_deck.deal_hand(5))
print(my_deck.deal_hand(10))    
\end{listingcont}


\paragraph{Colt's solution}

Here I should compare with Colt's solution. 
\begin{listing}{1}
from random import shuffle

class Card:
    def __init__(self, value, suit):
        self.value = value
        self.suit = suit

    def __repr__(self):
        # return "{} of {}".format(self.value, self.suit)
        return f"{self.value} of {self.suit}"    
\end{listing}
and 
\begin{listingcont}
class Deck:
	def __init__(self):
		suits = ["Hearts", "Diamonds", "Clubs", "Spades"]
		values = ['A','2','3','4','5','6','7','8','9','10','J','Q','K']
		self.cards = [Card(value, suit) for suit in suits for value in values]

	def __repr__(self):
		return f"Deck of {self.count()} cards"

	def count(self):
		return len(self.cards)

	def _deal(self, num):
		count = self.count()
		actual = min([count,num])
		if count == 0:
			raise ValueError("All cards have been dealt")
		cards = self.cards[-actual:]
		self.cards = self.cards[:-actual]
		return cards

	def deal_card(self):
		return self._deal(1)[0]

	def deal_hand(self, hand_size):
		return self._deal(hand_size)

	def shuffle(self):
		if self.count() < 52:
			raise ValueError("Only full decks can be shuffled")

		shuffle(self.cards)
		return self    
\end{listingcont}
with tests
\begin{listingcont}

d = Deck()
d.shuffle()
card = d.deal_card()
print(card)
hand = d.deal_hand(50)
card2 = d.deal_card()
print(card2)
print(d.cards)
card2 = d.deal_card()

# print(d.cards)
\end{listingcont}



\subsection{Inheritance}

Suppose we want to build users with different functionalities. For example, ordinary users and moderators. These two share a lot in common, but moderators have additional functionalities. We could define two distinct classes, but it is easier to use this idea of \emph{inheritance}.

As a simpler example 
\begin{listing}{1}
class Animal:
    cool = True # This is a class attribute

    def make_sound(self, sound):
        print(sound)
    
class Cat(Animal):
    pass 

gandalf = Cat()
gandalf.make_sound
gandalf.cool
\end{listing}
I could use the \verb|isinstance(instance, object)| function that verifies that the instance belongs to both classes. 
\begin{listingcont}
print(isinstance(gandalf, Animal))
> True
print(isinstance(gandalf, Cat))
> True
print(isinstance(gandalf, list))
> False
\end{listingcont}


\subsubsection{Properties}

Consider the class Human:
\begin{listing}{1}
class Human: 
    def __init__(self, first, last, age):
        self.first = first 
        self.last = last
        if age >= 0:
            self.age = age
        else:
            self.age = 0
\end{listing}
where the conditionals avoid, for example, negative ages. The solution above does not prevent, on the other hand, that we change the age afterwords, for example if we replace \verb|jane.age = -100|. 
\begin{listingcont}
jane = Human("Jane", "Goodall", 50)
print(jane.age)
jane.age = -100
print(jane.age)
\end{listingcont}
Then, we need to define some properties. First of all, we need to do some modifications
\begin{listing}{1}
class Human: 
    def __init__(self, first, last, age):
        self.first = first 
        self.last = last
        if _age >= 0:
            self._age = age
        else:
            self.age = 0

# Not the best way of doing that:

#       def get_age(self):
#          return self._age
#       def set_age(self, new_age):
#           if _age >= 0:
#               self._age = age
#           else:
#               self.age = 0
 
# The best way of doing that:

    @property
    def age(self):
        return self._age

    @age.setter 
    def age(self, value):
        if value >=0:
            self._age =value
        else:
            raise ValueError("age can't be negative")
\end{listing}
then we run
\begin{listingcont}
jane = Human("Jane", "Goodall", 50)
print(jane.age)
\end{listingcont}

\subsubsection{Introduction to Super()}

The Cat class that we have considered above is useless as it was. We want to give more functionalities to it. 
\begin{listing}{1}
    class Animal:
    def __init__(self, name, species):
        self.name = name 
        self.species = species

    def __repr__(self):
        return f"{self.name} is a {self.species}"
        
    def make_sound(self, sound):
        print(sound)
    
class Cat(Animal):
    def __init__(self, name, breed, toy): 
# Repetition we want to avoid:
#       self.name = name 
#       self.species = "Cat" 
# This is one possibility to avoid repetition:
#       Animal.__init__(self, name, species="Cat") 
# This is the pythonic way:
        super().__init__(name, species="Cat")
        self.breed = breed
        self.toy = toy

        def play(self):
            print(f"{self.name} plays with {self.toy}")

gandalf = Cat("Gandalf", "Cat", "Scottish Fold", "String")
print(gandalf)
gandalf.play() # Cat method
gandalf.make_sound("Meow") # Animal method
\end{listing}

\subsubsection{Example of Inheritance: Users \& Moderators}

Let me come back to the original \verb|Users| class:
\begin{listing}{1}
class User: 

    active_users = 0 # We define a class attribute

    @classmethod
    def display_active_users(cls):
        return f"There are currently {cls.active_users} active users."

    @classmethod
    def from_string(cls, data_str):
        first,last,age = data_str.split(",")        
        return cls(first,last,int(age))

    def __init__(self,first,last,age):
        self.first = first
        self.last = last
        self.age = age
        User.active_users += 1
        
    def __repr__(self):
        return f"{self.first}"	

    def logout(self): 
        User.active_users -= 1 
        return f"{self.first} has logged out"  

    def full_name(self): # full_name method
        return f"{self.first} {self.last}"

    def initials(self): # initials method
        return f"{self.first[0]} {self.last[0]}"

    def likes(self, thing): # likes method. This methods has an argument
        return f"{self.first} likes {thing}"

    def is_senior(self): # Is senhor method has conditionals
        return self.age >= 65    

    def birthday(self):
        self.age +=1
        return f"Happy {self.age}th, {self.first}"
\end{listing}

We can now define the class of \verb|Moderators| and print some results:
\begin{listingcont}
class Moderator(User):
    total_mods = 0
    def __init__(self, first, last, age, community):
        super().__init__(first, last, age)
        self.community = community
        Moderator.total_mods += 1

    @classmethod
    def display_active_mods(cls):
        return f"There are currently {cls.total_mods} active mods."

    def remove_post(self):
        return f"{self.full__name()} removed a 
        post from the {self.community} community"

`user1 = User('Tom', 'Morello', 57)
user2 = User('Adam', 'Jones', 57)
user3 = User('Danny', 'Carey', 60)
jasmine = Moderator('Thiago', "Araujo", 34, 'Bands')      
jasmine = Moderator('Aline', "Lima", 35, 'Piano')      
print(User.display_active_users())
print(Moderator.display_active_mods())'  
\end{listingcont}


\subsubsection{Multiple Inheritance}

It is not recommended to use multiple inheritance. It is better to organize the ideas a bit better. Consider the example
\begin{listing}{1}
class Aquatic:
    def __init__(self, name):
        self.name = name 

    def swim(self):
        return f"{self.name} is swimming"

    def greet(self):
        return f"I am {self.name} of the sea!"        

class Ambulatory:
    def __init__(self, name):
        self.name = name 

    def walk(self):
        return f"{self.name} is walking"

    def greet(self):
        return f"I am {self.name} of the land"

class Penguin(Ambulatory, Aquatic):
    def __init__(self, name):
        # super().__init__(name = name) In this particular case, 
        #it is better to be explicit
        Ambulatory.__init__(self, name=name)
        Aquatic.__init__(self, name=name)
\end{listing}
and we can try the code 
\begin{listingcont}
jaws = Aquatic('Jaws')
lassie = Ambulatory('Lassie')
captain_cook = Penguin('Captain Cook')

print(captain_cook.swim())
print(captain_cook.walk())    
\end{listingcont}

\subsubsection{Method Resolution Order (MRO)}

Whenever we create a class, Python sets a MRO for that class which is the order in which Python will loop for methods on instances of that class.

We can see it in three ways
\begin{itemize}
    \item \verb|__mro__| attribute on the class 
    \item Use the method \verb|mro()| on the class 
    \item Use the builtin \verb|help(cls)| method
\end{itemize}

Using the Penguin example above, we can run
\begin{listing}{1}
Penguin.__mro__    
Penguin.mro()
help(Penguin)
\end{listing}

Another example
\begin{shaded}
\begin{listing}{1}
class A:
    def do_something(self):
        print("Method defined in: A")    

class B(A):
    def do_something(self):
        print("Method defined in: B")        

class C(A):
    def do_something(self):
        print("Method defined in: C")

class D(B, C):
    def do_something(self):
        print("Method defined in: D")        

thing = D()        
thing.do_something()

help(D)
\end{listing}    
The methods are inherited from \verb|D|, \verb|B|, \verb|C|, \verb|A|. We can comment the methods to see as it happens.
\end{shaded}


\subsection{Polymorphism}

There are two types of polymorphism.

\subsubsection{Polymorphism \& Inheritance}

\# 1. When the class method works in a similar way for different classes. For example, when we have a method in a parent class and that is overridden by a subclass. This is called \emph{method overriding}. For example:
\begin{listing}{1}
class Animal():
    def speak(self):
        raise NotImplementedError('Subclass needs to implement this method'
        
class Dog(Animal):
    def speak(self):
        return 'woof'        

class Cat(Animal):
    def speak(self):
        return 'meow'  
\end{listing}
It is an example of polymorphism.

\subsubsection{Special methods}

2. It also is when the same operation works for different kinds of objects. The \verb|len| method for example. For Example
\begin{listing}{1}
8 + 2
> 10
'8' + '2'
> '82'    
\end{listing}
In the first case, it is a sum of integers, and in the second example it is a concatenation of strings. 

\paragraph{Example}

Let us now define our own special method. These are examples of dunder methods: \verb|__something__|, such as \verb|__init__|, \verb|__repr__|, \verb|__len__| and so forth.
\begin{listing}{1}
from copy import copy

class Human:
    def __init__(self, first, last, age):
        self.first = first
        self.last = last
        self.age = age

    def __repr__(self):
        return f"Human named {self.first} {self.last}"

    def __len__(self):
        return self.age      

    def __add__(self, other):
        if isinstance(other, Human):
            return Human(first = 'Newborn', last = self.last, age = 0)
        return "You can't add that!"

    def __mul__(self, other):
        if isinstance(other, int):
            return [copy(self) for i in range(other)]
        return "Can't multiply"

j = Human('James', 'T. Kirk', 35) 
k = Human('Nyota', 'Uhura', 30) 
print(j)
print(len(j))

print(j + k)
print(j * 2)
\end{listing}
