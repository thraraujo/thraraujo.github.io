\section{Github}

We now start a discussion of one of the big topics: Github. It is host platform for git repos. There
are important reasons to use Github, for example, backup our project, improve our collaborations
and be part of the open source community.

\subsection{Cloning repos}

With clone, we do not need to create all our project from scratch, we can close an existing repo. 
All we need is the URL of a repo we want to clone. The command is 
\begin{listing}{1}
> git clone <url>
\end{listing}
Observe that anyone can close a repository as long as it is public. Moreover, this command is not 
tied to Github.

\subsection{Github setup}

The first thing we need to do is to generate the SSH keys. See details 
here~\href{https://docs.github.com/en/authentication/connecting-to-github-with-ssh}{SSH}. The 
whole process is a bit tedious, but is not difficult. 

\subsection{Creating Github repos}

Now we want to create new Github respos. There are two situations:

\subsubsection{Uploading repos} If we already have repos in our machine, and we want to upload them
to Github, we do the following:
\begin{enumerate}
    \item Create a new github repo 
    \item Connect a local repo (add a remote)
    \item Push up our changes
\end{enumerate}

\subsubsection{Start from scratch} The other option is when we want to start a brand new project.
In this case, we do the following:
\begin{enumerate}
    \item Create a new github repo
    \item Clone it to our machine
    \item Do some work locally.
    \item Push up our changes
\end{enumerate}
This second option is easy if we understand the first. 

\subsubsection{Git remote}

Once we create a github repo, we need to tell git about the existence of the github project.
In order to view the existing remotes in my computer, we use the following command
\begin{listing}{1}
> git remote -v 
\end{listing}
where \verb|-v| means verbose. If the remove does not exist, we can create a new one with
\begin{listing}{1}
> git remote add <name> <url>    
\end{listing}
A standard name is \verb|origin|. We can change the name of a remote or delete it with 
\begin{listing}{1}
git remote rename <old_name> <new_name>
git remote remove <name>
\end{listing}

\subsubsection{Git push}

New we want to push our modifications. We do it with the command
\begin{listing}{1}
> git push <remote> <branch>
\end{listing}

Actually, we do not need to push the branch we have locally to the same branch on github, we can 
even change names. We can do it with the code
\begin{listing}{1}
> git push <remote> <local_branch>:<remote_branch>
\end{listing}

Finally, we want to pay attention to the flag \verb|-u| below
\begin{listing}{1}
git push -u <remote> <branch>
\end{listing}
What does this flag mean? It stands for upstream. It basically links the github to my computer, 
so it makes our life easier. In this case, we can push our modifications as
\begin{listing}{1}
git push 
\end{listing}

\subsubsection{Master \& Main}

In order to be in accordance with new decisions of Github, we can use the terminology \verb|main| 
instead \verb|master|. So we can change the name of our repos with 
\begin{listing}{1}
git branch -M main
\end{listing}

\subsubsection{Forcing push}

If we amend a commit and try to push this modification to github, then github 
rejects my modifications. This is a safety measure that github imposes to us. 
The solution is to force the push with 
\begin{listing}
git push -f
\end{listing}
This will obliterate the previous commit (the one we are amending), so if 
someone is working on it, it will cause big problems. One way to avoid it 
is to have personal branches if we are working in a team. If it is a solo 
work, it should be fine. 




